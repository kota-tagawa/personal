\documentclass[]{jarticle}          % 一段組
%\documentclass[twocolumn]{jarticle} % 二段組

\textwidth 180mm
\textheight 255mm
\oddsidemargin -12mm
\topmargin -15mm
\columnsep 10mm

%\vspace{0.5cm} % 一段組の場合はコメントアウトした方が体裁がよいx
%] % 一段組の場合はコメントアウトする

\usepackage{styles/labheadings}
\usepackage[dvipdfmx]{graphicx,color}
\usepackage{amsmath,amssymb}
\usepackage{url}
% 追加
\usepackage{listings,jvlisting} 
\usepackage[hang,small,bf]{caption}
\usepackage[subrefformat=parens]{subcaption}
\captionsetup{compatibility=false}

\input{numerical_definition.tex}
% report.texと同じディレクトリにnumerical_definition.texを入れておけば上の書き方でもいいはずです

\usepackage[
  dvipdfm,
  bookmarks=true,
  bookmarksnumbered=true,
  colorlinks=true]{hyperref}
\AtBeginDvi{\special{pdf:tounicode EUC-UCS2}}

%ここからソースコードの表示に関する設定
\lstset{
  basicstyle={\ttfamily},
  identifierstyle={\small},
  commentstyle={\smallitshape},
  keywordstyle={\small\bfseries},
  ndkeywordstyle={\small},
  stringstyle={\small\ttfamily},
  frame={tb},
  breaklines=true,
  columns=[l]{fullflexible},
  numbers=left,
  xrightmargin=0zw,
  xleftmargin=3zw,
  numberstyle={\scriptsize},
  stepnumber=1,
  numbersep=1zw,
  lineskip=-0.5ex
}
%ここまでソースコードの表示に関する設定

\pagestyle{labheadings}
\headerleft{研究計画}   % ヘッダの左側のタイトル
\headerright{2023年05月26日}  % ヘッダの右側のタイトル

\begin{document}

%\twocolumn % 一段組の場合はコメントアウトする

\vspace*{2ex}
\begin{center}
 {\Large \bf 研究計画}\\ % タイトル
 \vspace*{5mm}
 {\large B4 田川幸汰}% 発表者名
\end{center}

%\vspace{0.5cm} % 一段組の場合はコメントアウトした方が体裁がよいx
%] % 一段組の場合はコメントアウトする

%新しく作成したコマンド
% \newcommand{\reffig}[1]{\hyperref[#1]{図\ref{#1}}}
% \newcommand{\refeq}[1]{\hyperref[#1]{式(\ref{#1})}}
% \newcommand{\reftab}[1]{\hyperref[#1]{表\ref{#1}}}
% \newcommand{\refsec}[1]{\hyperref[#1]{\ref{#1}章}}
% \newcommand{\refsubsec}[1]{\hyperref[#1]{\ref{#1}節}}

\section{概要}
 B4年度の研究計画を発表する。 \\
 研究の目的と、目的を達成するために解決すべき要素技術、現時点で考えている解決のアプローチを記述する。また、卒論発表までの全体的な計画と、前期分の月単位の詳細な計画についてもまとめる。

\section{研究目的}
マスクあり顔画像から、マスクなし顔画像を生成する。対象者の標準三次元モデルを作成し、マスク有り顔画像から検出された顔の特徴点に作成した三次元モデルを描画することで、マスクなし画像を再現する。三次元モデルを用いることで、
マスクあり顔画像からマスクなし顔画像を生成する従来の手法を用いた場合と、生成画像のクオリティやコストを比較する。\hyperref[graphone]{図\ref{graphone}}は研究の大まかな概要図である。
% 図を作る
\begin{figure}[!ht]
  \begin{center}
    \includegraphics[scale=0.5]{figures/1.jpg}
    \caption{概要図}
    \label{graphone}
  \end{center}
\end{figure}

\section{解決すべき要素技術}
以下に本研究で解決すべき要素技術をまとめる。
\begin{enumerate}
  \item マスク有り画像からの特徴点の抽出
  \item 抽出された特徴点に対して、標準三次元モデルを描画
  \item 1.2の技術を用いて、マスクあり顔画像からマスクなし顔画像を生成
\end{enumerate}

\section{解決のアプローチ}
前節で列挙した要素技術に対し、それぞれどのようなアプローチで解決するかまとめる。
\subsection{特徴点の抽出}
google MediaPipeを用いて、顔の三次元特徴点を抽出する方法を検討している。
\subsubsection{google MediaPipe}
MediaPipeは、Googleが開発したオープンソースのフレームワークであり、コンピュータビジョンや機械学習を用いたリアルタイムのメディア処理を容易に行うためのツールセットである。
カメラやビデオ入力からのリアルタイムの動画処理や解析に用いられ、リアルタイム性能が高いこと、マルチプラットフォームであること、TensorFlowやTFLiteなどの機械学習フレームワークとの統合が容易であることなどが利点である。
\subsection{標準三次元モデルへの描画}
現時点では解決のアプローチは定かではないが、三次元モデルと抽出した特徴点がフィットするようにモデルを変更することが求められる。 \\
また、より自然に特徴点に三次元モデルを描画するために、二次元画像の場合と同じように\cite{bib_1}機械学習の手法を取り入れることについても検討していきたい。

\section{卒論発表までの全体的な計画}
\begin{itemize}
  \item 6月末まで...参考研究及び資料の調査、標準三次元モデルの作成
  \item 7月末まで...google MediaPipeを用いた三次元特徴量の抽出
  \item 夏休み期間...標準三次元モデルと三次元特徴量をマージして、マスクなし顔画像を再現
  \item 10月末まで...マスクなし顔画像の見えを調整し、これまでの手法と比較
  \item 11月...卒業論文の書き始め
  \item 12月...発表スライドの作成
\end{itemize}

%参考文献
\begin{thebibliography}{99}
\bibitem{bib_1} 小池 泰景,「同一人物の正面の顔を用いた StyleGAN2 による顔のマスク領域の補完」, 情報処理学会第84回全国大会, 2-279~2-280.
\end{thebibliography}

\end{document}
