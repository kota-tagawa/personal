\documentclass[]{jarticle}          % 一段組
%\documentclass[twocolumn]{jarticle} % 二段組

\textwidth 180mm
\textheight 255mm
\oddsidemargin -12mm
\topmargin -15mm
\columnsep 10mm

%\vspace{0.5cm} % 一段組の場合はコメントアウトした方が体裁がよいx
%] % 一段組の場合はコメントアウトする

\usepackage{styles/labheadings}
\usepackage[dvipdfmx]{graphicx,color}
\usepackage{amsmath,amssymb}
\usepackage{url}
% 追加
\usepackage{listings,jvlisting}
\usepackage[hang,small,bf]{caption}
\usepackage[subrefformat=parens]{subcaption}
\captionsetup{compatibility=false}

\input{numerical_definition.tex}
% report.texと同じディレクトリにnumerical_definition.texを入れておけば上の書き方でもいいはずです

\usepackage[
  dvipdfm,
  bookmarks=true,
  bookmarksnumbered=true,
  colorlinks=true]{hyperref}
\AtBeginDvi{\special{pdf:tounicode EUC-UCS2}}

%ここからソースコードの表示に関する設定
\lstset{
  basicstyle={\ttfamily},
  identifierstyle={\small},
  commentstyle={\smallitshape},
  keywordstyle={\small\bfseries},
  ndkeywordstyle={\small},
  stringstyle={\small\ttfamily},
  frame={tb},
  breaklines=true,
  columns=[l]{fullflexible},
  numbers=left,
  xrightmargin=0zw,
  xleftmargin=3zw,
  numberstyle={\scriptsize},
  stepnumber=1,
  numbersep=1zw,
  lineskip=-0.5ex
}
%ここまでソースコードの表示に関する設定

\pagestyle{labheadings}
\headerleft{研究計画}   % ヘッダの左側のタイトル
\headerright{2024年4月17日}  % ヘッダの右側のタイトル

\begin{document}

%\twocolumn % 一段組の場合はコメントアウトする

\vspace*{2ex}
\begin{center}
 {\Large \bf 今年度の研究計画}\\ % タイトル
 \vspace*{5mm}
 {\large M1 田川幸汰}% 発表者名
\end{center}

%\vspace{0.5cm} % 一段組の場合はコメントアウトした方が体裁がよいx
%] % 一段組の場合はコメントアウトする

%新しく作成したコマンド
% \newcommand{\reffig}[1]{\hyperref[#1]{図\ref{#1}}}
% \newcommand{\refeq}[1]{\hyperref[#1]{式(\ref{#1})}}
% \newcommand{\reftab}[1]{\hyperref[#1]{表\ref{#1}}}
% \newcommand{\refsec}[1]{\hyperref[#1]{\ref{#1}章}}
% \newcommand{\refsubsec}[1]{\hyperref[#1]{\ref{#1}節}}

% 数式
%\begin{equation}
%  数式記述  
%  \label{ラベル名}
%\end{equation}

% 図
% \begin{figure}[!ht]
%   \begin{center}
%     \includegraphics[scale=0.5]{figures/画像ファイル名}
%     \caption{キャプション名}
%     \label{ラベル名}
%   \end{center}
% \end{figure}

% リスト
% \begin{enumerate or itemize}
%   \item 
% \end{enumerate or itemize}

\section{概要}
はじめに、前年度の研究の概要について軽く説明する。次に今年度の研究について、現時点での研究のアプローチと以降の研究計画についてまとめる。

\section{前年度の研究の概要}

\section{今年度の研究}
\subsection{研究概要}
今年度は全方位カメラを用いてテクスチャを作成し、3次元モデルにテクスチャを割り当てる手法について研究する。
使用用途として、ショッピングモール内のモデルを作る際に利用することを想定している。
また、現時点では以下のような手順でモデルを生成することを考えている。
\begin{enumerate}
  \item ショッピングモールのフロアマップから3次元モデルを生成。
  \item ショッピングモール内を全方位カメラで撮影し、そこから透視投影画像を生成。
  \item 透視投影画像から得られる点、線特徴量からカメラ姿勢を推定する。
  \item 3次元モデルに対して正しい大きさ、位置、方向でテクスチャ画像を割り当てる。
\end{enumerate}
このうち、本研究では主に1,2,4の手順について担当する。

\subsection{研究計画}
現時点でまだ研究のアプローチが確定しているわけではないが、大まかな研究計画を以下に示す。
\begin{itemize}
  \item 4月...3次元モデル、テクスチャ生成手法に関する論文調査、研究手法の考案
  \item 5月...全方位画像から特定の方向を向いた透視投影画像の生成
  \item 6月...カメラ姿勢推定結果を利用したテクスチャ画像の割り当て方法を検討
  \item 8月...学校内で簡易的な3次元モデル及びテクスチャの作成
  \item 10月...フロアマップから3次元モデルを生成し、テクスチャ画像を割り当て
  \item 11月...研究結果に関する考察
  \item 12月...中間発表資料作成
\end{itemize}
%参考文献
\begin{thebibliography}{99}
\bibitem{bib_1} 菅谷保之,FaceMeshを利用した実寸サイズの3次元顔モデルの作成,閲覧日2023/7/26
\end{thebibliography}

\end{document}
