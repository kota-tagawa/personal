%\documentclass[11pt,a4paper,twoside,openright]{jreport} %part{}前後に空ページが入る
\documentclass[11pt,a4paper,twoside]{jreport}	%part{}前後に空ページが入らない

\usepackage[dvipdfm]{graphicx,color}
\usepackage{subcaption}
\usepackage{latexsym}
\usepackage{amsmath,amssymb}
\usepackage{bm}
\usepackage{layout}
\usepackage{tabularx}
\usepackage{indentfirst}
\usepackage{color}
\usepackage{float}

\usepackage[
  dvipdfm,
  bookmarks=true,
  bookmarksnumbered=true,
  colorlinks=true]{hyperref}
\AtBeginDvi{\special{pdf:tounicode EUC-UCS2}}


\definecolor{orange}{cmyk}{0,.61,.87,0}
\definecolor{purple}{cmyk}{.2,1,0,0}

\oddsidemargin          11pt
\evensidemargin         11pt
\topmargin               0pt
\headheight              0pt
\headsep                 0pt
\topskip                11pt
\textheight            680pt
\textwidth             450pt
\footskip               33pt
\parindent              11pt
\marginparwidth          0pt

\setlength{\doublerulesep}{0.4pt}
\setcounter{topnumber}{0}
\setcounter{bottomnumber}{5}
\setcounter{totalnumber}{5}
\setcounter{tocdepth}{2}
\setcounter{chapter}{0}
\setcounter{section}{0}
\setcounter{subsection}{0}

\renewcommand{\textfraction}{0.0}
\renewcommand{\topfraction}{0.0}
\renewcommand{\bottomfraction}{1.0}
\renewcommand{\bibname}{参考文献}

\input{numerical_definition.tex}%行列,ベクトル表記のための定義ファイル
\begin{document}

\begin{center}
    \vspace*{80pt}

    {\bf \huge\hspace{5pt} ワイヤーフレームと全方位画像による \hspace{5pt} } \\
    \vspace{2pt}
    {\bf \huge\hspace{5pt} 簡易モデルを用いた屋内環境での自己位置推定 \hspace{5pt} } \\

    \vspace{60pt}
    \large {2025} \\
    \vspace{20pt}
    \large {修士(工学)} \\
    \vspace{20pt}
    \large {情報$\cdot$知能工学専攻}

    \vspace{60pt}

    \large {\hspace{5pt} 田川 \hspace{10pt} 幸汰 \hspace{5pt}} \\
    \vspace{10pt}
    \large {223337}

    \vspace{60pt}
    \large {豊橋技術科学大学}

    \thispagestyle{empty}
\end{center}        % 表紙
\thispagestyle{empty}

%\chapter*{$BMW9`(B}

$B$3$N%Z!<%8$K$OMW9`$,A^F~$5$l$k(B.
% $B%+%a%i$N;#1F<jK!$K$O(B, "$B%P%l%C%H%?%$%`(B"$B$H8F$P$l$kHo<LBN$N<~0O$KB??t$N%+%a%i$rG[CV$7(B, $B;#1F$9$k<jK!$,$"$k(B. $B$7$+$7$3$N<jK!$O(B, $B<B:]$K;#1F$7$h$&$H$9$k$H%+%a%i$rJ#?tBfMQ0U$7(B, $BHo<LBN$,Cf1{$K1G$k$h$&$K8GDj$9$kI,MW$,$"$k$?$a(B, $B<B8=$,Fq$7$$(B. 
% 
% $B$=$3$G%P%l%C%H%?%$%`$r5<;wE*$K:F8=$9$k<jK!$H$7$F(B, $B<j;}$A$N%+%a%i$GHo<LBN$N<~$j$r;#1F$7$?2hA|$rMQ$$$F%P%l%C%H%?%$%`F02h$r@8@.$9$k<jK!$,$"$k(B. $B$7$+$7$J$,$i(B, $B$3$N<jK!$G$OJ#?t2s%7%c%C%?!<$r@Z$j$J$,$i0\F0$7$F;#1F$9$kI,MW$,$"$k$?$a(B, $B;#1F$K;~4V$H<j4V$,$+$+$k(B. $B$^$?(B, $B;#1F$7$?2hA|$K$h$C$F$O2hA|Cf$N%N%$%:Ey$N1F6A$G%+%a%i0LCV?dDj$K<:GT$7(B, 1$B%U%l!<%`$G$b<:GT$9$k$H@5>o$J%P%l%C%H%?%$%`F02h$,=PNO$5$l$J$$LdBj$,$"$k(B.
% 
% $BK\8&5f$G$O(B, $BHo<LBN$N<~$j$r;#1F$7$?F02h$rMQ$$$F%P%l%C%H%?%$%`F02h$r@8@.$9$k<jK!$rDs0F$9$k(B. $BDs0F<jK!$G$O%U%l!<%`4V$NBP1~E@$NC5:w$KFCD'E@DI@W$rMxMQ$7(B, $B$=$N7k2L$r%+%a%i0LCV?dDj$KMQ$$$k(B. $B%+%a%i0LCV?dDj$r9T$&:]$K$O(B, $B0LCV?dDj$,@5>o$K9T$&$3$H$,$G$-$k%N%$%:$N>/$J$$%U%l!<%`$rF02hCf$+$iA*Br$9$k$3$H$G(B, $B%+%a%i0LCV?dDj@)EY$N8~>e$rA@$&(B. $B$=$7$F4{B8<jK!$HDs0F<jK!$K$F%P%l%C%H%?%$%`F02h$N@8@.<B83$r9T$$(B, $B4{B8<jK!$h$j$b%+%a%i0LCV?dDj$K<:GT$9$k;v>]$,>/$J$/$J$C$?$3$H$r3NG'$7$?(B.     % 概要
%\thispagestyle{empty}

\tableofcontents         % 目次作成コマンド
\thispagestyle{empty}

\setcounter{page}{0}
\chapter{はじめに} \label{sec:chapter}

\section{研究背景} \label{sec:background}
近年、現実空間とデジタル空間を連携させるデジタルツインの概念が注目されている。
デジタルツインの社会実装に関する既存の調査によると、
国内企業の約7割がデジタルツインを導入済み、または導入を検討していると報告されている\cite{IDC_DT_Survey2024}。
特に、屋内空間をデジタル上に再現して利用者の位置を推定することで、それに応じた案内や情報提示を行う技術への期待が高まっている。

しかしながら、同時に市場レポートでは約41\%の企業が導入にあたって予算制約を主な阻害要因として挙げており\cite{GlobalGrowthInsights_DT}、
デジタルツイン導入への関心は高い一方で、3次元計測機器や専用センサーへの投資、システム構築や維持にかかるコストが
導入における大きな障壁となっていることが示されている。
このような社会的背景から、新たな設備投資を必要とせず、カメラで取得した画像のみから屋内環境の3次元モデルを生成する手法が望まれてきた。

しかし、屋内環境は壁や床、天井といった単調な構造が多く、特徴に乏しい場合が多い。
そのため、画像情報に基づいて3次元構造を推定する際には、十分な特徴点が得られず、空間構造を正確に復元することが困難である。
特に、Visual SLAM に代表されるような、自己位置推定を行いながら高精度なマップ生成を同時に行う手法では、
特徴点の不足が自己位置推定の不安定化や、環境マップ品質の低下につながることが指摘されており\cite{ORB_SLAM, Cadena2016}、
画像のみを用いて屋内環境の高精度な3次元モデルを安定して構築することは、依然として大きな課題となっている。

一方で、屋内環境における道案内などのナビゲーション用途では、
必ずしも幾何学的に高精度な3次元モデルや、精密な自己位置推定が常に必要であるとは限らない。
経路案内や現在位置の把握といった目的においては、空間の大まかな構造を表現できる3次元モデルが得られれば十分である場合も多いと考えられる。
このような背景を踏まえ、本研究では屋内ナビゲーション用途に必要な情報に着目し、
高精度な3次元形状復元に依存しない、特徴の乏しい環境にも適用可能な空間表現と、画像に基づく自己位置推定手法について検討する。


\section{研究目的} \label{sec:purpose}

本研究の目的は、特徴に乏しく Visual SLAM の適用が困難な屋内環境において、
ナビゲーション用途に求められる実用的な精度と安定性を有する自己位置推定手法を確立することである。
特に、高精度な自己位置推定を必ずしも前提としない屋内ナビゲーションを対象とし、
既存の建物情報を活用することで、設備投資を抑えつつ安定した動作の実現を目指す。

具体的には、2次元マップから生成したワイヤーフレームモデルに全方位画像から取得したテクスチャを付与した簡易的な3次元モデルを作成し、
入力画像とモデル上のテクスチャとの特徴点マッチングに基づいて自己位置を推定する手法を提案する。

本手法のアプローチには、大きく二つの特徴がある。

第一に、導入障壁が低い点である。
提案手法は、建物に既存の2次元マップと全方位画像という、比較的容易に取得可能な情報のみを用いて環境モデルを構築する。
そのため、高価な3次元計測機器による事前の精密スキャンや、ビーコン・マーカーなどの設備を環境側に設置する必要がなく、
デジタルツイン導入の課題となっているコストと手間を大幅に削減できる。

第二に、特徴の乏しい環境における安定性と、ナビゲーションの継続性を重視した設計思想である。
公共施設やオフィスビルでの経路案内においては、ミリメートル単位の厳密な自己位置推定精度よりも、
自己位置を見失わずに追跡し続ける「安定性」が極めて重要となる。
一般的な Visual SLAM は、特徴が豊かな環境では精密な自己位置推定が可能だが、
単調な壁面など特徴が乏しい環境では、特徴点不足によりトラッキングが破綻しやすい。
本研究では、空間の大まかな構造を表現した簡易モデルを参照することでこの問題を回避する。
これにより、厳密な推定精度よりも、特徴の乏しい環境下であっても安定して自己位置推定を行うことで、
ナビゲーション動作を継続させることを目指す。

以上より、本研究は高精度な三次元形状の復元や新たな設備投資に依存することなく、
画像情報のみに依存する従来の Visual SLAM では自己位置の維持が困難な特徴の乏しい環境下においても、
安定して動作可能な屋内ナビゲーション向け自己位置推定の枠組みを提示することを目的とする。


\section{関連研究}\label{sec:related}

\subsection{Visual SLAM による自己位置推定}

Visual SLAM は、カメラ画像から特徴点を抽出および追跡することで、自己位置推定と環境マップ生成を同時に行う代表的な手法である。
ORB-SLAM\cite{ORB_SLAM} や ORB-SLAM2\cite{ORB_SLAM2} に代表される手法では、ORB 特徴量を用いた高精度なトラッキングおよびループ検出により、
高精度な自己位置推定と三次元マップ生成が可能である。
また、近年では深層学習を導入した手法\cite{CNN_SLAM} も提案されており、特徴点抽出やマッチングの頑健性向上が試みられている。

一方で、Visual SLAM は十分な特徴点が安定して得られることを前提としており、
壁面や床面が単調な屋内環境では、特徴点不足によりトラッキングが不安定になることが指摘されている\cite{ORB_SLAM, Cadena2016}。
例えば、弱いテクスチャ領域を含む環境を対象とした実験では、追跡可能な特徴点数が大きく減少し、
その結果として自己位置推定軌道が真値から逸脱する様子が報告されている(図\ref{one:one})\cite{RobustWeakTextureSLAM}。

したがって、特徴点の乏しい屋内環境において、画像情報のみに依存する Visual SLAM 単独で
既知のマップ座標系上における絶対的な自己位置を高精度かつ安定して推定することは、依然として困難な課題である。

\begin{figure}[H]
    \centering
    \begin{subfigure}{0.4\textwidth}
        \centering
        \includegraphics[width=\linewidth]{figures/1/fr3snnnew.png}
    \end{subfigure}
    \begin{subfigure}{0.4\textwidth}
        \centering
        \includegraphics[width=\linewidth]{figures/1/snn1new.png}
    \end{subfigure}
    \begin{subfigure}{0.4\textwidth}
        \centering
        \includegraphics[width=\linewidth]{figures/1/fr3nnnnew.png}
    \end{subfigure}
    \begin{subfigure}{0.4\textwidth}
        \centering
        \includegraphics[width=\linewidth]{figures/1/nnn2new.png}
    \end{subfigure}
    \caption{弱いテクスチャ環境における自己位置推定結果および追跡特徴点の可視化. 出典:Y. Liu et al.(2022)\cite{RobustWeakTextureSLAM}, Fig.~4}
    \label{one:one}
\end{figure}

\subsection{画像ベースの自己位置推定手法}

環境マップ生成を行わず、既存の環境モデルとカメラ画像との対応付けにより自己位置推定を行う手法も多く提案されている。
代表的なアプローチとしては、SfM (Structure from Motion) により事前に構築した3次元点群と画像特徴量とのマッチングに基づく手法\cite{ImageLocalizationSfM} や、
大規模画像データベースを用いた画像検索に基づく位置推定手法\cite{ImageRetrievalLocalization} が挙げられる。

これらの手法は、事前に構築された環境モデルを利用するため、未知環境を探索するSLAMに比べて累積誤差の影響を受けにくく、安定した自己位置推定が可能である。
しかし、高密度かつ高精度な3次元モデルやデータベースの事前構築が前提となる。
そのため、テクスチャが単調な環境では特徴点マッチングに基づくモデル生成自体が困難である点や、
点群構築のために大規模な撮影や計測が必要となり、導入・運用面での負担が大きいという課題がある。

\subsection{簡易3次元モデルを用いた位置推定}

高精度な3次元形状復元に依存せず、簡易的な三次元環境モデルや幾何学的制約を用いて自己位置推定を行う試みも報告されている。

例えば、Sattlerら\cite{Sattler2011}は、事前に撮影された画像群に対して SfM (Structure from Motion) を適用し、
3次元空間内の特徴点群のみを抽出・保存することで、軽量な環境マップを構築している。
彼らは、このスパースな3次元点と入力画像の特徴量との直接的な対応付けを行うことで、
高密度なモデルを持たずとも高速かつ高精度な位置推定が可能であることを示した。

また、都市環境などを対象とした Schindlerら\cite{Schindler2007}の研究では、
建物のファサードが持つ幾何学的な繰り返しパターンや、空間内の主要な平面構造に着目している。
この手法では、環境全体を厳密に復元するのではなく、位置特定に有効な特徴を選択的にデータベース化したり、
建物を鉛直平面として近似するなどの幾何学的制約を利用することで、効率的な自己位置推定を実現している。

これらの手法は、必ずしも高密度な3次元形状復元を行わずとも、
特徴点の配置や簡易的な幾何情報のみで実用的な自己位置推定が可能であることを示している点で重要である。

本研究は、こうした「簡易モデルによる推定」というアプローチを踏襲しつつ、
モデル生成のソースとして、既に存在する「2次元マップ」を利用する点に独自性がある。
全方位画像を用いて効率的にテクスチャを付与した簡易3次元モデルを構築することで、現地での大規模な事前撮影やSfMによる点群生成プロセスを省略し、
特徴点が乏しい屋内環境においても低コストで導入可能なナビゲーション手法の実現を目指すものである。

\section{本論文の構成}
本論文の構成を以下に示す。
第2章では、本研究の基盤となる簡易3次元モデルの生成手法について説明する。
第3章では、入力画像と簡易モデルのテクスチャ間の特徴点マッチング、およびその結果に基づく自己位置推定手法について説明する。
第4章では、自己位置推定結果を用いた屋内ナビゲーション手法について説明する。
第5章では、提案手法により実際に生成された簡易3次元モデルおよび自己位置推定結果を示し、
推定精度の屋内ナビゲーションにおける有効性を検証するために行った実験について説明する。


      % はじめに

\chapter{簡易3次元モデルの生成}

\section{簡易3次元モデル生成の方針}

本研究では、屋内ナビゲーションに必要な自己位置推定を効率的に行うため、
現実空間を厳密に模倣するのではなく、計算コストとデータ量を最小限に抑えた「簡易三次元モデル」を生成する方針をとる。

一般に、屋内環境の3次元デジタル化には、レーザースキャナによる高密度な点群計測(図\ref{two:one} 左)や、
多数の画像を用いたフォトグラメトリによるメッシュ生成(図\ref{two:one} 中央)が用いられることが多い。
これらの手法は、環境の形状を詳細に再現できる反面、データ容量が肥大化しやすく、
モバイル端末上でのリアルタイムな描画や照合処理には不向きである。
また、モデル構築のために専門的な機材や多大な計算時間を要する点も課題となる。

これに対し、本研究で提案する簡易モデル(図\ref{two:one} 右)は、
2次元マップから抽出した少数の頂点と線分(ワイヤーフレーム)を幾何的な骨格とし、
そこに全方位画像から取得したテクスチャ情報を付与することで構成される。

このように、必要最低限の幾何構造をワイヤーフレームで、視覚情報をテクスチャとして表現するアプローチをとることで、
高密度な点群や精密なメッシュの生成に依存することなく、ナビゲーション用途として実用上十分な情報の確保と、システム運用における軽量性の両立を目指す。

\begin{figure}[H]
  \centering
  \begin{tabular}{ccc}
      \includegraphics[width=0.3\linewidth]{figures/2/pointscloud.jpg} &
      \includegraphics[width=0.3\linewidth]{figures/2/3DCG.jpg} &
      \includegraphics[width=0.3\linewidth]{figures/2/model.png} \\
      (a) 点群モデル & (b) フォトグラメトリモデル & (c) 本研究の簡易モデル
  \end{tabular}
  \caption{異なる3次元モデル表現の比較. 従来の点群(a)やメッシュ(b)と比較し、本研究(c)では構造を大幅に簡略化している.}
  \label{two:one}
\end{figure}


\section{座標系の定義}

\subsection{カメラ座標系}
簡易モデル生成に用いる各座標軸の関係を図\ref{three:one}に示す。

\begin{figure}[H]
  \centering
  \includegraphics[width=0.5\linewidth]{figures/3/axis.png}
  \caption{世界座標系とカメラ座標系、画像座標系の関係}
  \label{three:one}
\end{figure}

カメラ座標系はカメラの焦点位置を原点とし、
光軸方向を $z$ 軸、水平右方向を $x$ 軸、鉛直下方向を $y$ 軸と定める。
一方、世界座標系はモデル床面を $X$-$Y$ 平面、鉛直上向きの法線方向を $Z$ 軸として定義する。
世界座標系上の3次元点 $\bm{p}_w$ は、
カメラの回転行列 $\bm{R}$ と並進ベクトル $\bm{t}$ を用いて、次式によりカメラ座標系上の点 $\bm{p}_c$ に変換される。

\begin{equation}
\bm{p}_c = \bm{R}\bm{p}_w + \bm{t}
\end{equation}

\subsection{画像座標系}
画像座標系は画像左上を原点とし、水平方向を $u$ 軸、垂直方向を $v$ 軸と定める。
カメラ座標系上の3次元点 $\bm{p}_c = (x_c, y_c, z_c)^\top$ は、
カメラ内部パラメータ行列 $\bm{K}$ を用いて、次式により画像座標系上の同次座標 $\bm{p}_s$ へ射影される。

\begin{equation}
\bm{p}_s = \bm{K}\bm{p}_c
\end{equation}

ここで $\bm{p}_s = (u_s, v_s, w_s)^\top$ とすると、実際の透視投影画像上の画素座標 $(u, v)$ は、
$w_s$ による正規化(透視除算)により次式で得られる。

\begin{equation}
u = \frac{u_s}{w_s}, \quad v = \frac{v_s}{w_s}
\end{equation}

\subsection{カメラ内部パラメータの設定}
本研究では理想的な透視投影モデルを仮定し、カメラ内部パラメータを幾何学的に設定する。
内部パラメータ行列 $\bm{K}$ は次式で表される。

\begin{equation}
\bm{K} =
\begin{pmatrix}
f_x & 0 & c_x \\
0 & f_y & c_y \\
0 & 0 & 1
\end{pmatrix}
\end{equation}

ここで $f_x, f_y$ はピクセル単位の焦点距離を表す。
出力する透視投影画像の幅を $W_p$、高さを $H_p$ とし、
水平視野角を $\Theta$、垂直視野角を $\Phi$ と設定した場合、
各焦点距離は次式で表される。

\begin{equation}
f_x = \frac{W_p}{2 \tan(\Theta/2)}, \quad
f_y = \frac{H_p}{2 \tan(\Phi/2)}
\end{equation}

また $c_x, c_y$ は画像中心を表し、透視投影画像の中心座標 $(W_p/2, H_p/2)$ に設定する。


\section{ワイヤーフレーム生成の方針}

本節では、2次元マップを基盤として、屋内環境の幾何構造を表す3次元ワイヤーフレームモデルを構築する一連の手順について述べる。
本手法におけるワイヤーフレーム生成は、以下の3つの主要なフェーズから構成される。

\begin{enumerate}
  \item \textbf{2次元マップと3次元空間の対応付け}: \\
  画像として与えられる2次元マップ上の位置と、実際の世界座標との対応関係を定義する。
  \item \textbf{床境界の定義と最適化}: \\
  2次元マップ上で指定された床領域の輪郭に対し、カメラ位置を考慮して、適切な間隔で床境界点を追加する。
  \item \textbf{3次元幾何構造の構築}: \\
  確定した床境界に基づき、床面および壁面のメッシュ化を行うことで、最終的な3次元ワイヤーフレームを出力する。
\end{enumerate}

なお、ここでいう床境界点とは、屋内空間における床面と壁面の境界線を構成する一連の頂点を指す。
本研究では、ユーザがマップ上での対応点指示や大まかな床領域指定を行い、それ以降の詳細な頂点生成や構造化を計算機が自動で行う「半自動」なアプローチを採用する。
以下、この処理フローに必要な入力情報と前提条件について述べた後、各フェーズの詳細について説明する。

\subsection{ワイヤーフレーム生成の前提条件}\label{wire_input}

ワイヤーフレーム生成にあたっては、以下の情報が利用可能であることを前提とする。

\begin{itemize}
  \item \textbf{屋内環境を表す2次元マップ}: \\
  建物のフロアマップや設計図などの画像データ。
  \item \textbf{2次元マップと3次元空間の対応点}: \\
  マップ上の画素座標と、実空間の3次元座標との対応関係を示す4点以上の点対。これは、マップと実空間の位置合わせのために用いられる。
  \item \textbf{屋内環境内でテクスチャを取得したカメラ位置}: \\
  全方位画像の撮影位置。これは後述する床境界点のサンプリングにおいて、形状の詳細度を決定するために用いられる。
\end{itemize}

次節より、これらの入力情報に基づいた具体的な生成アルゴリズムについて述べる。


\section{2次元マップと3次元空間の対応付け}

本研究では、2次元マップ上で定義された床境界の頂点座標を、実際の3次元空間内の床面上へ写像することで位置合わせを行う。
床面は水平であると仮定し、実空間上の3次元座標は常に $Z=0$ に固定されるものとする。

この写像を行う手法として、本研究では「アフィン変換による自動推定」と、
マップの歪みを考慮した「手動による直接対応付け」の2種類のアプローチを状況に応じて使い分ける。

\subsection{アフィン変換による写像}
建築図面や正確なフロアマップが得られる場合、2次元マップと実環境との関係は、
平行移動・回転・スケーリング・せん断を含むアフィン変換によって十分に表現可能である。

まず、2次元マップ上の点 $\bm{p}_i = (x_i, y_i)^\top$ と、
それに対応する実空間(床面)上の点 $\bm{q}_i = (X_i, Y_i)^\top$ の対応ペアを $N$ 点($N \geq 3$、推奨は4点以上)取得する。
両者の関係は、アフィン変換行列 $\bm{A}$ を用いて以下のように表される。

\begin{equation}
\begin{pmatrix}
X_i \\
Y_i
\end{pmatrix}
=
\bm{A}
\begin{pmatrix}
x_i \\
y_i \\
1
\end{pmatrix}, \quad
\bm{A}
=
\begin{pmatrix}
a_{11} & a_{12} & a_{13} \\
a_{21} & a_{22} & a_{23}
\end{pmatrix}
\end{equation}

ここで、すべての対応点に対して変換誤差が最小となるような行列 $\bm{A}$ を推定する。
具体的には、以下の再投影誤差の二乗和 $E$ を最小化する最小二乗法により $\bm{A}$ を決定する。

\begin{equation}
\hat{\bm{A}} = \underset{\bm{A}}{\operatorname{argmin}} \sum_{i=1}^{N} \left\| \bm{q}_i - \bm{A} \tilde{\bm{p}}_i \right\|^2
\end{equation}

ただし、$\tilde{\bm{p}}_i = (x_i, y_i, 1)^\top$ は同次座標表現である。
このようにして求めた変換行列 $\hat{\bm{A}}$ を用いることで、
2次元マップ上で定義されたすべての頂点を、一括して3次元座標系へ変換する。

\subsection{非線形な歪みへの対応}
一方、簡易的な2次元マップを利用する場合、マップ自体が非線形な歪みを含んでいることがあり、
単一のアフィン変換行列では十分な精度で写像できない場合がある。
このような場合には、上述の自動変換を用いず、床境界の頂点に対して実測した3次元座標を手動で直接割り当てる手法を採用する。
これにより、局所的な歪みが全体の位置合わせに悪影響を与えることを防ぎ、整合性の取れたモデル生成を可能とする。

\subsection{床境界の整合性確認と補正}
上記いずれかの手法により3次元座標を得た後、床境界の頂点列としての整合性を確認する。
まず、頂点列が閉ループを形成していない場合には、始点と終点を接続することで多角形となるように修正する。
さらに、頂点列の並び順を外積により判定し、
反時計回りとなっている場合には順序を反転させることで、すべての床境界が時計回りとなるように統一する。


\section{床境界におけるカメラ位置に基づくサンプリング}

高品質なテクスチャを生成するためには、テクスチャの歪みを抑えるための詳細な頂点が必要となる。
しかし、これらすべてを手作業で指定することは多大な労力を要するため、現実的ではない。 
そこで本研究では、床境界上においてカメラ配置を考慮しながら適切な間隔で境界点を自動生成する手法を提案する。
図\ref{two:two}に、カメラ位置に基づいて床境界点をサンプリングする概念図を示す。

\begin{figure}[H]
  \centering
  \includegraphics[width=0.8\linewidth]{figures/2/sampling.png}
  \caption{床境界点のサンプリング方法. カメラからの射影点周辺に重点的に頂点を配置する.}
  \label{two:two}
\end{figure}

床境界は閉じた多角形として定義され、隣接する2点 $\bm{p}_0$ と $\bm{p}_1$ が1つの境界エッジを構成する。
本手法では、まず各エッジの始点 $\bm{p}_0$ を固定の境界点として採用した上で、カメラ位置に応じた点を動的に追加していく。

具体的には、各カメラ位置 $\bm{c}$ から境界エッジへの正射影を考える。
エッジの方向ベクトル $\bm{d}$ および、始点からの距離比を表す射影係数 $t$ は次式で与えられる。

\begin{align}
\bm{d} &= \bm{p}_1 - \bm{p}_0 \\
t &= \frac{(\bm{c} - \bm{p}_0)^\top \bm{d}}{\bm{d}^\top \bm{d}}
\end{align}

これより、境界直線上への射影点 $\bm{p}_{\mathrm{proj}}$ は次式で表される。

\begin{equation}
\bm{p}_{\mathrm{proj}} = \bm{p}_0 + t \bm{d}
\end{equation}

ここで、射影係数 $t$ が $0 \leq t \leq 1$ (エッジの範囲内)を満たし、かつカメラとの距離が閾値以下である場合、すなわち
\begin{equation}
\lVert \bm{c} - \bm{p}_{\mathrm{proj}} \rVert \leq d_{\mathrm{th}}
\end{equation}
を満たす場合に、その射影点を有効な候補とする。

次に、得られた有効な射影点に基づき、最終的なサンプリング位置(追加の境界点)を決定する。 
テクスチャマッピング時の歪みを最小限に抑えるには、カメラ正面に近い領域、すなわち射影点付近の幾何形状を重点的に保持することが望ましい。
そこで、同一エッジ上に複数の有効な射影点 $\bm{p}_{\mathrm{proj1}}, \bm{p}_{\mathrm{proj2}}$ が存在する場合には、
その点間距離とユーザが定義するサンプリング間隔 $d_{\mathrm{sample}}$ との大小関係に応じて処理を分岐する。

具体的には、射影点間の距離が十分に大きい場合($\|\bm{p}_{\mathrm{proj1}} - \bm{p}_{\mathrm{proj2}}\| > d_{\mathrm{sample}}$)には、
2つの射影点に挟まれた区間内において、それぞれの射影点から相手側へ一定距離だけ進んだ位置に境界点を追加する。
一方、射影点間の距離が小さい場合($\|\bm{p}_{\mathrm{proj1}} - \bm{p}_{\mathrm{proj2}}\| \leq d_{\mathrm{sample}}$)には、
過度な細分化を防ぐため、両者の中点を1つの境界点として追加する。
これにより、カメラ視点に対して最適な密度で床境界点を配置することが可能となる。

最後に、生成された床境界点列全体について隣接点間の距離を確認し、分布の均一化を行う。
間隔が大きすぎる部分(疎な領域)には中点を追加し、間隔が小さすぎる部分(密な領域)では点を統合する。
この後処理により、床境界全体が極端に偏ることなく、おおむね一定のサンプリング間隔で分布するように調整し、メッシュ生成の安定性を確保する。


\section{床面および壁面の幾何構造}

床面は、前節までで得られた床境界頂点列 $\{\bm{P}_i\}$ によって囲まれた領域として定義される。
ここで、床境界は辺同士が互いに交差しない「単純多角形」を形成しているものとする。
また、各床境界頂点は床面上に存在し、その $z$ 座標は $z=0$ に固定されている。

床面に対応する天井境界点列 $\{\bm{P}_i^{\mathrm{ceil}}\}$ は、床境界頂点列の平面形状を保ったまま、
高さ方向に一定量 $H$ だけ平行移動することで生成する。
すなわち、床境界頂点 $\bm{P}_i = (x_i, y_i, 0)$ に対し、
対応する天井境界頂点は $\bm{P}_i^{\mathrm{ceil}} = (x_i, y_i, H)$ として与えられる。

\subsection{床面のメッシュ化}
実際の屋内環境における床形状は、L字路や柱の出っ張りなどが存在するため、必ずしも凸多角形とはならず、凹部を含む一般多角形となる。
このような形状に対して、均質で安定したメッシュを生成するため、
本研究では「制約付き Delaunay 三角形分割(Constrained Delaunay Triangulation)」を用いて床面を分割する。
この手法は、Shewchuk によって提案された高品質な2次元メッシュ生成手法として知られている\cite{Shewchuk1996Triangle}。

具体的には、床境界頂点を入力頂点集合とし、隣接する床境界頂点同士を結ぶラインを拘束辺として設定する。
これにより、本来存在しない壁の外側や空間の穴にあたる部分にメッシュが生成されるのを防ぐことができる。
また、三角形の最大面積を制約条件として与えることで、過度に細長い三角形の生成を抑制し、テクスチャマッピングに適した形状を保つ。

得られた三角形メッシュに対しては、上方($+z$ 方向)から見たときに、すべての三角形の頂点が時計回りとなるように順序を統一する。
これは、法線ベクトルを鉛直上向きに揃えて、面の向きを統一するために必須の処理である。

\subsection{壁面の構成}
壁面の幾何構造は、床境界と天井境界の対応関係に基づいて生成される。
床境界および天井境界は同一の頂点数と順序を持つ閉ループであるため、
対応する上下の隣接頂点対を用いることで壁面を構成できる。

具体的には、床境界の隣接する頂点 $\bm{P}_i, \bm{P}_{i+1}$ と、
それらに対応する天井境界頂点 $\bm{P}_i^{\mathrm{ceil}}, \bm{P}_{i+1}^{\mathrm{ceil}}$ を結ぶことで、
一枚の壁面を表す四角形を定義する。
このようにして生成された床面および壁面の幾何構造により、屋内環境の簡易3次元ワイヤーフレームモデルが構築される。


\section{全方位画像を用いたテクスチャ取得の方針}
3次元モデルへのテクスチャ割り当てにおいては、モデル表面を十分に覆う視点から撮影された画像を効率的に取得することが重要である。 
一般的な透視投影カメラを用いる場合、多方向のテクスチャ情報を網羅するためには、カメラの向きを変えながら多数の画像を撮影する必要があり、撮影およびデータ管理に関わるコストが増大するという課題がある。

そこで本研究では、単一の撮影によって全周囲の視覚情報を取得可能な全方位カメラを用いる。
全方位画像は、カメラ位置を中心とする前後・左右・上下の全周囲の風景を一度に記録した画像である。
幾何学的には、カメラを中心とした球面上の情報として定義され、
データ形式としては、一般に正距円筒図法(Equirectangular Projection)などを用いて2次元平面画像に展開し保存されている。

この画像に対し、任意の視線方向を持つ仮想カメラを定義し、その画像平面へ画素を再投影することで、特定の方向に対応する透視投影画像を生成できる。
この特性を利用することで、実際に複数視点から撮影を行うことなく、任意の方向を向いた多数の画像群を計算機上で効率的に生成することが可能となる。

\section{透視投影画像変換}
本研究では、全方位画像を球面上の画素情報として扱う。 
生成する透視投影画像の各画素について、その視線方向に対応する全方位画像の画素値を取得することで、任意方向の画像を生成する。

正距円筒画像の幅および高さをそれぞれ $W_e, H_e$ とする。
全方位カメラの焦点距離 $f$ は全方位画像幅 $W_e$ を用いて次式で表される。

\begin{equation}
f = \frac{W_e}{2 \pi}
\end{equation}

生成する透視投影画像の幅および高さをそれぞれ $W_p, H_p$ とし、
透視投影画像上の画素座標を $(u_p, v_p)$ とすると、
この画素に対応する光軸方向を $z$ 軸とする3次元の視線ベクトル $\bm{x}$ は次式で定義される。
ただし、$N[ \cdot ]$はベクトルのノルムを1にする正規化する正規化作用素である。

\begin{equation}
\bm{x} = N[
\begin{pmatrix}
u_p - W_p/2 \\
v_p - H_p/2 \\
f
\end{pmatrix}
]
\end{equation}

カメラの視線方向を変更するための回転行列 $\bm{R}$ を定義する。
本研究では光軸周りの回転は考慮せず、水平方向の回転$\theta_{eye}$ および 垂直方向の回転$\phi_{eye}$ のみにより姿勢を決定する。
これに対応する回転行列 $\bm{R}(\theta_{eye}, \phi_{eye})$ は次式で与えられる。

\begin{equation}
\bm{R}(\theta_{eye}, \phi_{eye}) = 
\begin{pmatrix}
\cos \theta_{eye} & 0 & \sin \theta_{eye} \\
0 & 1 & 0 \\
-\sin \theta_{eye} & 0 & \cos \theta_{eye}
\end{pmatrix}
\begin{pmatrix}
1 & 0 & 0 \\
0 & \cos \phi_{eye} & -\sin \phi_{eye} \\
0 & \sin \phi_{eye} & \cos \phi_{eye}
\end{pmatrix}
\end{equation}

回転後の視線ベクトル $\bm{x}' = (X', Y', Z')^\top$ は、次式で計算される。

\begin{equation}
\bm{x}' = \bm{R}(\theta_{eye}, \phi_{eye}) \bm{x}
\end{equation}

得られた視線ベクトル $\bm{x}'=(X,Y,Z)$ を用いて、全方位画像の球面座標系 $(\theta_e, \phi_e)$ は次式で計算される。
ただし、$(X',Y',Z')^\top$は$(X,Y,Z)^\top$のノルムを1に正規化したものである。

\begin{equation}
\theta_e = \tan^{-1}\left(\frac{X'}{Z'}\right)
\end{equation}
\begin{equation}
\phi_e = \sin^{-1}\left(\frac{Y'}{\sqrt{X'^2 + Y'^2 + Z'^2}}\right)
\end{equation}

ここで、全方位画像(正距円筒画像)の幅を $W_e$、高さを $H_e$ とすると、
球面座標 $(\theta_e, \phi_e)$ に対応する全方位画像上の画素座標 $(u_e, v_e)$ は次式で与えられる。

\begin{equation}
u_e = \left( \theta_e + \pi \right) \frac{W_e}{2\pi}
\end{equation}
\begin{equation}
v_e = \left( \phi_e + \frac{\pi}{2} \right) \frac{H_e}{\pi}
\end{equation}

以上の手順により、全方位画像の画素 $(u_e, v_e)$ の輝度値を参照することで、
透視投影画像上の画素 $(u_p, v_p)$ の画素値を決定する。


\section{透視投影カメラによる全方位カメラ位置姿勢推定}
点特徴および線特徴を使ってカメラ位置を推定するため、直交射影に基づく点特徴と線特徴を用いたカメラ姿勢推定\cite{Sugaya2024}を行う。
これは、直交射影に基づく点の共線性誤差と線の共面性誤差に対して同一の定式化を行うことで点特徴と線特徴を同時に扱い、カメラ姿勢を推定する手法である。

直交射影の共線性と共面性に基づくカメラ姿勢推定に必要な入力は以下の通りである。
\begin{itemize}
  \item $\bm{p}_a$ : 世界座標系で表現された空間点の座標。透視投影画像に映らない点は除外する。
  \item $\bm{v}_a$ : $\bm{p}_a$ に対応する画像上の特徴点座標(カメラ座標系)。
  \item $\bm{d}_a$ : 世界座標系で表現された直線 $L_a$ の方向ベクトル。
  \item $\bm{r}_a$ : 世界座標系で表現された直線 $L_a$ 上の点の座標。
  \item $\bm{n}_a$ : $L_a$ に対応する画像上の直線の法線ベクトル(カメラ座標系)。
\end{itemize}

$\bm{v}_a$ は透視投影画像上で $\bm{p}_a$ に対応する画素を選択し、前節で定義したカメラモデルを用いてカメラ座標系上のベクトルとして表現する。
画像中心を原点とした画素座標 $(u', v')$ を以下のように求める。
\begin{equation}
u' = u_p - \frac{W_p}{2}, \quad v' = v_p - \frac{H_p}{2}
\end{equation}
これに焦点距離 $f$ を追加し、$\bm{v}_a = (u', v', f)^\top$ とする。

$\bm{d}_a$ は 3 次元モデルの頂点座標を用いて、以下の式で求める。
\begin{equation}
\bm{d}_i = \frac{\bm{p}_{i+1} - \bm{p}_i}{\|\bm{p}_{i+1} - \bm{p}_i\|}
\end{equation}

$\bm{n}_a$ は画像上の直線端点に対応する 2 つのベクトル $\bm{v}_i, \bm{v}_{i+1}$ から外積を計算して求める。
\begin{equation}
\bm{n}_i = \frac{\bm{v}_i \times \bm{v}_{i+1}}{\|\bm{v}_i \times \bm{v}_{i+1}\|}
\end{equation}

これらの入力を用いて、複数視線方向の透視投影画像に対する自己位置推定を反復的に行うことで、点特徴および線特徴に基づく目的関数を最小化する。
最終的に得られるカメラの回転行列 $\bm{R}$ と並進ベクトル $\bm{t}$ は、世界座標系の三次元点をカメラ座標系に変換するための変換行列である。





\section{メッシュの座標系変換と投影}

メッシュ頂点の世界座標
$\bm{v}_{\rm world}^{(i)} \in \mathbb{R}^3$($i=1,2,3$)を,
カメラの回転行列 $\bm{R}$ および並進ベクトル $\bm{t}$ を用いて,
次式によりカメラ座標系へ変換する。

\begin{equation}
\bm{v}_{\rm cam}^{(i)} = \bm{R}\bm{v}_{\rm world}^{(i)} + \bm{t}
\end{equation}

変換後の頂点座標を用いて,
メッシュの重心方向ベクトル $\bm{c}_{\rm cam}$ を次式で求める。

\begin{equation}
\bm{c}_{\rm cam} = \frac{1}{3} \sum_{i=1}^{3} \bm{v}_{\rm cam}^{(i)}
\end{equation}

本研究では,各メッシュの重心方向にカメラ視線を向けることで,
当該メッシュが透視投影画像の中心付近に投影されるよう制御する。
これにより,射影歪みを抑えた状態でテクスチャを取得することが可能となる。

この目的のため,メッシュ重心方向 $\bm{c}_{\rm cam}$ を視線方向とする
射影用回転行列 $\bm{R}_{\rm proj}$ を構成する。
具体的には,重心方向を前方軸とし,
上方向ベクトルとの外積により左右方向を定めることで,
右手系の直交基底を形成する。
得られた直交基底を列ベクトルとして並べることで,
射影用回転行列を定義する。

この回転行列を用いて,
カメラ座標系上の頂点を回転させる。

\begin{equation}
\bm{v}_{\rm proj}^{(i)} = \bm{R}_{\rm proj}^{\top} \bm{v}_{\rm cam}^{(i)}
\end{equation}

回転後の3次元点をカメラ内部パラメータ行列 $\bm{K}$ を用いて画像座標系へ投影する。
得られる画素座標が画像範囲内に収まらない場合には,画像中心からの最大距離に基づいて,
透視投影画像のサイズおよび投影スケールを調整する。
この処理により,メッシュが画像外に切り落とされることを防ぎ,単一の透視投影画像からメッシュ全体のテクスチャを確実に取得できる。

\subsection{テクスチャ候補の評価}
距離が遠いテクスチャや、正面方向から撮影していないテクスチャはテクスチャの視覚的品質劣化の原因となるため、テクスチャ割り当ての条件として、以下の物理量を計算する。

\begin{itemize}
  \item メッシュ重心までの距離
  \begin{equation}
  d = \|\bm{c}_{\rm cam}\|
  \end{equation}

  \item メッシュ重心方向ベクトル $\bm{c}_{\rm cam}$ とメッシュ法線ベクトル $\bm{n}$ のなす角
  \begin{equation}
    \begin{aligned}
    \bm{n} &= \frac{(\bm{v}_{\rm cam}^{(2)} - \bm{v}_{\rm cam}^{(1)}) \times (\bm{v}_{\rm cam}^{(3)} - \bm{v}_{\rm cam}^{(1)})}
    {\|(\bm{v}_{\rm cam}^{(2)} - \bm{v}_{\rm cam}^{(1)}) \times (\bm{v}_{\rm cam}^{(3)} - \bm{v}_{\rm cam}^{(1)})\|}
    \\
    \theta &= \arccos \left( \frac{\bm{c}_{\rm cam} \cdot \bm{n}}{\|\bm{c}_{\rm cam}\| \, \|\bm{n}\|} \right)
    \end{aligned}
  \end{equation}
\end{itemize}
これらの評価量を用いて、各メッシュに対して距離が近く、かつ正面に近い視点から取得された
テクスチャを優先的に選択することで、視覚的品質の高いテクスチャ割り当てを実現する。

\subsection{テクスチャの射影変換}

透視投影画像におけるメッシュの対応領域は、視点方向や撮影条件に起因する透視歪みを含み、
画像座標系上では任意形状の多角形となる。 
この歪んだ領域をそのまま使用すると、モデルの三次元形状との幾何学的な整合性が損なわれ、
適切なテクスチャマッピングが困難となる。

そこで本研究では、透視投影画像上のメッシュ領域に対して幾何変換を適用し、
正規化されたテクスチャ座標系上で定義される本来の形状へと変換する。 
本処理により透視歪みを除去し、テクスチャ画素と三次元メッシュ上の点との正しい対応関係を確立する。

変換手法はメッシュの形状に応じて決定する。 三角形メッシュにはアフィン変換を適用し、元の三角形形状へとマッピングを行う。
 一方、四角形メッシュにはホモグラフィ変換を適用し、正規化された長方形テクスチャを生成する。

最後に、変換後の画像に対してマスク処理を行い、メッシュ領域外の不要な画素を除去することで、
各メッシュ形状に適合したテクスチャ画像を抽出する。


\section{テクスチャの視覚的品質の改善}

複数の視点から取得されたテクスチャを同一メッシュに割り当てる場合、視点差や撮影条件の違いにより、
テクスチャ境界に不連続が生じ、四角的品質が悪化してしまう。
本研究では、この不連続を低減するため、隣接する2つのテクスチャに対してブレンド処理を行う。

各テクスチャには視線方向に基づく左右情報が付与されており、これを用いて左右関係を判定する。
左側テクスチャを $I_{\mathrm{L}}$、右側テクスチャを $I_{\mathrm{R}}$とする。

ブレンドは画像全体ではなく、画像幅 $W$ の中央付近に設定した、$x$ 方向の狭い領域に限定して適用する。
画素位置 $(x, y)$ におけるブレンド結果 $I(x, y)$ は、次式で与えられる。
\begin{equation}
I(x, y)
=
w_{\mathrm{L}}(x)\, I_{\mathrm{L}}(x, y)
+
w_{\mathrm{R}}(x)\, I_{\mathrm{R}}(x, y),
\end{equation}
ただし、
\begin{equation}
w_{\mathrm{L}}(x) + w_{\mathrm{R}}(x) = 1
\end{equation}
を満たす。

ブレンド領域内では,$x$ 方向に沿って連続的に変化する重みを用いるため、
シグモイド関数に基づき右側テクスチャの重み $w_{\mathrm{R}}(x)$ を次式で定義する。
\begin{equation}
w_{\mathrm{R}}(x)
=
\frac{1}{1 + \exp\!\left(-k\,\tilde{x}\right)},
\end{equation}
ここで,$\tilde{x} \in [-1,1]$ はブレンド領域内で正規化された水平方向位置を表し、
$k$ は重みの遷移の急峻さを制御するパラメータである。
左側テクスチャの重みは
\begin{equation}
w_{\mathrm{L}}(x) = 1 - w_{\mathrm{R}}(x)
\end{equation}
として与える。

ブレンド領域外では、一方のテクスチャのみを用いることで、テクスチャ全体の解像感および幾何的整合性を維持する。
このように、境界付近に限定した方向性を持つブレンドを行うことで、テクスチャ境界に生じる輝度不連続を効果的に低減できる。

      % モデル生成

\chapter{簡易モデルへのテクスチャ割り当て}

\section{全方位画像から透視投影画像生成}

\subsection{全方位画像を用いたテクスチャ取得の方針}
3次元モデルへのテクスチャ割り当てにおいては、モデル表面を十分に覆う視点から撮影された画像を効率的に取得することが重要である。
一般的な透視カメラを用いる場合、多方向のテクスチャ情報を得るためには、カメラ姿勢を変更しながら多数の画像を撮影する必要があり、
撮影および管理に関わるコストが増大するという課題がある。

そこで本研究では、単一の撮影によって全周囲の視覚情報を取得可能な全方位カメラを用いる。
全方位画像は、カメラ中心を基準とした全方向の視線情報を一枚の画像として保持しており、
幾何学的変換を行うことで、任意の視線方向に対応する透視投影画像を生成できる。
この特性により、複数視点から撮影した場合と同等の画像群を、高い効率で取得することが可能となる。
(複数方向の透視投影画像を生成する図を示した方がわかりやすい)

\subsection{透視投影画像変換}
本研究では、全方位画像を球面上の輝度分布として扱い、
透視投影画像の各画素に対応する視線方向を幾何学的に定義することで、
全方位画像から任意視線方向の透視投影画像を生成する。

透視投影画像の幅および高さをそれぞれ $W_p, H_p$ とし、
画像上の画素座標を $(u, v)$ とする。
透視投影面を $z=1$ に固定すると、画素 $(u, v)$ に対応する視線ベクトル $\bm{d}$ は
次式で与えられる。
\begin{equation}
\bm{d} =
\begin{pmatrix}
(u - W_p/2)\,\Delta x \\
(v - H_p/2)\,\Delta y \\
1
\end{pmatrix}
\end{equation}

ここで、$\Delta x, \Delta y$ はそれぞれ水平方向および垂直方向の画素間隔を表し、
水平画角 $\Theta$、垂直画角 $\Phi$ を用いて次式で定義される。
\begin{equation}
\Delta x = \frac{2 \tan(\Theta/2)}{W_p}, \quad
\Delta y = \frac{2 \tan(\Phi/2)}{H_p}
\end{equation}

生成したい透視投影画像の視線方向を表す回転行列を $\bm{R}$ とすると、
回転後の視線ベクトル $\bm{d}'$ は次式で与えられる。
\begin{equation}
\bm{d}' = \bm{R}\bm{d}
\end{equation}

回転後の視線ベクトル $\bm{d}' = (d'_x, d'_y, d'_z)^\top$ を球面座標系へ変換し、
方位角 $\Theta$ および仰角 $\Phi$ を次式で求める。
\begin{equation}
\Theta = \arctan2(d'_x, d'_z)
\end{equation}
\begin{equation}
\Phi = -\arctan\!\left(\frac{d'_y}{\sqrt{d'^2_x + d'^2_z}}\right)
\end{equation}

全方位画像(正距円筒画像)の幅および高さをそれぞれ $W, H$ とすると、
球面座標 $(\Theta, \Phi)$ に対応する全方位画像上の画素座標 $(x, y)$ は
次式で与えられる。
\begin{equation}
x = W\left(\frac{\Theta}{2\pi} + \frac{1}{2}\right)
\end{equation}
\begin{equation}
y = H\left(\frac{1}{2} - \frac{\Phi}{\pi}\right)
\end{equation}


\section{座標系の定義}

\subsection{カメラ座標系}
カメラ座標系はカメラの焦点位置を原点とし、
光軸方向を $z$ 軸、
水平方向右向きを $x$ 軸、
鉛直方向下向きを $y$ 軸と定める。
モデル床面を $X$-$Y$ 平面、法線方向を $Z$ 軸とする世界座標系で表された三次元点 $\bm{p}_w$ は、
カメラの回転行列 $\bm{R}$ と並進ベクトル $\bm{t}$ を用いて、
次式によりカメラ座標系上の点 $\bm{p}_c$ に変換される。

\begin{equation}
\bm{p}_c = \bm{R}\bm{p}_w + \bm{t}
\end{equation}

\subsection{スクリーン座標系}
スクリーン座標系は画像左上を原点とし、水平方向を $u$ 軸、鉛直方向を $v$ 軸と定める。
カメラ座標系上の3次元点 $\bm{p}_c = (x_c, y_c, z_c)^\top$ は、
カメラ内部パラメータ行列 $\bm{K}$ を用いて、次式によりスクリーン座標系へ射影される。

\begin{equation}
\bm{p}_s = \bm{K}\bm{p}_c
\end{equation}

ここで $\bm{p}_s = (u_s, v_s, w_s)^\top$ とすると、透視投影画像上の画素座標 $(u, v)$ は正規化により次式で得られる。

\begin{equation}
u = \frac{u_s}{w_s}, \quad v = \frac{v_s}{w_s}
\end{equation}

\subsection{カメラ内部パラメータの設定}
理想的な透視投影画像を仮定し、カメラ内部パラメータを幾何学的に設定する。
内部パラメータ行列 $\bm{K}$ は次式で表される。

\begin{equation}
\bm{K} =
\begin{pmatrix}
f_x & 0 & c_x \\
0 & f_y & c_y \\
0 & 0 & 1
\end{pmatrix}
\end{equation}

ここで $f_x$、$f_y$ はピクセル単位の焦点距離を表す。
全方位画像の幅 $W_e$ と出力透視投影画像の幅 $W_p$、水平視野角 $\theta$ および垂直視野角 $\phi$ から
$f_x = \frac{W_p}{2 \tan(\theta/2)}$、$f_y = \frac{H_p}{2 \tan(\phi/2)}$とあらわされる。
また$c_x$、$c_y$ は光軸中心を表し、透視投影画像の中心に設定する。


\section{複数の透視投影カメラによる全方位カメラ位置姿勢推定}
点特徴および線特徴を使ってカメラ位置を推定するため、直交射影に基づく点特徴と線特徴を用いたカメラ姿勢推定\cite{sugaya2024}を行う。
これは、直交射影に基づく点の共線性誤差と線の共面性誤差に対して同一の定式化を行うことで点特徴と線特徴を同時に扱い、カメラ姿勢を推定する手法である。

直交射影の共線性と共面性に基づくカメラ姿勢推定に必要な入力は以下の通りである。
\begin{itemize}
  \item $\bm{p}_a$ : 世界座標系で表現された空間点の座標。透視投影画像に映らない点は除外する。
  \item $\bm{v}_a$ : $\bm{p}_a$ に対応する画像上の特徴点座標。
  \item $\bm{d}_a$ : 世界座標系で表現された直線 $L_a$ の方向ベクトル。
  \item $\bm{r}_a$ : 世界座標系で表現された直線 $L_a$ 上の点の座標。
  \item $\bm{n}_a$ : $L_a$ に対応する画像上の直線の法線ベクトル。
\end{itemize}

$\bm{v}_a$ は透視投影画像上で $\bm{p}_a$ に対応する画素を選択することで求める。
また、画像座標 $(v_x, v_y)$ は光軸中心を原点とした座標に平行移動する:
\begin{equation}
v_x = x - \frac{W}{2}, \quad v_y = y - \frac{H}{2}
\end{equation}
さらに焦点距離 $f$ を追加し、$\bm{v}_a = (v_x, v_y, f)$ とする。

$\bm{d}_a$ は 3 次元モデルの頂点座標を用いて、以下の式で求める。
\begin{equation}
\bm{d}_i = \frac{\bm{p}_{i+1} - \bm{p}_i}{|\bm{p}_{i+1} - \bm{p}_i|}
\end{equation}

$\bm{n}_a$ は画像上の 2 点 $(\bm{v}_i, \bm{v}_{i+1})$ から外積を計算して求める:
\begin{equation}
\bm{n}_i = \frac{\bm{v}_i \times \bm{v}_{i+1}}{|\bm{v}_i \times \bm{v}_{i+1}|}
\end{equation}

これらの入力を用いて、複数視線方向の透視投影画像に対する自己位置推定を反復的に行うことで、点特徴および線特徴に基づく目的関数を最小化する。
最終的に得られるカメラの回転行列 $\bm{R}$ と並進ベクトル $\bm{t}$ は、世界座標系の三次元点をカメラ座標系に変換するための変換行列である。


\subsection{座標系変換と投影}

メッシュ頂点の世界座標
$\bm{v}_{\rm world}^{(i)} \in \mathbb{R}^3$($i=1,2,3$)を,
カメラの回転行列 $\bm{R}$ および並進ベクトル $\bm{t}$ を用いて,
次式によりカメラ座標系へ変換する。

\begin{equation}
\bm{v}_{\rm cam}^{(i)} = \bm{R}\bm{v}_{\rm world}^{(i)} + \bm{t}
\end{equation}

変換後の頂点座標を用いて,
メッシュの重心方向ベクトル $\bm{c}_{\rm cam}$ を次式で求める。

\begin{equation}
\bm{c}_{\rm cam} = \frac{1}{3} \sum_{i=1}^{3} \bm{v}_{\rm cam}^{(i)}
\end{equation}

本研究では,各メッシュの重心方向にカメラ視線を向けることで,
当該メッシュが透視投影画像の中心付近に投影されるよう制御する。
これにより,射影歪みを抑えた状態でテクスチャを取得することが可能となる。

この目的のため,メッシュ重心方向 $\bm{c}_{\rm cam}$ を視線方向とする
射影用回転行列 $\bm{R}_{\rm proj}$ を構成する。
具体的には,重心方向を前方軸とし,
上方向ベクトルとの外積により左右方向を定めることで,
右手系の直交基底を形成する。
得られた直交基底を列ベクトルとして並べることで,
射影用回転行列を定義する。

この回転行列を用いて,
カメラ座標系上の頂点を回転させる。

\begin{equation}
\bm{v}_{\rm proj}^{(i)} = \bm{R}_{\rm proj}^{\top} \bm{v}_{\rm cam}^{(i)}
\end{equation}

回転後の3次元点をカメラ内部パラメータ行列 $\bm{K}$ を用いて画像座標系へ投影する。
得られる画素座標が画像範囲内に収まらない場合には,画像中心からの最大距離に基づいて,
透視投影画像のサイズおよび投影スケールを調整する。
この処理により,メッシュが画像外に切り落とされることを防ぎ,単一の透視投影画像からメッシュ全体のテクスチャを確実に取得できる。

\subsection{テクスチャ候補の評価}
距離が遠いテクスチャや、正面方向から撮影していないテクスチャはテクスチャの視覚的品質劣化の原因となるため、テクスチャ割り当ての条件として、以下の物理量を計算する。

\begin{itemize}
  \item メッシュ重心までの距離
  \begin{equation}
  d = \|\bm{c}_{\rm cam}\|
  \end{equation}

  \item メッシュ重心方向ベクトル $\bm{c}_{\rm cam}$ とメッシュ法線ベクトル $\bm{n}$ のなす角
  \begin{equation}
    \begin{aligned}
    \bm{n} &= \frac{(\bm{v}_{\rm cam}^{(2)} - \bm{v}_{\rm cam}^{(1)}) \times (\bm{v}_{\rm cam}^{(3)} - \bm{v}_{\rm cam}^{(1)})}
    {\|(\bm{v}_{\rm cam}^{(2)} - \bm{v}_{\rm cam}^{(1)}) \times (\bm{v}_{\rm cam}^{(3)} - \bm{v}_{\rm cam}^{(1)})\|}
    \\
    \theta &= \arccos \left( \frac{\bm{c}_{\rm cam} \cdot \bm{n}}{\|\bm{c}_{\rm cam}\| \, \|\bm{n}\|} \right)
    \end{aligned}
  \end{equation}
\end{itemize}
これらの評価量を用いて、各メッシュに対して距離が近く、かつ正面に近い視点から取得された
テクスチャを優先的に選択することで、視覚的品質の高いテクスチャ割り当てを実現する。

\section{テクスチャの射影変換}

透視投影画像から得られるメッシュ対応領域は、透視歪みを含む任意形状の多角形として画像座標系上に表現される。
この領域は、撮影条件や視点方向に依存して形状が変化するため、そのままでは、後段でテクスチャ画像上の画素座標から
対応する三次元位置を一意に復元することが困難となる。

そこで本研究では、透視投影画像上で得られたメッシュ領域を、元のメッシュ形状に対応した正規化された画像座標系へ射影変換する。
これにより、テクスチャ画像上の画素座標とモデルが保持する三次元形状との対応関係を維持する。

三角形メッシュの場合は、三頂点の対応関係に基づいてアフィン変換を適用し、元の三角形形状を保ったテクスチャを生成する。
一方、四角形メッシュの場合は、四頂点の対応関係に基づく射影変換を適用し、透視歪みを補正した長方形テクスチャを生成する。

(実際に多角形変形前後の画像を載せたい)

さらに、変換後の画像に対してマスク処理を行い、メッシュ領域のみを抽出することで、元のメッシュ形状と整合したテクスチャ画像を得る。


\section{テクスチャの視覚的品質の改善}

複数の視点から取得されたテクスチャを同一メッシュに割り当てる場合、視点差や撮影条件の違いにより、
テクスチャ境界に不連続が生じ、四角的品質が悪化してしまう。
本研究では、この不連続を低減するため、隣接する2つのテクスチャに対してブレンド処理を行う。

各テクスチャには視線方向に基づく左右情報が付与されており、これを用いて左右関係を判定する。
左側テクスチャを $I_{\mathrm{L}}$、右側テクスチャを $I_{\mathrm{R}}$とする。

ブレンドは画像全体ではなく、画像幅 $W$ の中央付近に設定した、$x$ 方向の狭い領域に限定して適用する。
画素位置 $(x, y)$ におけるブレンド結果 $I(x, y)$ は、次式で与えられる。
\begin{equation}
I(x, y)
=
w_{\mathrm{L}}(x)\, I_{\mathrm{L}}(x, y)
+
w_{\mathrm{R}}(x)\, I_{\mathrm{R}}(x, y),
\end{equation}
ただし、
\begin{equation}
w_{\mathrm{L}}(x) + w_{\mathrm{R}}(x) = 1
\end{equation}
を満たす。

ブレンド領域内では,$x$ 方向に沿って連続的に変化する重みを用いるため、
シグモイド関数に基づき右側テクスチャの重み $w_{\mathrm{R}}(x)$ を次式で定義する。
\begin{equation}
w_{\mathrm{R}}(x)
=
\frac{1}{1 + \exp\!\left(-k\,\tilde{x}\right)},
\end{equation}
ここで,$\tilde{x} \in [-1,1]$ はブレンド領域内で正規化された水平方向位置を表し、
$k$ は重みの遷移の急峻さを制御するパラメータである。
左側テクスチャの重みは
\begin{equation}
w_{\mathrm{L}}(x) = 1 - w_{\mathrm{R}}(x)
\end{equation}
として与える。

ブレンド領域外では、一方のテクスチャのみを用いることで、テクスチャ全体の解像感および幾何的整合性を維持する。
このように、境界付近に限定した方向性を持つブレンドを行うことで、テクスチャ境界に生じる輝度不連続を効果的に低減できる。

     % 特徴点マッチング, 自己位置推定

\chapter{入力画像とテクスチャの特徴点マッチング}

\section{特徴点検出およびマッチングの概要}

本研究では、自己位置推定を行うために、前節までに生成した簡易モデルのテクスチャ画像と入力画像との対応付けを行う。
具体的には、画像間で対応する特徴点を検出し、それらの対応関係を用いて、入力画像上の2次元特徴点と、3次元モデルに対応付けられたテクスチャ上の点との対応を構築することで、カメラの位置および姿勢を推定する。
なお、本研究で使用するテクスチャは側面テクスチャに限定する。
床面は同一テクスチャが連続して用いられることが多く、特徴点の識別が困難であるため、対応付けの対象から除外する。

この対応付けを実現するため、本章では特徴点検出および特徴点マッチングの処理を行う。
特徴点検出およびマッチング手法は、大きく従来手法と学習ベース手法の二つに分類される。
従来手法は、画像の輝度勾配や局所構造に基づいて特徴点および記述子を設計する手法であり、計算過程が明確であるという利点を持つ。
一方で、撮影条件や環境の変化に対する適応性には限界がある。
これに対して、学習ベース手法は、深層学習を用いて特徴点検出やマッチングの過程をデータから学習する手法である。
画像全体の文脈情報を考慮した対応付けが可能であり、テクスチャの少ない環境においても比較的安定した対応が得られる。
その一方で、従来手法と比較して計算コストが大きいという特徴がある。


\section{従来手法による特徴点マッチング}
\subsection{特徴点検出}
本研究では、代表的な局所特徴点検出手法として SIFT および AKAZE を用いる。
SIFT は、ガウシアン平滑化により構築されるスケール空間上で極値点を検出し、スケールおよび回転に対して不変な特徴点を得る手法である\cite{Lowe2004}。
各特徴点に対しては、周囲の勾配分布に基づく特徴量が計算され、高い識別性能を持つ。
AKAZE は、非線形スケール空間に基づいて特徴点を検出する手法であり、Fast Explicit Diffusion を用いることで高速な処理を可能としている\cite{Alcantarilla2013}。
記述子にはバイナリ表現が用いられ、後段のマッチング処理を効率的に行うことができる。
本研究では、入力画像および簡易モデルに対応するテクスチャ画像をグレースケール化した後、それぞれに対して特徴点検出および記述子計算を行う。
また、データベースとなるテクスチャ画像に対しては、特徴点を事前に計算しておくことで、マッチング時の計算負荷を低減している。

\subsection{特徴点マッチング} 
従来手法では、最近傍探索に基づくマッチング手法を採用する。
記述子間の距離を計算し、最も距離の近い特徴点同士を対応点として選択することで、初期的な対応点集合を得る。
SIFT のような実数値記述子に対しては KD-tree を用いた探索を行い、AKAZE のようなバイナリ記述子に対しては LSH を用いた探索を行うことで、
記述子の特性に応じた最近傍探索を実現している。

\subsection{マッチングの精度向上}
最近傍探索によって得られる対応点集合には、特徴量の類似度のみでは除去できない誤対応が含まれる可能性がある。  
そこで本研究では、複数の手法を組み合わせることで、マッチング精度の向上を図る。
まず、Lowe の比率テストを適用し、第一近傍と第二近傍の距離比が一定以下となる対応点のみを採用することで、曖昧な対応を除去する。
その後、RANSAC を用いて幾何的整合性に基づく外れ値除去を行う。対応点集合から射影変換モデルを推定し、モデルに適合しない対応点を除外することで、幾何的に整合した対応点集合を得る。
これらの処理により、自己位置推定に用いる対応点の精度を向上させ、後段の位置および姿勢推定における安定性の確保を図る。

\section{学習ベース手法による特徴点マッチング}
\subsection{特徴点検出}
学習ベース手法における特徴点検出手法として SuperPoint を用いる。
SuperPoint は、画像中の特徴点(Interest Point)とそれらに対応する記述子(Descriptor)を同時に推定する特徴点検出手法である(図\ref{four:one})。
画像全体を入力とする畳み込み型ネットワークとして構成されており、共有されたエンコーダによって特徴を抽出した後、特徴点検出用と記述子生成用の2つのデコーダに分岐することで、単一のフォワードパスで両者を出力する。
この構成により、検出と記述を個別に行う手法と異なり、両タスク間で計算および特徴表現を効率的に共有することが可能となっている\cite{DeTone2018SuperPoint}。
従来手法では、輝度勾配や局所構造といった人手設計された指標に基づいて特徴点検出が行われるのに対し、大量の画像データを用いた学習による安定した特徴点検出が可能である。

\begin{figure}[H]
  \begin{center}
    \includegraphics[width=0.5\textwidth]{figures/4/superpoint_architecture.png}
    \caption{SuperPointのアーキテクチャ. 出典:DeTone et al.(2018)\cite{DeTone2018SuperPoint} Fig.~1}
    \label{four:one}
  \end{center}
\end{figure}

本研究では、入力画像および簡易モデルに対応するテクスチャ画像をグレースケール化し、SuperPoint に入力することで特徴点検出および記述子計算を行う。
画像サイズが大きい場合には画像を複数のタイルに分割し、それぞれに対して特徴点検出を行った後、
検出結果を統合することで、画像全体の特徴点を抽出している。

\subsection{特徴点マッチング} 
学習ベース手法における特徴点マッチング手法として SuperGlue を用いる。
SuperGlue は、注意機構付きグラフニューラルネットワークと最適マッチング層の2つから構成される特徴点マッチング手法である(図\ref{four:two})。
注意機構付きグラフニューラルネットワークでは、キーポイントの位置と記述子を統合した特徴表現を自己注意および相互注意を交互に用いて段階的に更新する。
最適マッチング層では、キーポイント間のスコア行列にdustbin(いずれのキーポイントとも対応しない点を扱うための仮想ノード)を追加した上で、
Sinkhorn アルゴリズムにより最適な部分対応を推定する\cite{Sarlin2020SuperGlue}。
従来手法では、各特徴点を独立に対応付ける最近傍探索が行われるのに対し、特徴点集合全体の文脈情報を考慮した対応付けが可能である。

\begin{figure}[H]
  \begin{center}
    \includegraphics[width=0.9\textwidth]{figures/4/superglue_architecture.png}
    \caption{SuperGlueのアーキテクチャ. 出典:Sarlin et al.(2020)\cite{Sarlin2020SuperGlue} Fig.~3}
    \label{four:two}
  \end{center}
\end{figure}

\subsection{マッチングの精度向上}
SuperGlue により得られた対応点集合には、学習に基づく推定結果であるため、局所的に信頼度の低い対応点が含まれる可能性がある。
まず、従来手法における ratio test に相当する処理として、SuperGlue の matching score に基づく対応点の選別を行う。
matching score は,各特徴点対が正しく対応している可能性の高さを表す指標であり、スコアが所定の閾値以上となる対応点のみを採用することで、曖昧な対応を除去する。
そのあとは同様に、RANSAC を用いて外れ値除去を行う。これにより、学習ベース手法においても従来手法と同等の基準で誤対応を抑制し、自己位置推定に用いる対応点の精度を向上させる。      % 屋内ナビゲーション

\chapter{特徴点マッチングに基づく自己位置推定}

\section{テクスチャ画像座標から世界座標への変換}
第4章で取得した簡易モデルのテクスチャと入力画像の対応点は、いずれも画像平面上の2次元座標として表現されている。  
自己位置推定を行うためには、これらの2次元特徴点を世界座標系における3次元座標へ変換する必要がある。
本研究では、テクスチャに割り当てられた画像座標系と、対応する四角形メッシュの四隅の世界座標を用いて、画像座標系から世界座標系への変換を行う。  
以下に、2次元座標を3次元座標へ変換する手順を示す。

\subsection{2次元座標から3次元座標への変換}。 
それぞれの世界座標系における3次元座標を左上から順に時計回りに $P_0,P_1,P_2,P_3$ とすると,平面基底ベクトルは次式で表される。
\begin{equation}
  \bm{B}_1 = P_1 - P_0,\quad
  \bm{B}_2 = P_3 - P_0
\end{equation}

同様に、対応するテクスチャのUV座標を $p_0,p_1,p2,p_3$ とし、UV空間における幅および高さを次式で定義する。
\begin{equation}
  w = p_1^{(u)} - p_0^{(u)},\quad
  h = p_3^{(v)} - p_0^{(v)}
\end{equation}

入力画像上で検出された2次元特徴点をピクセル座標 $(x,y)$ とし,画像サイズを $(W,H)$ とすると、特徴点は次式によりUV空間上の正規化座標 $(u,v)\in[0,1]\times[0,1]$ に変換される。
\begin{equation}
  u = \frac{x/W - p_0^{(u)}}{w},\quad
  v = \frac{y/H - p_0^{(v)}}{h}
\end{equation}

得られた正規化係数 $u,v$ を、世界座標系における平面基底ベクトルの線型結合として用いることで、対応する3次元座標 $\bm{P}$ を次式により計算する。
\begin{equation}
  \bm{P} = P_0 + u\bm{B}_1 + v\bm{B}_2
\end{equation}
これにより、テクスチャ画像上の2次元特徴点を、対応する四角形メッシュ平面上の世界座標系3次元特徴点へ変換することができる。

\section{自己位置推定}
前節で取得した入力画像上の2次元特徴点と、対応するテクスチャ上の3次元特徴点の対応関係を用いて自己位置推定を行う。

本研究では,松下らによって提案された,直交射影誤差に基づく PnPL 問題に対する大域最適解の計算手法を用いる \cite{matsushita2024}。  
一般的な反復的最適化手法では,初期値に依存して局所最適解に陥る可能性があり,また複数の解が存在する場合にそれらを網羅的に求めることが困難である。  
さらに,使用する対応点の配置によっては反復回数が増加し,実行時間が長くなるため,拡張現実などのリアルタイム処理への適用が難しい。
これに対し,本手法は非反復的に大域最適解を求めることが可能であり,計算時間の削減と複数解候補の同時導出を実現する。

本研究では,点特徴の対応のみを入力とする自己位置推定を行う。  
入力として、入力画像上の2次元特徴点に焦点距離を加えたベクトル $(x,y,f)$ と、対応する世界座標系の3次元特徴点 $(X,Y.Z)$ を用いる。   
出力として,虚数解を除いたすべての解候補について,目的関数値 $J$ と対応する回転行列 $\bm{R}$ および並進ベクトル $\bm{t}$ が得られる。
得られた複数の解候補の中から,以下の手順により最終的な自己位置推定結果を選択する。なお、最終的な最終的な自己位置推定結果を選ぶ際に
姿勢誤差が一番小さいものを選んでいるが、理由として姿勢誤差が大きい推定結果は、画像のマッチングが極端に偏りがあるなど、に不備がある可能性が高かったからである。
\begin{enumerate}
  \item 大域最適解法により得られた解候補のうち,天地が反転している解を除外する。  
        本研究では,カメラ座標系の $Y$ 軸が世界座標系の $Z$ 軸の負方向を向くという幾何学的制約に基づき,天地反転の有無を判定する。
  \item 残った各解候補について,回転行列 $\bm{R}$ および並進ベクトル $\bm{t}$ から,カメラ中心位置 $\bm{C}$ を
        \begin{equation}
          \bm{C} = -\bm{R}^{\top} t
        \end{equation}
        により算出する。
  \item 現在位置 $\bm{C}_{\mathrm{gt}}$ が既知である場合,推定されたカメラ中心位置 $\bm{C}$ とのユークリッド距離を,位置誤差 $e_{\mathrm{pos}}$ として次式で定義する。
        \begin{equation}
          e_{\mathrm{pos}} = \lVert \bm{C} - \bm{C}_{\mathrm{gt}} \rVert_2
        \end{equation}
  \item 現在の視線方向 $\bm{f}_{\mathrm{gt}}$ が既知である場合,世界座標系における視線ベクトル $\bm{f}_{\mathrm{world}}$ を次式で定義する。
        \begin{equation}
          \bm{f}_{\mathrm{world}} = \bm{R}^{\top} (0,0,1)^{\top}
        \end{equation}
  \item 推定された視線方向 $\bm{f}_{\mathrm{world}}$ と基準となるカメラ視線方向 $\bm{f}_{\mathrm{gt}}$ とのなす角を,姿勢誤差 $e_{\mathrm{rot}}$ として次式で定義する。
        \begin{equation}
          e_{\mathrm{rot}} =
          \cos^{-1}
          \left(
            \frac{
              \bm{f}_{\mathrm{world}} \cdot \bm{f}_{\mathrm{gt}}
            }{
              \lVert \bm{f}_{\mathrm{world}} \rVert
              \lVert \bm{f}_{\mathrm{gt}} \rVert
            }
          \right)
        \end{equation}
  \item 位置誤差 $e_{\mathrm{pos}}$ および姿勢誤差 $e_{\mathrm{rot}}$ が,それぞれ所定の閾値以下となる解のみを有効な解候補として残す。
  \item 有効な解候補の中から,姿勢誤差 $e_{\mathrm{rot}}$ が最小となる解を,最終的な自己位置推定結果として採用する。
\end{enumerate}
    % 実験

\chapter{自己位置推定結果を用いた屋内ナビゲーション}

\section{屋内ナビゲーションにおける自己位置推定の課題}
実際のカメラ入力を用いてリアルタイムに動作させる場合、特徴点マッチング処理に起因する遅延により、推定位置
が端末の実際の位置とずれる可能性がある。また、屋内環境では常に十分な特徴量が得られるとは限らず、マッチング
に失敗して自己位置が更新されない状況も想定される。

\section{提案手法に基づくナビゲーションの基本方針}
本研究では、画像特徴点マッチングに基づく自己位置推定結果を用いて、屋内空間におけるナビゲーションを実現することを目的とする。
提案手法では、事前に構築された三次元モデルまたは参照データと、
カメラから取得される画像との対応付けを行うことで、端末の位置および姿勢を推定する。
推定された自己位置情報は、ユーザに対して進行方向や目的地までの誘導情報を提示するために用いられる。しかしながら、前節で述べたように、特徴点マッチングに基づく手法は計算コストが高く、リアルタイム性や安定性の観点から単独での利用には課題がある。
そのため、本研究では提案する自己位置推定手法をナビゲーションの基準情報として用いつつ、
状況に応じた補完手段を導入する方針とする。

\section{モバイル端末の自己位置推定機能による補完}
屋内環境における自己位置推定の安定性を向上させるため、本研究ではモバイル端末に搭載された自己位置推定機能を併用する。
具体的には、iOS端末において利用可能なARKitを用い、
カメラ画像とIMUセンサ情報を統合したVisual-Inertial Odometry(VIO)による自己位置推定を行う。
ARKitによる自己位置推定は、端末上でリアルタイムに実行され、
高頻度かつ連続的に位置・姿勢情報を取得できるという特徴を持つ。この処理はモバイル端末上で動作するSLAMに相当すると考えられる。
本研究では、特徴点マッチングに失敗した場合や、推定結果が得られない期間において、
ARKitによる自己位置推定結果を用いることで、ナビゲーションの連続性を維持する。

\section{自己位置推定手法の使い分けと比較方針}
本研究では、以下の三つの自己位置推定手法を用いたナビゲーションを対象とし、
それぞれの特性を比較・評価する。
1つ目は、本研究で提案する画像特徴点マッチングに基づく自己位置推定結果のみを用いる方法である。
2つ目は、モバイル端末上のSLAM(ARKit)による自己位置推定結果のみを用いる方法である。
3つ目は、両者を併用し、状況に応じて自己位置推定結果を切り替える、あるいは補完的に利用する方法である。
これらの手法について、自己位置推定精度、推定の安定性、およびナビゲーションの継続性といった観点から比較を行う。
これにより、屋内ナビゲーションにおいて自己位置推定手法をどのように使い分けることが有効であるかを明らかにする。

\section{屋内ナビゲーションシステムの構成}
屋内ナビゲーションシステムの構成について説明する
\subsection{目的地設定}
画面上にマップ画像を表示し、ユーザーはタップ操作により目的地および経路上の右左折ポイントを選択できる。選
択したポイントは、ホモグラフィー行列を用いて画像座標から世界座標に変換する。
\subsection{端末とPC間の通信}
WebSocket を用いて端末と PC を接続し、カメラフレーム、内部パラメータ、センサ情報に基づき更新された自己位
置を PC へ送信する。また、PC 側で計算された自己位置推定結果を非同期に受信し、目的地の AR 表示および自己位置
の更新に利用する。
\subsection{端末の自己位置の更新}
Apple が提供する AR 開発フレームワークである ARKit は、センサ情報と画面上の特徴点の追跡を組み合わせて、毎
フレーム端末の位置と姿勢を推定する。端末側では、この位置姿勢変化をもとに、PC から受信した自己位置情報と統合
することで自己位置を更新する仕組みを構築している。
\subsection{道案内を行うオブジェクトの描画}
目的地を示すピンオブジェクトと、現在位置から目的地方向を指し示す矢印オブジェクトを AR 空間に描画する。    % まとめ

\chapter{実験}

\section{実験準備}
表\ref{seven:one}に実験で使用した機材および実行環境示す。
\begin{table}[H]
  \centering
  \caption{実験で使用した機材および実行環境}
  \label{seven:one}
  \begin{tabular}{l|l}
    \hline
    項目 & 内容 \\
    \hline \hline
    全方位カメラ & RECOH THETA 360 \\
    全方位カメラ解像度 & 11K $(11008 \times 5504)$ \\
    全方位カメラ焦点距離 & 約1752 \\
    \hline
    OS & Windows 11 \\
    CPU & Intel Core i7-14700 \\
    GPU & NVIDIA GeForce RTX 4060 Ti \\
    メモリ & 32 GB \\
    使用言語 & Python 3.9.13 \\
    主な使用ライブラリ & PyTorch 2.5.1+cu121 ,OpenCV 4.10 \\
    \hline
    屋内ナビゲーション端末 & iPad Pro 12.9インチ (第5世代) \\
    OS (iOS) & 17.5 \\
    端末カメラ解像度 & ($1920 \times 1440$) \\
    端末カメラ焦点距離 & 約1595 \\
    \hline
    屋内ナビゲーション実装環境 & MacBook Air 13インチ \\
    使用言語 & swift 5.10 \\
    主な使用ライブラリ & ARKit, SceanKit \\
    \hline
  \end{tabular}
\end{table}

また、簡易モデルの作成および屋内ナビゲーションに関する実験は、所属する大学の C 棟 5 階において実施した。
当該フロアは廊下および複数の部屋から構成されており、単色壁面が連続するテクスチャの少ない領域が存在する。
C棟5階のフロアマップおよび、全方位画像の撮影位置を図\ref{seven:two}に示す。
全方位画像は、一部の例外を除き、およそ 4\,m 間隔で合計 $N$ 箇所において撮影した。
撮影時のカメラ高さは床面から 1.4\,m である。

ただし、当該環境には類似した外観を有する領域が多く、特徴点マッチングに必要な情報が十分に得られない箇所が存在する。
そこで本研究では、過度に環境のテクスチャ量を増加させることなく、最低限の特徴情報を得ることを目的として、
壁面の一部に A2 サイズのポスターを掲示した。
ポスターの掲示間隔はおよそ 4\,m に 1 枚程度とし、環境全体が高テクスチャ化することを避けるよう配慮した。

\begin{figure}[H]
  \centering
  \begin{tabular}{c}
      \includegraphics[width=0.4\textwidth]{figures/7/camera.png}
  \end{tabular}
  \caption{世界座標系とカメラ座標系、画像座標系の関係}
  \label{seven:two}
\end{figure}

各全方位カメラについては、2方向以上の透視投影画像を画角$90 \times 90$で生成し、
透視投影画像上の2次元座標と世界座標の3次元座標とを対応付けることで、全方位カメラの位置と姿勢を推定した(図\ref{seven:three})。
すべての全方位カメラにおいて、推定位置と実測位置のずれが$10cm$未満であることを確認している。

\begin{figure}[H]
  \centering
  \begin{tabular}{ccc}
    \includegraphics[width=0.4\textwidth]{figures/7/omni.png} &
    \includegraphics[width=0.3\textwidth]{figures/7/pers1.png} &
    \includegraphics[width=0.3\textwidth]{figures/7/pers2.png}
  \end{tabular}
  \caption{全方位画像(10番)と前後の透視投影画像}
  \label{seven:three}
\end{figure}

\section{簡易モデル生成}
\subsection{データセットおよびパラメータ設定}
図\ref{seven:four}に、ワイヤーフレーム生成の入力として用いた、
簡易モデルの床面の境界点を示す。
これらの点は、2次元マップ上で手動により入力されたものである。

\begin{figure}[H]
  \centering
  \begin{tabular}{c}
      \includegraphics[width=0.4\textwidth]{figures/7/corners.png}
  \end{tabular}
  \caption{ワイヤーフレームの床面の境界点}
  \label{seven:four}
\end{figure}

床面は三角形メッシュとして分割し、各三角形の最大面積を 2.0\,m$^2$ に制限した。
側面は四角形メッシュとして分割し、天井高は 2.3\,m とした。
サンプリング間隔については、間隔の違いがテクスチャ割り当て後の外観および自己位置推定結果に与える影響を確認するため、2\,m、3\,m、4\,m の3種類を設定した。

\subsection{ワイヤーフレーム生成結果}
図\ref{seven:five}に、これらのコーナー点を結ぶ辺に沿って所定のサンプリング間隔で補間点を生成することで構築した
ワイヤーフレームモデルを示す。サンプリング間隔は左から 2\,m、3\,m、4\,m である。
床面の三角形メッシュは、コーナー点および補間点に基づいて自動的に三角形分割されていることが確認できる。
一方、壁面の四角形メッシュについては、サンプリング間隔の変化に伴い生成されるメッシュ形状が変化していることが観察される。

\begin{figure}[H]
  \centering
  \begin{tabular}{ccc}
    \includegraphics[width=0.3\textwidth]{figures/7/wire1.png} &
    \includegraphics[width=0.3\textwidth]{figures/7/wire2.png} &
    \includegraphics[width=0.3\textwidth]{figures/7/wire3.png}
  \end{tabular}
  \caption{ワイヤーフレームの床面の境界点(サンプリング間隔:左から 4\,m, 3\,m, 2\,m)}
  \label{seven:five}
\end{figure}

\subsection{テクスチャ割り当て結果}
図\ref{seven:six}に、前節で生成したワイヤーフレームモデルに対して、全方位画像からテクスチャを割り当てた結果を示す。
なお、壁面ではなく奥行きを有する面については、誤った特徴点座標が自己位置推定に用いられることを防ぐため、あらかじめテクスチャ割り当ての対象から除外している。

\begin{figure}[H]
  \centering
  \begin{tabular}{c}
    \includegraphics[width=0.8\textwidth]{figures/7/3dmodel.png}
  \end{tabular}
  \caption{簡易モデルの全体図}
  \label{seven:six}
\end{figure}

図\ref{seven:seven}に、サンプリング間隔の異なるワイヤーフレームごとに、拡大表示した簡易モデルを示す。
いずれのサンプリング間隔においても、テクスチャは大きな位置ずれを生じることなく、
各面に対して適切に割り当てられていることが確認できる。
また、ブレンド処理を行った場合には、隣接するテクスチャ間の接続が滑らかになり、
視覚的品質の向上に寄与していることがわかる。

サンプリング間隔の違いによる全体的な再現度の差は大きくないものの、
サンプリング間隔が 4\,m の場合には、テクスチャを大きく拡大して生成する必要があるため、
引き延ばしによる劣化が確認される。
一方で、3\,m の場合には、面ごとに割り当てられるテクスチャの大きさにばらつきが生じる点が確認された。
これらの点から、視覚的品質の観点では、2\,m のサンプリング間隔が最も適していると考えられる。

\begin{figure}[H]
  \centering
  \begin{tabular}{ccc}
    \includegraphics[width=0.3\textwidth]{figures/7/4m.png}
    \includegraphics[width=0.3\textwidth]{figures/7/3m.png}
    \includegraphics[width=0.3\textwidth]{figures/7/2m.png}
  \end{tabular}
  \caption{簡易モデルの拡大図(サンプリング間隔:左から 4\,m, 3\,m, 2\,m)}
  \label{seven:seven}
\end{figure}

\section{特徴点マッチング手法の比較実験}

\subsection{実験条件}
前節で生成した簡易モデルのテクスチャと、実環境で撮影された入力画像との間で特徴点マッチングを行い、その精度を評価した。
入力データには、大学構内C棟5階の廊下環境において、屋内ナビゲーション端末を保持して歩行撮影した動画を用いた。
歩行経路と、入力動画を一部切り抜いたものを表\ref{seven:input}に示す。
動画の解像度は$1920 \times 1080$ピクセルであり、これを1 fps (frame per second) の間隔で静止画として切り出し、評価用画像セットとした。

\begin{figure}[H]
    \centering
    \begin{minipage}[c]{0.35\textwidth}
        \centering
        \includegraphics[width=\linewidth]{figures/7/directions.png}
    \end{minipage}
    \hspace{1cm}
    \begin{minipage}[c]{0.35\textwidth}
        \centering
        \begin{subfigure}{\linewidth}
            \centering
            \includegraphics[width=\linewidth]{figures/7/input1.jpg}
        \end{subfigure}
        \vspace{2em}
        \begin{subfigure}{\linewidth}
            \centering
            \includegraphics[width=\linewidth]{figures/7/input2.jpg}
        \end{subfigure}
    \end{minipage}
    \caption{歩行経路と入力画像}
    \label{seven:input}
\end{figure}

屋内ナビゲーションを目的とした走行環境においては、カメラが壁面や掲示物に対して正対する状況は限定的であり、進行方向に対して壁面を斜め方向から観測する頻度が高いと考えられる。
したがって本実験では、視点変化や変形に対する頑健性を確保することを最優先としつつ、ナビゲーション用途としてのリアルタイム性も考慮してパラメータ調整を行った。
表\ref{seven:eight}にSIFTおよびAKAZE検出器の主なパラメータを示す。

\begin{table}[H]
  \centering
  \caption{従来手法(SIFT, AKAZE)のパラメータ設定}
  \label{seven:eight}
  \begin{tabular}{l|l|c|c}
    \hline
    手法 & パラメータ項目 & 標準値 & \textbf{本実験設定値} \\
    \hline \hline
    SIFT & \texttt{nOctaveLayers} & 3 & \textbf{5} \\
    & \texttt{contrastThreshold} & 0.04 & \textbf{0.03} \\
    & \texttt{edgeThreshold} & 10 & \textbf{20} \\
    \hline
    AKAZE & \texttt{threshold} & 0.001 & \textbf{0.0005} \\
    & \texttt{nOctaveLayers} & 4 & \textbf{6} \\
    \hline
  \end{tabular}
\end{table}

一方、深層学習ベースの手法であるSuperPointおよびSuperGlueについても、同様に計算効率と精度のトレードオフを考慮した設定を用いた。
表\ref{seven:nine}にSuperPointおよびSuperGlueのパラメータを示す。

\begin{table}[H] 
  \centering 
  \caption{深層学習手法(SuperPoint, SuperGlue)のパラメータ設定} 
  \label{seven:nine} 
  \begin{tabular}{l|l|c|l} 
    \hline 
    モデル & パラメータ項目 & 標準値 & \textbf{本実験設定値} \\ 
    \hline \hline 
    SuperPoint & \texttt{max\_keypoints} & \mbox{無制限} & \textbf{1024} \\
    & \texttt{keypoint\_threshold} & 0.005 & \textbf{0.001} \\ 
    \hline 
    SuperGlue & \texttt{weights} & --- & \textbf{outdoor} \\ 
    & \texttt{sinkhorn\_iterations} & 20 & \textbf{5} \\
    \hline 
  \end{tabular} 
\end{table}

SuperGlueの重みパラメータ(weights)については、屋内環境での実験であるものの、屋外データで学習された \texttt{outdoor} モデルを採用した。
一般に屋内では \texttt{indoor} モデルが推奨されるが、本研究の対象である「大学構内の廊下」においては、以下の3点の理由から屋外モデルの方が適していると判断した。

\begin{enumerate}
    \item \textbf{環境の構造的特徴:}
    一般的な屋内学習データ(ScanNetなど)は、狭い部屋に置かれた家具や雑貨などの「豊富な模様(テクスチャ)」を頼りに学習されている傾向がある。
    一方、本実験の環境である廊下は、模様が少なく、長い白壁や天井のラインといった「直線的な構造」が大部分を占めている。この特徴は、複雑な室内よりも、むしろビルの外観や道路といった屋外環境の構造に近い。

    \item \textbf{視点変化への強さ:}
    ナビゲーション中のカメラ映像は、壁に近づいたり、斜め方向から撮影したりと、見え方が大きく変化する。
    屋外モデルは、建物を様々な角度から撮影したデータで学習されているため、こうした大きな視点変化に対して頑健である。細かい模様に頼りがちな屋内モデルと比較して、廊下のような大まかな空間構造を捉える能力に長けていると考えられる。

    \item \textbf{予備実験による裏付け:}
    実際に本環境のデータを用いて比較実験を行ったところ、屋内モデルを使用した場合よりも、屋外モデルを使用した方が安定して多くのマッチング点が得られる傾向が確認された。
\end{enumerate}

マッチング後の誤対応除去(幾何学的検証)については、各手法の純粋な性能を公平に比較するため、処理条件を統一した。 
ホモグラフィ行列の推定アルゴリズムには、従来のRANSACと比較してパラメータ依存性が低く、
ノイズに対してロバストな MAGSAC++ を採用した\cite{magsac}。 
再投影誤差の許容閾値は、入力が高解像度画像であることを考慮し 5.0 pixel に設定した。 
また、計算コストの増大を防ぎ推定精度を安定化させるため、検出された全マッチング点を用いるのではなく、
マッチングスコア上位の 100点 を選抜して幾何学的検証に入力する構成とした。

\subsection{実験結果}
図\ref{seven:ten}に、従来手法(SIFT, AKAZE)および学習ベースの手法(SuperPoint+SuperGlue)を用いた
特徴点マッチングの定性的な評価結果の一部を示す。
図の左列に示すように、生成モデルのテクスチャと入力画像との間の視点差が小さいケースにおいては、
3手法ともに十分なインライア数を確保し、安定したマッチングに成功した。各手法ごとの詳細な傾向は以下の通りである。

\begin{itemize} 
  \item \textbf{SIFT:} 
  視点角度が類似している条件下であれば、撮影距離が離れている場合であってもマッチングが可能であった。
  しかし、壁面を斜めから見るなど視点変化が大きくなると、マッチングに失敗する事例が多発した。

  \item \textbf{AKAZE:} 
  SIFTと同様の傾向を示したが、SIFTと比較して視点変化に対する耐性がわずかに優れており、
  より広い角度範囲でマッチングを維持できる傾向が見られた。

  \item \textbf{SuperPoint+SuperGlue:} 
  従来手法と比較して、視点変化に対する頑健性が著しく向上した。
  極端にテクスチャが乏しい領域や、ドア等類似構造しか含まない領域といった高難度なケースを除き、
  ほぼ全てのフレームにおいてマッチングに成功した。
\end{itemize}

\begin{figure}[H]
  \centering
  \begin{tabular}{cc}
    \includegraphics[width=0.5\textwidth]{figures/7/sift1.jpg}
    \includegraphics[width=0.5\textwidth]{figures/7/sift2.jpg}\\
    \includegraphics[width=0.5\textwidth]{figures/7/akaze1.jpg}
    \includegraphics[width=0.5\textwidth]{figures/7/akaze2.jpg}\\
    \includegraphics[width=0.5\textwidth]{figures/7/deep1.jpg}
    \includegraphics[width=0.5\textwidth]{figures/7/deep2.jpg}\\
  \end{tabular}
  \caption{特徴点マッチング結果の比較\\上段から SIFT、AKAZE、SuperPoint+SuperGlue を示し、画像内左はテクスチャ画像,右は入力画像である。}
  \label{seven:ten}
\end{figure}

表\ref{seven:eleven}に、各手法におけるマッチング成功枚数および1フレームあたりの平均処理時間を示す。
マッチング成功数については、学習ベースの手法(SuperPoint+SuperGlue)が従来手法(SIFT, AKAZE)と比較して圧倒的に高い値を示した。
この主な要因は、学習ベース手法が持つ視点変化に対する高い頑健性にある。
テクスチャ情報が有効に撮像されているフレームにおいては、角度やスケールの変化に関わらず、
ほぼ全てのケースでマッチングに成功していることが確認された。

一方、平均処理時間に関しては、本実験の設定下ではAKAZEが最も高速な結果となった。
学習ベースの手法は推論処理を伴うため計算コストが高い傾向にある。
しかし、本実験では評価のために全フレームに対して探索を行っているが、実際のナビゲーション運用時においては、
一度自己位置が推定された後は近傍のテクスチャのみを探索対象とする処理を導入することで、計算時間は大幅に削減可能であると考えられる。
加えて、学習ベースの手法は低解像度画像に対しても高い特徴記述能力を維持する特性があるため、入力解像度をさらに低減させることによる高速化の余地も残されている。

以上の結果より、視点変化が大きく特徴点の抽出が困難な屋内ナビゲーション環境においては、
処理速度の課題を運用上の工夫で吸収可能であることを踏まえると、ロバスト性に優れる学習ベースの手法が最も適しているといえる。
したがって、以降の章では特徴点マッチングに学習ベースの手法を採用する。

\begin{table}[H]
  \centering
  \caption{各手法におけるマッチング成功枚数と平均処理時間}
  \label{seven:eleven}
  \begin{tabular}{l|c|c}
    \hline
    手法 & マッチング成功枚数(全 156 枚中) & 平均処理時間[ms] \\
    \hline \hline
    SIFT & 12 & 4628 \\
    AKAZE & 20 & \textbf{147} \\
    SuperPoint+SuperGlue & \textbf{89} & 4112 \\
    \hline
  \end{tabular}
\end{table}

\section{特徴点マッチングに基づく自己位置推定結果の評価}
図\ref{seven:twelve}に、前節で述べた学習ベースの手法による特徴点マッチングを用いた自己位置推定の結果を示す。
同図では、マッチングに成功したフレームについて、推定された自己位置および正規化された視線方向ベクトルをプロットしている。 
なお、実数解が得られなかった場合や、天地反転などの幾何学的に不整合な解しか得られなかった場合は、
有効な解が算出されなかったものとみなし、結果から除外している。
本実験では、図\ref{seven:seven}に示したサンプリング間隔の異なる3種類の簡易モデルを用いて比較を行った。

\begin{figure}[H]
  \centering
  \begin{tabular}{ccc}
    \includegraphics[width=0.3\textwidth]{figures/7/deep4m.png}
    \includegraphics[width=0.3\textwidth]{figures/7/deep3m.png}
    \includegraphics[width=0.3\textwidth]{figures/7/deep2m.png}
  \end{tabular}
  \caption{自己位置推定結果(サンプリング間隔:左から 4\,m, 3\,m, 2\,m)}
  \label{seven:twelve}
\end{figure}

実験の結果、学習ベース手法の場合いずれのサンプリング間隔においても、多くのフレームで自己位置推定が可能であることが確認された。 
しかし、サンプリング間隔4\,mの場合については、3\,mおよび2\,mの場合と比較して結果に顕著な差異が見られた。
具体的には、特徴点マッチングには成功しているものの、前後のフレーム間で姿勢が不連続に変化したり、推定位置が経路から大きく逸脱したりする誤推定のフレームが多く確認された。
全158フレームのうち、このような誤推定を除外した有効なフレーム数は、サンプリング間隔4\,mで93枚、3\,mで108枚、2\,mで105枚であった。 
3\,mと2\,mはいずれも高い推定成功率を示し、その差はわずかであった。 そのため、テクスチャの視覚的品質と自己位置推定の安定性を総合的に評価し、
本研究ではサンプリング間隔2\,mが最も適していると判断した。

\section{屋内ナビゲーションにおける自己位置推定結果の比較}

\subsection{実験条件}
本実験では、屋内ナビゲーションにおける自己位置推定手法の差異が、最終的なナビゲーションの成否や精度に与える影響を検証するため、
以下の3つの条件を設定し、比較を行った。

\begin{enumerate}
    \item \textbf{画像マッチング単独手法:}
    PCサーバー上での画像特徴点マッチングに基づく自己位置推定のみを用いる条件。
    \item \textbf{VIO(Visuakl-Inertial Odometory)単独手法:}
    モバイル端末に搭載されたVIOベースの自己位置推定のみを用いる条件。
    \item \textbf{併用手法:}
    両者を併用し、VIOによる追従と画像マッチングによる補正を組み合わせた条件。
\end{enumerate}

歩行経路および経由地は、前節の図\ref{seven:input}に示したものと同一である。
ただし、本実験では入力データとして録画済みの動画ではなく、リアルタイムに取得されるカメラ画像(静止画フレーム)を用いた。 
ナビゲーションの具体的な手順は以下の通りである。

\begin{enumerate}
    \item \textbf{開始処理:}
    開始地点にて静止し、初回の自己位置推定を行う。位置が特定され次第、第一の経由地へ向けて歩行を開始する。
    \item \textbf{経由地判定:}
    設定された経由地の半径2\,m以内に到達した時点で、システムは当該地点への到達と判定し、直ちに次の経由地へのナビゲーションに切り替える。
    \item \textbf{終了判定:}
    最終目的地の半径2\,m以内に到達した時点でナビゲーションを終了とする。
    \item \textbf{リカバリ処理:}
    PCサーバーによる位置推定が10秒間連続して成功しなかった場合、トラッキングが喪失したと判定し、再度初期位置推定処理を実行する。
\end{enumerate}

PCサーバー側で行う2回目以降の自己位置推定(トラッキング処理)においては、計算効率と精度向上のため、
モバイル端末から得られる直前の推定位置および視線ベクトルを事前情報として利用する。
この際、誤マッチングによる外れ値を排除するため、許容誤差の閾値を位置については1\,m、姿勢(角度)については10$^{\circ}$と設定した。 
また、マッチング対象の絞り込みとして、特徴点マッチングを行う参照画像は、現在の推定端末位置から半径5\,m以内に存在する画像のみを探索対象とした。

\subsection{実験結果}
図13に実験結果を示す。
まず、自己位置推定にモバイル端末のVIO(Visual Inertial Odometry)のみを用いた場合、
時間の経過に伴い、推定位置が真値から徐々に乖離していくドリフト現象が確認された。 
この要因として、センサノイズの累積に加え、実験環境の視覚的特徴が影響していると考えられる。
本実験環境は白壁などのテクスチャに乏しい平面が多く、VIOの処理において十分な数の特徴点を安定して追跡することが困難な区間が存在した。
その結果、視覚情報による自己位置の補正が十分に機能せず、累積誤差が増大する結果となった。

次に、PCサーバーの画像特徴点マッチングのみを用いた場合、マッチングに成功したフレームにおいては高精度な自己位置が得られた。
しかし、自己位置を連続的かつリアルタイムに取得することは困難であった。
その要因として、環境のテクスチャ不足によるマッチングの不成立に加え、特徴点抽出・照合にかかる計算処理時間、および画像データの送受信に伴う通信タイムラグが挙げられる。
これらが複合的に影響することで、推定結果の更新が断続的になったり、移動に対して提示が遅れたりする現象が生じ、
滑らかな移動が求められるナビゲーション用途においては、即時性と連続性の欠如が課題として確認された。

一方、モバイル端末のVIOと画像特徴点マッチングによる推定を併用した手法では、
VIOによる連続的なトラッキングを行いつつ、画像マッチングによる高精度な絶対位置情報を適宜参照することで、
VIO単独時に見られた累積誤差(ドリフト)を効果的に補正できることが確認された。 
これにより、特徴の少ない環境下においても、自己位置推定における\textbf{精度の安定性}と\textbf{時間的な連続性}の両立が可能となり、
本システムが屋内ナビゲーションの実運用において有効であることが示唆された。     % 謝辞

%引用(著者アルファベット順にする)
\begin{thebibliography}{3}

\bibitem{IDC_DT_Survey2024}
IDC Japan.
Digital Twin / IoT / AI 技術調査レポート(概要).
IDC Japan, 2024.

\bibitem{GlobalGrowthInsights_DT}
Global Growth Insights.
Digital Twin Technology Market Trends 2024-2033.
Global Growth Insights, 2024.

\bibitem{ORB_SLAM}
Raul Mur-Artal, J. M. M. Montiel, and Juan D. Tard\'os.
ORB-SLAM: A Versatile and Accurate Monocular SLAM System.
IEEE Transactions on Robotics,
vol. 31, no. 5, pp. 1147--1163, 2015.

\bibitem{ORB_SLAM2}
Raul Mur-Artal and Juan D. Tard\'os.
ORB-SLAM2: An Open-Source SLAM System for Monocular, Stereo, and RGB-D Cameras.
IEEE Transactions on Robotics,
vol. 33, no. 5, pp. 1255--1262, 2017.

\bibitem{CNN_SLAM}
Keisuke Tateno, Federico Tombari, and Nassir Navab.
CNN-SLAM: Real-Time Dense Monocular SLAM with Learned Depth Prediction.
Proceedings of the IEEE Conference on Computer Vision and Pattern Recognition (CVPR),
pp. 6243--6252, 2017.

\bibitem{Cadena2016}
Cesar Cadena, Luca Carlone, Henry Carrillo, Yasir Latif, Davide Scaramuzza,
Jos\'e Neira, Ian Reid, and John J. Leonard.
Past, Present, and Future of Simultaneous Localization and Mapping:
Toward the Robust-Perception Age.
International Journal of Robotics Research,
vol. 35, no. 14, pp. 1309--1332, 2016.

\bibitem{ImageLocalizationSfM}
Torsten Sattler, Bastian Leibe, and Leif Kobbelt.
Fast Image-Based Localization using Direct 2D-to-3D Matching.
IEEE International Conference on Computer Vision (ICCV),
pp. 667--674, 2011.

\bibitem{ImageRetrievalLocalization}
Herv\'e J\'egou, Matthijs Douze, and Cordelia Schmid.
Aggregating Local Descriptors into a Compact Image Representation.
IEEE Conference on Computer Vision and Pattern Recognition (CVPR),
pp. 3304--3311, 2010.

\bibitem{IndoorSurvey}
Yuan Liu, Wei Dong, Yifan Liu, and Xiaohua Tian.
Image-Based Indoor Localization: A Survey.
IEEE Communications Surveys \& Tutorials,
vol. 22, no. 4, pp. 2923--2955, 2020.

\bibitem{IndoorLiDARNav}
J. Zhang and S. Singh.
LOAM: Lidar Odometry and Mapping in Real-time.
Robotics: Science and Systems (RSS),
pp. 1--9, 2014.

\bibitem{2DMapLocalization}
Paul Biber and Wolfgang Straßer.
The Normal Distributions Transform: A New Approach to Laser Scan Matching.
IEEE/RSJ International Conference on Intelligent Robots and Systems (IROS),
pp. 2743--2748, 2003.

\bibitem{SimplifiedModelLocalization}
Stephan Kohlbrecher, Johannes Meyer, Thomas Graber, et al.
Hector SLAM: Real-Time SLAM with a Single Laser Scanner.
IEEE International Conference on Robotics and Automation (ICRA),
pp. 3987--3994, 2011.

\bibitem{Schindler2007}
Andreas Schindler, Jan-Michael Frahm, and Marc Pollefeys.
City-Scale Location Recognition.
IEEE Conference on Computer Vision and Pattern Recognition (CVPR),
pp. 1--8, 2007.

\bibitem{Sattler2011}
Torsten Sattler, Bastian Leibe, and Leif Kobbelt.
Fast Image-Based Localization using Direct 2D-to-3D Matching.
IEEE International Conference on Computer Vision (ICCV),
pp. 667--674, 2011.

\bibitem{Shewchuk1996Triangle}
Jonathan R. Shewchuk,
Triangle: Engineering a 2D Quality Mesh Generator and Delaunay Triangulator,
in \textit{Applied Computational Geometry: Towards Geometric Engineering},
First ACM Workshop on Applied Computational Geometry,
Lecture Notes in Computer Science, Vol. 1148, pp. 203--222, 1996.

\bibitem{Sugaya2024}
松下侑聖, 菅谷保之, 直交射影に基づく点特徴と線特徴を用いたハイブリッドなカメラ姿勢推定,
第27回画像の認識・理解シンポジウム, 2024年8月.

\bibitem{Lowe2004}
D. G. Lowe,
Distinctive Image Features from Scale-Invariant Keypoints,
International Journal of Computer Vision,
Vol. 60, No. 2, pp. 91--110, 2004.

\bibitem{Alcantarilla2013}
P. F. Alcantarilla,J. Nuevo,A. Bartoli,
Fast Explicit Diffusion for Accelerated Features in Nonlinear Scale Spaces,
British Machine Vision Conference (BMVC), 2013.

\bibitem{DeTone2018SuperPoint}
D.DeTone, T.Malisiewicz, and A.Rabinovich,
SuperPoint: Self-Supervised Interest Point Detection and Description,
arXiv preprint arXiv:1712.07629, 2017.

\bibitem{Sarlin2020SuperGlue}
P.E.Sarlin, D.DeTone, T.Malisiewicz, and A.Rabinovich,
SuperGlue: Learning Feature Matching with Graph Neural Networks,
IEEE/CVF Conference on Computer Vision and Pattern Recognition (CVPR)*, 2020.

\bibitem{matsushita2024}
松下侑聖, 
直交射影に基づく PnPL 問題に対する大域最適解の計算,
pp. 4--11, 2024.

\bibitem{AppleARKitTracking}
Apple Inc.,
Managing Session Life Cycle and Tracking Quality,
Apple Developer Documentation,
\url{https://developer.apple.com/documentation/arkit/managing_session_lifecycle_and_tracking_quality},
(参照日:2026年1月).

\bibitem{AppleARKitWorldTracking}
Apple Inc.,
Understanding World Tracking,
Apple Developer Documentation,
\url{https://developer.apple.com/documentation/arkit/world_tracking/understanding_world_tracking},
(参照日:2026年1月).

\bibitem{magsac}
D. Barath, J. Matas, and J. Noskova, 
MAGSAC: marginalizing sample consensus,
IEEE/CVF Conference on Computer Vision and Pattern Recognition (CVPR), 
2019, pp. 10197--10205.

%  \bibitem{bib:sample}加藤博一, ``拡張現実感システム構築ツールARToolKitの開発'',
% 電子情報通信学会研究技術報告, PRMU2001-232, Vol. 101, No. 652, pp. 79--86, 2002.
% 
%  \bibitem{IMA}今泉一崇, ``円形マーカーを用いた複合現実感システムの構築'',
% 豊橋技術科学大学卒業論文, 2008年度.
% 
%  \bibitem{KOBA}小林大起, ``複数の円形マーカーを用いた複合現実感システム
% の構築'', 学士学位論文, 2009年度.
% 
%  \bibitem{IMAKOBA}今泉一崇, 小林大起, 菅谷保之, ``複数の円形マーカーを用いた複合現実感システムの構築'', 第16回画像センシングシンポジウム(SSII10)講演論文集,
% 	 pp. IS2--19, 2010.

\end{thebibliography}
 % 参考文献

\end{document}
