\chapter{はじめに} \label{sec:chapter}

\section{研究背景} \label{sec:background}
近年、現実空間とデジタル空間を連携させるデジタルツインの概念が注目されている。
デジタルツインの社会実装に関する既存の調査によると、
国内企業の約7割がデジタルツインを導入済み、または導入を検討していると報告されている\cite{IDC_DT_Survey2024}。
特に、屋内空間をデジタル上に再現して利用者の位置を推定することで、それに応じた案内や情報提示を行う技術への期待が高まっている。

しかしながら、同時に市場レポートでは約41\%の企業が導入にあたって予算制約を主な阻害要因として挙げており\cite{GlobalGrowthInsights_DT}、
デジタルツイン導入への関心は高い一方で、3次元計測機器や専用センサーへの投資、システム構築や維持にかかるコストが
導入における大きな障壁となっていることが示されている。
このような社会的背景から、新たな設備投資を必要とせず、カメラで取得した画像のみから屋内環境の3次元モデルを生成する手法が望まれてきた。

しかし、屋内環境は壁や床、天井といった単調な構造が多く、特徴に乏しい場合が多い。
そのため、画像情報に基づいて3次元構造を推定する際には、十分な特徴点が得られず、空間構造を正確に復元することが困難である。
特に、Visual SLAM に代表されるような、自己位置推定を行いながら高精度なマップ生成を同時に行う手法では、
特徴点の不足が自己位置推定の不安定化や、環境マップ品質の低下につながることが指摘されており\cite{ORB_SLAM, Cadena2016}、
画像のみを用いて屋内環境の高精度な3次元モデルを安定して構築することは、依然として大きな課題となっている。

一方で、屋内環境における道案内などのナビゲーション用途では、
必ずしも幾何学的に高精度な3次元モデルや、精密な自己位置推定が常に必要であるとは限らない。
経路案内や現在位置の把握といった目的においては、空間の大まかな構造を表現できる3次元モデルが得られれば十分である場合も多いと考えられる。
このような背景を踏まえ、本研究では屋内ナビゲーション用途に必要な情報に着目し、
高精度な3次元形状復元に依存しない、特徴の乏しい環境にも適用可能な空間表現と、画像に基づく自己位置推定手法について検討する。


\section{研究目的} \label{sec:purpose}

本研究の目的は、特徴に乏しく Visual SLAM の適用が困難な屋内環境において、
ナビゲーション用途に求められる実用的な精度と安定性を有する自己位置推定手法を確立することである。
特に、高精度な自己位置推定を必ずしも前提としない屋内ナビゲーションを対象とし、
既存の建物情報を活用することで、設備投資を抑えつつ安定した動作の実現を目指す。

具体的には、2次元マップから生成したワイヤーフレームモデルに全方位画像から取得したテクスチャを付与した簡易的な3次元モデルを作成し、
入力画像とモデル上のテクスチャとの特徴点マッチングに基づいて自己位置を推定する手法を提案する。

本手法のアプローチには、大きく二つの特徴がある。

第一に、導入障壁が低い点である。
提案手法は、建物に既存の2次元マップと全方位画像という、比較的容易に取得可能な情報のみを用いて環境モデルを構築する。
そのため、高価な3次元計測機器による事前の精密スキャンや、ビーコン・マーカーなどの設備を環境側に設置する必要がなく、
デジタルツイン導入の課題となっているコストと手間を大幅に削減できる。

第二に、特徴の乏しい環境における安定性と、ナビゲーションの継続性を重視した設計思想である。
公共施設やオフィスビルでの経路案内においては、ミリメートル単位の厳密な自己位置推定精度よりも、
自己位置を見失わずに追跡し続ける「安定性」が極めて重要となる。
一般的な Visual SLAM は、特徴が豊かな環境では精密な自己位置推定が可能だが、
単調な壁面など特徴が乏しい環境では、特徴点不足によりトラッキングが破綻しやすい。
本研究では、空間の大まかな構造を表現した簡易モデルを参照することでこの問題を回避する。
これにより、厳密な推定精度よりも、特徴の乏しい環境下であっても安定して自己位置推定を行うことで、
ナビゲーション動作を継続させることを目指す。

以上より、本研究は高精度な三次元形状の復元や新たな設備投資に依存することなく、
画像情報のみに依存する従来の Visual SLAM では自己位置の維持が困難な特徴の乏しい環境下においても、
安定して動作可能な屋内ナビゲーション向け自己位置推定の枠組みを提示することを目的とする。


\section{関連研究}\label{sec:related}

\subsection{Visual SLAM による自己位置推定}

Visual SLAM は、カメラ画像から特徴点を抽出および追跡することで、自己位置推定と環境マップ生成を同時に行う代表的な手法である。
ORB-SLAM\cite{ORB_SLAM} や ORB-SLAM2\cite{ORB_SLAM2} に代表される手法では、ORB 特徴量を用いた高精度なトラッキングおよびループ検出により、
高精度な自己位置推定と三次元マップ生成が可能である。
また、近年では深層学習を導入した手法\cite{CNN_SLAM} も提案されており、特徴点抽出やマッチングの頑健性向上が試みられている。

一方で、Visual SLAM は十分な特徴点が安定して得られることを前提としており、
壁面や床面が単調な屋内環境では、特徴点不足によりトラッキングが不安定になることが指摘されている\cite{ORB_SLAM, Cadena2016}。
例えば、弱いテクスチャ領域を含む環境を対象とした実験では、追跡可能な特徴点数が大きく減少し、
その結果として自己位置推定軌道が真値から逸脱する様子が報告されている(図\ref{one:one})\cite{RobustWeakTextureSLAM}。

したがって、特徴点の乏しい屋内環境において、画像情報のみに依存する Visual SLAM 単独で
既知のマップ座標系上における絶対的な自己位置を高精度かつ安定して推定することは、依然として困難な課題である。

\begin{figure}[H]
    \centering
    \begin{subfigure}{0.4\textwidth}
        \centering
        \includegraphics[width=\linewidth]{figures/1/fr3snnnew.png}
    \end{subfigure}
    \begin{subfigure}{0.4\textwidth}
        \centering
        \includegraphics[width=\linewidth]{figures/1/snn1new.png}
    \end{subfigure}
    \begin{subfigure}{0.4\textwidth}
        \centering
        \includegraphics[width=\linewidth]{figures/1/fr3nnnnew.png}
    \end{subfigure}
    \begin{subfigure}{0.4\textwidth}
        \centering
        \includegraphics[width=\linewidth]{figures/1/nnn2new.png}
    \end{subfigure}
    \caption{弱いテクスチャ環境における自己位置推定結果および追跡特徴点の可視化. 出典:Y. Liu et al.(2022)\cite{RobustWeakTextureSLAM}, Fig.~4}
    \label{one:one}
\end{figure}

\subsection{画像ベースの自己位置推定手法}

環境マップ生成を行わず、既存の環境モデルとカメラ画像との対応付けにより自己位置推定を行う手法も多く提案されている。
代表的なアプローチとしては、SfM (Structure from Motion) により事前に構築した3次元点群と画像特徴量とのマッチングに基づく手法\cite{ImageLocalizationSfM} や、
大規模画像データベースを用いた画像検索に基づく位置推定手法\cite{ImageRetrievalLocalization} が挙げられる。

これらの手法は、事前に構築された環境モデルを利用するため、未知環境を探索するSLAMに比べて累積誤差の影響を受けにくく、安定した自己位置推定が可能である。
しかし、高密度かつ高精度な3次元モデルやデータベースの事前構築が前提となる。
そのため、テクスチャが単調な環境では特徴点マッチングに基づくモデル生成自体が困難である点や、
点群構築のために大規模な撮影や計測が必要となり、導入・運用面での負担が大きいという課題がある。

\subsection{簡易3次元モデルを用いた位置推定}

高精度な3次元形状復元に依存せず、簡易的な三次元環境モデルや幾何学的制約を用いて自己位置推定を行う試みも報告されている。

例えば、Sattlerら\cite{Sattler2011}は、事前に撮影された画像群に対して SfM (Structure from Motion) を適用し、
3次元空間内の特徴点群のみを抽出・保存することで、軽量な環境マップを構築している。
彼らは、このスパースな3次元点と入力画像の特徴量との直接的な対応付けを行うことで、
高密度なモデルを持たずとも高速かつ高精度な位置推定が可能であることを示した。

また、都市環境などを対象とした Schindlerら\cite{Schindler2007}の研究では、
建物のファサードが持つ幾何学的な繰り返しパターンや、空間内の主要な平面構造に着目している。
この手法では、環境全体を厳密に復元するのではなく、位置特定に有効な特徴を選択的にデータベース化したり、
建物を鉛直平面として近似するなどの幾何学的制約を利用することで、効率的な自己位置推定を実現している。

これらの手法は、必ずしも高密度な3次元形状復元を行わずとも、
特徴点の配置や簡易的な幾何情報のみで実用的な自己位置推定が可能であることを示している点で重要である。

本研究は、こうした「簡易モデルによる推定」というアプローチを踏襲しつつ、
モデル生成のソースとして、既に存在する「2次元マップ」を利用する点に独自性がある。
全方位画像を用いて効率的にテクスチャを付与した簡易3次元モデルを構築することで、現地での大規模な事前撮影やSfMによる点群生成プロセスを省略し、
特徴点が乏しい屋内環境においても低コストで導入可能なナビゲーション手法の実現を目指すものである。

\section{本論文の構成}
本論文の構成を以下に示す。
第2章では、本研究の基盤となる簡易3次元モデルの生成手法について説明する。
第3章では、入力画像と簡易モデルのテクスチャ間の特徴点マッチング、およびその結果に基づく自己位置推定手法について説明する。
第4章では、自己位置推定結果を用いた屋内ナビゲーション手法について説明する。
第5章では、提案手法により実際に生成された簡易3次元モデルおよび自己位置推定結果を示し、
推定精度の屋内ナビゲーションにおける有効性を検証するために行った実験について説明する。


