\chapter{簡易モデルへのテクスチャ割り当て}

\section{全方位画像から透視投影画像生成}

\subsection{全方位画像を用いたテクスチャ取得の方針}
3次元モデルへのテクスチャ割り当てにおいては、モデル表面を十分に覆う視点から撮影された画像を効率的に取得することが重要である。
一般的な透視カメラを用いる場合、多方向のテクスチャ情報を得るためには、カメラ姿勢を変更しながら多数の画像を撮影する必要があり、
撮影および管理に関わるコストが増大するという課題がある。

そこで本研究では、単一の撮影によって全周囲の視覚情報を取得可能な全方位カメラを用いる。
全方位画像は、カメラ中心を基準とした全方向の視線情報を一枚の画像として保持しており、
幾何学的変換を行うことで、任意の視線方向に対応する透視投影画像を生成できる。
この特性により、複数視点から撮影した場合と同等の画像群を、高い効率で取得することが可能となる。
(複数方向の透視投影画像を生成する図を示した方がわかりやすい)

\subsection{透視投影画像変換}
本研究では、全方位画像を球面上の輝度分布として扱い、
透視投影画像の各画素に対応する視線方向を幾何学的に定義することで、
全方位画像から任意視線方向の透視投影画像を生成する。

透視投影画像の幅および高さをそれぞれ $W_p, H_p$ とし、
画像上の画素座標を $(u, v)$ とする。
透視投影面を $z=1$ に固定すると、画素 $(u, v)$ に対応する視線ベクトル $\bm{d}$ は
次式で与えられる。
\begin{equation}
\bm{d} =
\begin{pmatrix}
(u - W_p/2)\,\Delta x \\
(v - H_p/2)\,\Delta y \\
1
\end{pmatrix}
\end{equation}

ここで、$\Delta x, \Delta y$ はそれぞれ水平方向および垂直方向の画素間隔を表し、
水平画角 $\Theta$、垂直画角 $\Phi$ を用いて次式で定義される。
\begin{equation}
\Delta x = \frac{2 \tan(\Theta/2)}{W_p}, \quad
\Delta y = \frac{2 \tan(\Phi/2)}{H_p}
\end{equation}

生成したい透視投影画像の視線方向を表す回転行列を $\bm{R}$ とすると、
回転後の視線ベクトル $\bm{d}'$ は次式で与えられる。
\begin{equation}
\bm{d}' = \bm{R}\bm{d}
\end{equation}

回転後の視線ベクトル $\bm{d}' = (d'_x, d'_y, d'_z)^\top$ を球面座標系へ変換し、
方位角 $\Theta$ および仰角 $\Phi$ を次式で求める。
\begin{equation}
\Theta = \arctan2(d'_x, d'_z)
\end{equation}
\begin{equation}
\Phi = -\arctan\!\left(\frac{d'_y}{\sqrt{d'^2_x + d'^2_z}}\right)
\end{equation}

全方位画像(正距円筒画像)の幅および高さをそれぞれ $W, H$ とすると、
球面座標 $(\Theta, \Phi)$ に対応する全方位画像上の画素座標 $(x, y)$ は
次式で与えられる。
\begin{equation}
x = W\left(\frac{\Theta}{2\pi} + \frac{1}{2}\right)
\end{equation}
\begin{equation}
y = H\left(\frac{1}{2} - \frac{\Phi}{\pi}\right)
\end{equation}


\section{座標系の定義}

\subsection{カメラ座標系}
カメラ座標系はカメラの焦点位置を原点とし、
光軸方向を $z$ 軸、
水平方向右向きを $x$ 軸、
鉛直方向下向きを $y$ 軸と定める。
モデル床面を $X$-$Y$ 平面、法線方向を $Z$ 軸とする世界座標系で表された三次元点 $\bm{p}_w$ は、
カメラの回転行列 $\bm{R}$ と並進ベクトル $\bm{t}$ を用いて、
次式によりカメラ座標系上の点 $\bm{p}_c$ に変換される。

\begin{equation}
\bm{p}_c = \bm{R}\bm{p}_w + \bm{t}
\end{equation}

\subsection{スクリーン座標系}
スクリーン座標系は画像左上を原点とし、水平方向を $u$ 軸、鉛直方向を $v$ 軸と定める。
カメラ座標系上の3次元点 $\bm{p}_c = (x_c, y_c, z_c)^\top$ は、
カメラ内部パラメータ行列 $\bm{K}$ を用いて、次式によりスクリーン座標系へ射影される。

\begin{equation}
\bm{p}_s = \bm{K}\bm{p}_c
\end{equation}

ここで $\bm{p}_s = (u_s, v_s, w_s)^\top$ とすると、透視投影画像上の画素座標 $(u, v)$ は正規化により次式で得られる。

\begin{equation}
u = \frac{u_s}{w_s}, \quad v = \frac{v_s}{w_s}
\end{equation}

\subsection{カメラ内部パラメータの設定}
理想的な透視投影画像を仮定し、カメラ内部パラメータを幾何学的に設定する。
内部パラメータ行列 $\bm{K}$ は次式で表される。

\begin{equation}
\bm{K} =
\begin{pmatrix}
f_x & 0 & c_x \\
0 & f_y & c_y \\
0 & 0 & 1
\end{pmatrix}
\end{equation}

ここで $f_x$、$f_y$ はピクセル単位の焦点距離を表す。
全方位画像の幅 $W_e$ と出力透視投影画像の幅 $W_p$、水平視野角 $\theta$ および垂直視野角 $\phi$ から
$f_x = \frac{W_p}{2 \tan(\theta/2)}$、$f_y = \frac{H_p}{2 \tan(\phi/2)}$とあらわされる。
また$c_x$、$c_y$ は光軸中心を表し、透視投影画像の中心に設定する。


\section{複数の透視投影カメラによる全方位カメラ位置姿勢推定}
点特徴および線特徴を使ってカメラ位置を推定するため、直交射影に基づく点特徴と線特徴を用いたカメラ姿勢推定\cite{sugaya2024}を行う。
これは、直交射影に基づく点の共線性誤差と線の共面性誤差に対して同一の定式化を行うことで点特徴と線特徴を同時に扱い、カメラ姿勢を推定する手法である。

直交射影の共線性と共面性に基づくカメラ姿勢推定に必要な入力は以下の通りである。
\begin{itemize}
  \item $\bm{p}_a$ : 世界座標系で表現された空間点の座標。透視投影画像に映らない点は除外する。
  \item $\bm{v}_a$ : $\bm{p}_a$ に対応する画像上の特徴点座標。
  \item $\bm{d}_a$ : 世界座標系で表現された直線 $L_a$ の方向ベクトル。
  \item $\bm{r}_a$ : 世界座標系で表現された直線 $L_a$ 上の点の座標。
  \item $\bm{n}_a$ : $L_a$ に対応する画像上の直線の法線ベクトル。
\end{itemize}

$\bm{v}_a$ は透視投影画像上で $\bm{p}_a$ に対応する画素を選択することで求める。
また、画像座標 $(v_x, v_y)$ は光軸中心を原点とした座標に平行移動する:
\begin{equation}
v_x = x - \frac{W}{2}, \quad v_y = y - \frac{H}{2}
\end{equation}
さらに焦点距離 $f$ を追加し、$\bm{v}_a = (v_x, v_y, f)$ とする。

$\bm{d}_a$ は 3 次元モデルの頂点座標を用いて、以下の式で求める。
\begin{equation}
\bm{d}_i = \frac{\bm{p}_{i+1} - \bm{p}_i}{|\bm{p}_{i+1} - \bm{p}_i|}
\end{equation}

$\bm{n}_a$ は画像上の 2 点 $(\bm{v}_i, \bm{v}_{i+1})$ から外積を計算して求める:
\begin{equation}
\bm{n}_i = \frac{\bm{v}_i \times \bm{v}_{i+1}}{|\bm{v}_i \times \bm{v}_{i+1}|}
\end{equation}

これらの入力を用いて、複数視線方向の透視投影画像に対する自己位置推定を反復的に行うことで、点特徴および線特徴に基づく目的関数を最小化する。
最終的に得られるカメラの回転行列 $\bm{R}$ と並進ベクトル $\bm{t}$ は、世界座標系の三次元点をカメラ座標系に変換するための変換行列である。


\subsection{座標系変換と投影}

メッシュ頂点の世界座標
$\bm{v}_{\rm world}^{(i)} \in \mathbb{R}^3$($i=1,2,3$)を,
カメラの回転行列 $\bm{R}$ および並進ベクトル $\bm{t}$ を用いて,
次式によりカメラ座標系へ変換する。

\begin{equation}
\bm{v}_{\rm cam}^{(i)} = \bm{R}\bm{v}_{\rm world}^{(i)} + \bm{t}
\end{equation}

変換後の頂点座標を用いて,
メッシュの重心方向ベクトル $\bm{c}_{\rm cam}$ を次式で求める。

\begin{equation}
\bm{c}_{\rm cam} = \frac{1}{3} \sum_{i=1}^{3} \bm{v}_{\rm cam}^{(i)}
\end{equation}

本研究では,各メッシュの重心方向にカメラ視線を向けることで,
当該メッシュが透視投影画像の中心付近に投影されるよう制御する。
これにより,射影歪みを抑えた状態でテクスチャを取得することが可能となる。

この目的のため,メッシュ重心方向 $\bm{c}_{\rm cam}$ を視線方向とする
射影用回転行列 $\bm{R}_{\rm proj}$ を構成する。
具体的には,重心方向を前方軸とし,
上方向ベクトルとの外積により左右方向を定めることで,
右手系の直交基底を形成する。
得られた直交基底を列ベクトルとして並べることで,
射影用回転行列を定義する。

この回転行列を用いて,
カメラ座標系上の頂点を回転させる。

\begin{equation}
\bm{v}_{\rm proj}^{(i)} = \bm{R}_{\rm proj}^{\top} \bm{v}_{\rm cam}^{(i)}
\end{equation}

回転後の3次元点をカメラ内部パラメータ行列 $\bm{K}$ を用いて画像座標系へ投影する。
得られる画素座標が画像範囲内に収まらない場合には,画像中心からの最大距離に基づいて,
透視投影画像のサイズおよび投影スケールを調整する。
この処理により,メッシュが画像外に切り落とされることを防ぎ,単一の透視投影画像からメッシュ全体のテクスチャを確実に取得できる。

\subsection{テクスチャ候補の評価}
距離が遠いテクスチャや、正面方向から撮影していないテクスチャはテクスチャの視覚的品質劣化の原因となるため、テクスチャ割り当ての条件として、以下の物理量を計算する。

\begin{itemize}
  \item メッシュ重心までの距離
  \begin{equation}
  d = \|\bm{c}_{\rm cam}\|
  \end{equation}

  \item メッシュ重心方向ベクトル $\bm{c}_{\rm cam}$ とメッシュ法線ベクトル $\bm{n}$ のなす角
  \begin{equation}
    \begin{aligned}
    \bm{n} &= \frac{(\bm{v}_{\rm cam}^{(2)} - \bm{v}_{\rm cam}^{(1)}) \times (\bm{v}_{\rm cam}^{(3)} - \bm{v}_{\rm cam}^{(1)})}
    {\|(\bm{v}_{\rm cam}^{(2)} - \bm{v}_{\rm cam}^{(1)}) \times (\bm{v}_{\rm cam}^{(3)} - \bm{v}_{\rm cam}^{(1)})\|}
    \\
    \theta &= \arccos \left( \frac{\bm{c}_{\rm cam} \cdot \bm{n}}{\|\bm{c}_{\rm cam}\| \, \|\bm{n}\|} \right)
    \end{aligned}
  \end{equation}
\end{itemize}
これらの評価量を用いて、各メッシュに対して距離が近く、かつ正面に近い視点から取得された
テクスチャを優先的に選択することで、視覚的品質の高いテクスチャ割り当てを実現する。

\section{テクスチャの射影変換}

透視投影画像から得られるメッシュ対応領域は、透視歪みを含む任意形状の多角形として画像座標系上に表現される。
この領域は、撮影条件や視点方向に依存して形状が変化するため、そのままでは、後段でテクスチャ画像上の画素座標から
対応する三次元位置を一意に復元することが困難となる。

そこで本研究では、透視投影画像上で得られたメッシュ領域を、元のメッシュ形状に対応した正規化された画像座標系へ射影変換する。
これにより、テクスチャ画像上の画素座標とモデルが保持する三次元形状との対応関係を維持する。

三角形メッシュの場合は、三頂点の対応関係に基づいてアフィン変換を適用し、元の三角形形状を保ったテクスチャを生成する。
一方、四角形メッシュの場合は、四頂点の対応関係に基づく射影変換を適用し、透視歪みを補正した長方形テクスチャを生成する。

(実際に多角形変形前後の画像を載せたい)

さらに、変換後の画像に対してマスク処理を行い、メッシュ領域のみを抽出することで、元のメッシュ形状と整合したテクスチャ画像を得る。


\section{テクスチャの視覚的品質の改善}

複数の視点から取得されたテクスチャを同一メッシュに割り当てる場合、視点差や撮影条件の違いにより、
テクスチャ境界に不連続が生じ、四角的品質が悪化してしまう。
本研究では、この不連続を低減するため、隣接する2つのテクスチャに対してブレンド処理を行う。

各テクスチャには視線方向に基づく左右情報が付与されており、これを用いて左右関係を判定する。
左側テクスチャを $I_{\mathrm{L}}$、右側テクスチャを $I_{\mathrm{R}}$とする。

ブレンドは画像全体ではなく、画像幅 $W$ の中央付近に設定した、$x$ 方向の狭い領域に限定して適用する。
画素位置 $(x, y)$ におけるブレンド結果 $I(x, y)$ は、次式で与えられる。
\begin{equation}
I(x, y)
=
w_{\mathrm{L}}(x)\, I_{\mathrm{L}}(x, y)
+
w_{\mathrm{R}}(x)\, I_{\mathrm{R}}(x, y),
\end{equation}
ただし、
\begin{equation}
w_{\mathrm{L}}(x) + w_{\mathrm{R}}(x) = 1
\end{equation}
を満たす。

ブレンド領域内では,$x$ 方向に沿って連続的に変化する重みを用いるため、
シグモイド関数に基づき右側テクスチャの重み $w_{\mathrm{R}}(x)$ を次式で定義する。
\begin{equation}
w_{\mathrm{R}}(x)
=
\frac{1}{1 + \exp\!\left(-k\,\tilde{x}\right)},
\end{equation}
ここで,$\tilde{x} \in [-1,1]$ はブレンド領域内で正規化された水平方向位置を表し、
$k$ は重みの遷移の急峻さを制御するパラメータである。
左側テクスチャの重みは
\begin{equation}
w_{\mathrm{L}}(x) = 1 - w_{\mathrm{R}}(x)
\end{equation}
として与える。

ブレンド領域外では、一方のテクスチャのみを用いることで、テクスチャ全体の解像感および幾何的整合性を維持する。
このように、境界付近に限定した方向性を持つブレンドを行うことで、テクスチャ境界に生じる輝度不連続を効果的に低減できる。

