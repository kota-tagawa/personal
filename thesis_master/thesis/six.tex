\chapter{まとめ}

\section{本研究の成果} 
本研究では、特徴の乏しい屋内環境において、特別な設備や幾何学的に厳密な形状復元を必要とせず、
既存の2次元マップと全方位画像のみを用いた簡易3次元モデルによる自己位置推定手法を確立した。 
本研究により得られた主な成果は以下の通りである。

\subsubsection*{簡易3次元モデルによる自己位置推定の実現}
モデル生成においては、ユーザによる2次元マップ上の頂点指定のみで、撮影位置に基づくメッシュ分割や、
各メッシュの重心にカメラの光軸を整合させたテクスチャ生成を半自動で行う手法を構築した。
また、メッシュ法線と光軸の角度が大きいメッシュにはブレンド処理を適用し、視覚的品質の向上を図った。
生成したモデルと入力画像との照合にはSuperPointおよびSuperGlueを採用した。
SIFTやAKAZEといった従来手法と比較して、特徴の乏しい領域や大きな視点変化に対して著しく高い頑健性を示し、
マッチングに成功したフレームの約95\%において有効な自己位置推定が可能であることを確認した。

\subsubsection*{実用的な屋内ナビゲーションの実装}
本実験では、画像マッチング単独、VIO単独、および提案手法のナビゲーション性能を検証した。
画像マッチング単独では、通信遅延やマッチング失敗によりリアルタイム性に課題が残り、
VIO単独では、特徴の乏しい環境下で発生する累積誤差の蓄積により正確な位置追従が困難であった。
これに対し、両者を統合した提案システムでは、VIOによる推定が画像マッチングの遅延や欠落区間を補間し、
同時に定期的なマッチング結果がVIOの累積誤差を補正することで、双方の欠点を相互に補完した。
以上より、提案手法は特徴の乏しい屋内環境においても、
ナビゲーションに不可欠なリアルタイム性と推定の安定性を両立する実用的な手法であることが実証された。

\section{今後の展望} 
本研究の今後の展望として、線特徴の活用による頑健性の向上が挙げられる。 
現状の手法は点特徴のみに依存しているため、点特徴の乏しい領域や、
壁面を斜めから観測する際の透視投影歪みに起因する誤差により、推定精度が不安定になる課題がある。 
これに対し、ワイヤーフレームに基づく本モデルは線特徴との親和性が高く、柱やドア枠などの幾何構造は環境内に広く分布している。 
これらの線特徴は、点特徴が不安定になる環境下で強力な幾何学的制約を提供できる可能性があり、
これらを統合することで自己位置推定精度のさらなる向上が期待される。