\chapter{まとめ}

\section{本研究の成果} 
本研究では、テクスチャ情報の乏しい屋内環境において、
既存の二次元マップと全方位画像のみを用いた簡易三次元モデルによる自己位置推定手法を確立した。 
本研究により得られた主な成果は以下の通りである。

\begin{enumerate} 
  \item \textbf{簡易三次元モデルによる実用的な自己位置推定の実現}: 
  特別な設備や厳密な形状復元を必要とせず、カメラ視点に基づく最適なサンプリングとブレンド処理を導入することで、
  視覚的品質と推定精度を両立するモデル生成手法を構築した。 
  さらに、生成したモデルに対して深層学習ベースの特徴点抽出を適用することで、
  従来手法では対応が困難であった特徴の乏しい壁面や視点変化の激しい状況下でも、
  実用的な精度で自己位置推定が可能であることを示した。

\item \textbf{屋内ナビゲーションの実証}: 
  提案手法を実装した屋内ナビゲーションシステムを開発し、大学構内での実証実験を通じてその有効性を検証した。 
  実験の結果、モバイル端末の VIO は特徴の乏しい領域でドリフトが生じやすく、
  提案手法による補正は断続的であるというそれぞれの課題が確認された。 
  これらを統合することで、VIOが移動の連続性を維持しつつ、提案手法が累積誤差を定期的に補正するという相互補完効果が得られ、
  特徴点が検出されにくい区間においてもスムーズなナビゲーションを実現できることを実証した。 
\end{enumerate}

\section{今後の展望}
本研究の課題および今後の発展として、主に以下の2点が挙げられる。

\begin{enumerate} 
  \item \textbf{線特徴の活用による頑健性の向上}: 
  現状の手法ではテクスチャの点特徴のみに依存しているため、白壁など特徴の乏しい領域では精度が低下しやすい。
  これに対し、本研究の簡易モデルはワイヤーフレームを骨格としているため、画像から検出した線特徴との親和性が高い。
  柱やドア枠などの幾何構造を直接活用することで、点特徴の乏しい領域における推定精度の向上を図る。

  \item \textbf{実利用者によるナビゲーション評価}: 
  システムの実用性を検証するためには、多様な環境下で、多様な属性を持つ被験者を対象とした実証実験が必要となる。
  目的地到達の正確性といった定量的な評価に加え、ナビゲーションの円滑さや、AR 表示の分かりやすさなど、
  ユーザビリティの観点からシステムを多角的に評価・改善する必要がある。
\end{enumerate}