\chapter{特徴点マッチングに基づく自己位置推定}

\section{テクスチャ画像座標から世界座標への変換}
第4章で取得した簡易モデルのテクスチャと入力画像の対応点は、いずれも画像平面上の2次元座標として表現されている。  
自己位置推定を行うためには、これらの2次元特徴点を世界座標系における3次元座標へ変換する必要がある。
本研究では、テクスチャに割り当てられた画像座標系と、対応する四角形メッシュの四隅の世界座標を用いて、画像座標系から世界座標系への変換を行う。  
以下に、2次元座標を3次元座標へ変換する手順を示す。

\subsection{2次元座標から3次元座標への変換}。 
それぞれの世界座標系における3次元座標を左上から順に時計回りに $P_0,P_1,P_2,P_3$ とすると,平面基底ベクトルは次式で表される。
\begin{equation}
  \bm{B}_1 = P_1 - P_0,\quad
  \bm{B}_2 = P_3 - P_0
\end{equation}

同様に、対応するテクスチャのUV座標を $p_0,p_1,p2,p_3$ とし、UV空間における幅および高さを次式で定義する。
\begin{equation}
  w = p_1^{(u)} - p_0^{(u)},\quad
  h = p_3^{(v)} - p_0^{(v)}
\end{equation}

入力画像上で検出された2次元特徴点をピクセル座標 $(x,y)$ とし,画像サイズを $(W,H)$ とすると、特徴点は次式によりUV空間上の正規化座標 $(u,v)\in[0,1]\times[0,1]$ に変換される。
\begin{equation}
  u = \frac{x/W - p_0^{(u)}}{w},\quad
  v = \frac{y/H - p_0^{(v)}}{h}
\end{equation}

得られた正規化係数 $u,v$ を、世界座標系における平面基底ベクトルの線型結合として用いることで、対応する3次元座標 $\bm{P}$ を次式により計算する。
\begin{equation}
  \bm{P} = P_0 + u\bm{B}_1 + v\bm{B}_2
\end{equation}
これにより、テクスチャ画像上の2次元特徴点を、対応する四角形メッシュ平面上の世界座標系3次元特徴点へ変換することができる。

\section{自己位置推定}
前節で取得した入力画像上の2次元特徴点と、対応するテクスチャ上の3次元特徴点の対応関係を用いて自己位置推定を行う。

本研究では,松下らによって提案された,直交射影誤差に基づく PnPL 問題に対する大域最適解の計算手法を用いる \cite{matsushita2024}。  
一般的な反復的最適化手法では,初期値に依存して局所最適解に陥る可能性があり,また複数の解が存在する場合にそれらを網羅的に求めることが困難である。  
さらに,使用する対応点の配置によっては反復回数が増加し,実行時間が長くなるため,拡張現実などのリアルタイム処理への適用が難しい。
これに対し,本手法は非反復的に大域最適解を求めることが可能であり,計算時間の削減と複数解候補の同時導出を実現する。

本研究では,点特徴の対応のみを入力とする自己位置推定を行う。  
入力として、入力画像上の2次元特徴点に焦点距離を加えたベクトル $(x,y,f)$ と、対応する世界座標系の3次元特徴点 $(X,Y.Z)$ を用いる。   
出力として,虚数解を除いたすべての解候補について,目的関数値 $J$ と対応する回転行列 $\bm{R}$ および並進ベクトル $\bm{t}$ が得られる。
得られた複数の解候補の中から,以下の手順により最終的な自己位置推定結果を選択する。なお、最終的な最終的な自己位置推定結果を選ぶ際に
姿勢誤差が一番小さいものを選んでいるが、理由として姿勢誤差が大きい推定結果は、画像のマッチングが極端に偏りがあるなど、に不備がある可能性が高かったからである。
\begin{enumerate}
  \item 大域最適解法により得られた解候補のうち,天地が反転している解を除外する。  
        本研究では,カメラ座標系の $Y$ 軸が世界座標系の $Z$ 軸の負方向を向くという幾何学的制約に基づき,天地反転の有無を判定する。
  \item 残った各解候補について,回転行列 $\bm{R}$ および並進ベクトル $\bm{t}$ から,カメラ中心位置 $\bm{C}$ を
        \begin{equation}
          \bm{C} = -\bm{R}^{\top} t
        \end{equation}
        により算出する。
  \item 現在位置 $\bm{C}_{\mathrm{gt}}$ が既知である場合,推定されたカメラ中心位置 $\bm{C}$ とのユークリッド距離を,位置誤差 $e_{\mathrm{pos}}$ として次式で定義する。
        \begin{equation}
          e_{\mathrm{pos}} = \lVert \bm{C} - \bm{C}_{\mathrm{gt}} \rVert_2
        \end{equation}
  \item 現在の視線方向 $\bm{f}_{\mathrm{gt}}$ が既知である場合,世界座標系における視線ベクトル $\bm{f}_{\mathrm{world}}$ を次式で定義する。
        \begin{equation}
          \bm{f}_{\mathrm{world}} = \bm{R}^{\top} (0,0,1)^{\top}
        \end{equation}
  \item 推定された視線方向 $\bm{f}_{\mathrm{world}}$ と基準となるカメラ視線方向 $\bm{f}_{\mathrm{gt}}$ とのなす角を,姿勢誤差 $e_{\mathrm{rot}}$ として次式で定義する。
        \begin{equation}
          e_{\mathrm{rot}} =
          \cos^{-1}
          \left(
            \frac{
              \bm{f}_{\mathrm{world}} \cdot \bm{f}_{\mathrm{gt}}
            }{
              \lVert \bm{f}_{\mathrm{world}} \rVert
              \lVert \bm{f}_{\mathrm{gt}} \rVert
            }
          \right)
        \end{equation}
  \item 位置誤差 $e_{\mathrm{pos}}$ および姿勢誤差 $e_{\mathrm{rot}}$ が,それぞれ所定の閾値以下となる解のみを有効な解候補として残す。
  \item 有効な解候補の中から,姿勢誤差 $e_{\mathrm{rot}}$ が最小となる解を,最終的な自己位置推定結果として採用する。
\end{enumerate}
