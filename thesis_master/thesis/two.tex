\chapter{簡易3次元モデルの生成}

\section{簡易3次元モデル生成の方針}

本研究では、屋内ナビゲーションに必要な自己位置推定を効率的に行うため、
現実空間を厳密に模倣するのではなく、計算コストとデータ量を最小限に抑えた簡易的な3次元モデルを生成する方針をとる。

一般に、屋内環境の3次元デジタル化には、レーザースキャナによる高密度な点群計測(図\ref{two:one} 左)や、
多数の画像を用いたフォトグラメトリによるメッシュ生成(図\ref{two:one} 中央)が用いられることが多い。
これらの手法は、環境の形状を詳細に再現できる反面、データ容量が肥大化しやすく、
モバイル端末上でのリアルタイムな描画や照合処理には不向きである。
また、モデル構築のために専門的な機材や多大な計算時間を要する点も課題となる。

これに対し、本研究で提案する簡易モデル(図\ref{two:one} 右)は、
2次元マップから抽出した少数の頂点と線分を幾何的な骨格とし、
そこに全方位画像から取得したテクスチャ情報を付与することで構成される。

このように、必要最低限の幾何構造をワイヤーフレームで、視覚情報をテクスチャとして表現するアプローチをとることで、
高密度な点群や精密なメッシュの生成に依存することなく、ナビゲーション用途として実用上十分な情報の確保と、システム運用における軽量性の両立を目指す。

\begin{figure}[H]
  \centering
  \begin{tabular}{ccc}
      \includegraphics[width=0.3\linewidth]{figures/2/pointscloud.jpg} &
      \includegraphics[width=0.3\linewidth]{figures/2/3DCG.jpg} &
      \includegraphics[width=0.3\linewidth]{figures/2/model.png} \\
      (a) 点群モデル & (b) フォトグラメトリモデル & (c) 本研究の簡易モデル
  \end{tabular}
  \caption{異なる3次元モデル表現の比較. 従来の点群(a)やメッシュ(b)と比較し、本研究(c)では構造を大幅に簡略化している。}
  \label{two:one}
\end{figure}


\section{座標系の定義}

\subsection{カメラ座標系}
簡易モデル生成に用いる各座標軸の関係を図\ref{two:two}に示す。

\begin{figure}[H]
  \centering
  \includegraphics[width=0.5\linewidth]{figures/2/axis.png}
  \caption{世界座標系とカメラ座標系、画像座標系の関係}
  \label{two:two}
\end{figure}

カメラ座標系はカメラの焦点位置を原点とし、
光軸方向を $Z$ 軸、水平右方向を $X$ 軸、鉛直下方向を $Y$ 軸と定める。
一方、世界座標系はモデル床面を $X$-$Y$ 平面、鉛直上向きの法線方向を $Z$ 軸として定義する。
世界座標系上の3次元点 $\bm{p}_w$ は、
カメラの回転行列 $\bm{R}$ と並進ベクトル $\bm{t}$ を用いて、次式によりカメラ座標系上の点 $\bm{p}_c$ に変換される。

\begin{equation}
\bm{p}_c = \bm{R}\bm{p}_w + \bm{t}
\end{equation}

\subsection{画像座標系}
画像座標系は画像左上を原点とし、水平方向を $U$ 軸、垂直方向を $V$ 軸と定める。
カメラ座標系上の3次元点 $\bm{p}_c = (x_c, y_c, z_c)^\top$ は、
カメラ内部パラメータ行列 $\bm{K}$ を用いて、次式により画像座標系上の同次座標 $\bm{p}_s$ へ射影される。

\begin{equation}
\bm{p}_s = \bm{K}\bm{p}_c
\end{equation}

ここで $\bm{p}_s = (u_s, v_s, w_s)^\top$ とすると、実際の透視投影画像上の画素座標 $(u, v)$ は、
$w_s$ による正規化を行うことで、次式で与えられる。

\begin{equation}
u = \frac{u_s}{w_s}, \quad v = \frac{v_s}{w_s}
\end{equation}

\subsection{カメラ内部パラメータの設定}
本研究では理想的な透視投影モデルを仮定し、カメラ内部パラメータを幾何学的に設定する。
内部パラメータ行列 $\bm{K}$ は次式で表される。

\begin{equation}
\bm{K} =
\begin{pmatrix}
f_x & 0 & c_x \\
0 & f_y & c_y \\
0 & 0 & 1
\end{pmatrix}
\end{equation}

ここで $f_x, f_y$ はピクセル単位の焦点距離を表す。
出力する透視投影画像の幅を $W_p$、高さを $H_p$ とし、
水平視野角を $\Theta$、垂直視野角を $\Phi$ と設定した場合、
各焦点距離は次式で表される。

\begin{equation}
f_x = \frac{W_p}{2 \tan(\Theta/2)}, \quad
f_y = \frac{H_p}{2 \tan(\Phi/2)}
\end{equation}

また $c_x, c_y$ は画像中心を表し、透視投影画像の中心座標 $(W_p/2, H_p/2)$ に設定する。


\section{ワイヤーフレーム生成の方針}

本節では、2次元マップ画像を基盤として、屋内環境の幾何構造を記述する3次元ワイヤーフレームモデルを構築する手法について述べる。
本手法における生成プロセスは、以下の3つの主要なフェーズから構成される。

\begin{enumerate}
  \item \textbf{2次元マップと3次元空間の対応付け}: \\
  画像上の画素座標と世界座標の幾何学的対応関係を定義し、モデル生成の基盤となる床領域を確定する。
  
  \item \textbf{カメラ位置に基づく床境界点の最適化}: \\
  定義された床領域に対し、カメラ位置を考慮して床境界点を追加し、形状再現に必要な密度を確保する。
  
  \item \textbf{3次元幾何構造の構築}: \\
  最適化された境界点列に基づき、床面および壁面のメッシュ化処理を行い、最終的な3次元モデルを生成する。
\end{enumerate}

なお、本稿において「床境界点」とは、屋内空間における床面と壁面の境界線を構成する一連の頂点列を指す。
本研究では、ユーザがマップ上で大まかな対応点や領域の指示を与え、
それ以降の詳細な頂点配置や構造化をシステムが自動で行う「半自動」なアプローチを採用する。
これにより、複雑な屋内形状への柔軟な対応と、モデリング作業の効率化の両立を図る。
以下、本処理に必要な前提条件について示した後、各フェーズの詳細について説明する。

\subsection{ワイヤーフレーム生成の前提条件}

ワイヤーフレーム生成にあたっては、以下の情報が利用可能であることを前提とする。

\begin{itemize}
  \item \textbf{屋内環境を表す2次元マップ}: \\
  建物のフロアマップや設計図などの画像データ。
  \item \textbf{2次元マップと3次元空間の対応点}: \\
  マップ上の画素座標と、実空間の3次元座標との対応関係を示す3点以上の点対。これは、マップと実空間の位置合わせのために用いられる。
  \item \textbf{屋内環境内でテクスチャを取得したカメラ位置}: \\
  全方位画像の撮影位置。これは後述する床境界点のサンプリングにおいて、形状の詳細度を決定するために用いられる。
\end{itemize}

次節より、これらの入力情報に基づいた具体的な生成アルゴリズムについて述べる。


\section{2次元マップと3次元空間の対応付け}

本研究では、2次元マップ上で定義された床境界の頂点座標を、実際の3次元空間内の床面上へ写像することで位置合わせを行う。
床面は水平であると仮定し、実空間上の3次元座標は常に $Z=0$ に固定されるものとする。

この写像を行う手法として、本研究では「アフィン変換による自動推定」と、
マップの歪みを考慮した「手動による直接対応付け」の2種類のアプローチを状況に応じて使い分ける。

\subsection{アフィン変換による写像}

建築図面や正確なフロアマップが得られる場合、2次元マップと実環境との関係は、
平行移動・回転・スケーリング・せん断を含むアフィン変換によって十分に表現可能である。

まず、2次元マップ上の点 $\bm{p}_i = (x_i, y_i)^\top$ と、
それに対応する実空間(床面)上の点 $\bm{q}_i = (X_i, Y_i)^\top$ の対応ペアを $N$ 点($N \geq 3$)取得する。
両者の関係は、アフィン変換行列 $\bm{A}$ を用いて以下のように表される。

\begin{equation}
\begin{pmatrix}
X_i \\
Y_i
\end{pmatrix}
=
\bm{A}
\begin{pmatrix}
x_i \\
y_i \\
1
\end{pmatrix}, \quad
\bm{A}
=
\begin{pmatrix}
a_{11} & a_{12} & a_{13} \\
a_{21} & a_{22} & a_{23}
\end{pmatrix}
\end{equation}

ここで、すべての対応点に対して変換誤差が最小となるような行列 $\bm{A}$ を推定する。
具体的には、以下の再投影誤差の二乗和 $E$ を最小化する最小二乗法により $\bm{A}$ を決定する。

\begin{equation}
\hat{\bm{A}} = \underset{\bm{A}}{\operatorname{argmin}} \sum_{i=1}^{N} \left\| \bm{q}_i - \bm{A} \tilde{\bm{p}}_i \right\|^2
\end{equation}

ただし、$\tilde{\bm{p}}_i = (x_i, y_i, 1)^\top$ は同次座標表現である。
このようにして求めた変換行列 $\hat{\bm{A}}$ を用いることで、
2次元マップ上で定義されたすべての頂点を、一括して3次元座標系へ変換する。

\subsection{非線形な歪みへの対応}

一方、簡易的な2次元マップを利用する場合、マップ自体が非線形な歪みを含んでいることがあり、
単一のアフィン変換行列では十分な精度で写像できない場合がある。
このような場合には、上述の自動変換を用いず、床境界の頂点に対して実測した3次元座標を手動で直接割り当てる手法を採用する。
これにより、局所的な歪みが全体の位置合わせに悪影響を与えることを防ぎ、整合性の取れたモデル生成を可能とする。

\subsection{頂点列の回転方向の判定と統一}\label{clockwise}

床境界の多角形の幾何学的整合性を保つため、頂点順序を時計回りに統一する処理を行う。

一般に、平面上の $N$ 角形の頂点列 $\bm{p}_i = (x_i, y_i)^\top$ ($i=1, \dots, N$) が与えられたとき、
その経路が囲む符号付き面積 $S$ は次式で定義される。

\begin{equation}
  S = \frac{1}{2} \sum_{i=1}^{N} (x_i y_{i+1} - x_{i+1} y_i)
\end{equation}

ここで、$\bm{p}_{N+1} = \bm{p}_1$ とする。
本研究の設定するマップ画像の座標系は$Y$軸下向きであるため、頂点列が時計回りの場合、この符号付き面積 $S$ は正の値をとる。
すべての頂点を時計回りに統一するため、算出された $S$ が負の値となった場合、
頂点列の順序を逆順 $(\bm{p}_N, \bm{p}_{N-1}, \dots, \bm{p}_1)$ に並べ替える補正を行う。


\section{カメラ位置に基づく床境界点の最適化}

高品質なテクスチャを生成するためには、テクスチャの歪みを抑えるための詳細な頂点が必要となる。
しかし、これらすべてを手作業で指定することは多大な労力を要するため、現実的ではない。 
そこで本研究では、床境界上においてカメラ配置を考慮しながら適切な間隔で境界点を自動生成する手法を提案する。
図\ref{two:three}に、カメラ位置に基づいて床境界点を追加する概念図を示す。

\begin{figure}[H]
  \centering
  \includegraphics[width=0.8\linewidth]{figures/2/sampling.png}
  \caption{床境界点の追加方法。カメラからの射影点周辺に重点的に頂点を配置する。}
  \label{two:three}
\end{figure}

床境界は閉じた多角形として定義され、隣接する2点 $\bm{p}_0$ と $\bm{p}_1$ が1つの境界エッジを構成する。
本手法では、まず各エッジの始点 $\bm{p}_0$ を固定の境界点として採用した上で、カメラ位置に応じた点を動的に追加していく。

具体的には、各カメラ位置 $\bm{c}$ から境界エッジへの正射影を考える。
エッジの方向ベクトル $\bm{d}$ および、始点からの距離比を表す射影係数 $t$ は次式で与えられる。

\begin{align}
\bm{d} &= \bm{p}_1 - \bm{p}_0 \\
t &= \frac{(\bm{c} - \bm{p}_0)^\top \bm{d}}{\bm{d}^\top \bm{d}}
\end{align}

これより、境界直線上への射影点 $\bm{p}_{\mathrm{proj}}$ は次式で表される。

\begin{equation}
\bm{p}_{\mathrm{proj}} = \bm{p}_0 + t \bm{d}
\end{equation}

ここで、射影係数 $t$ が $0 \leq t \leq 1$ (エッジの範囲内)を満たし、かつカメラとの距離が閾値以下である場合、すなわち
\begin{equation}
\lVert \bm{c} - \bm{p}_{\mathrm{proj}} \rVert \leq d_{\mathrm{th}}
\end{equation}
を満たす場合に、その射影点を有効な候補とする。

次に、得られた有効な射影点に基づき、最終的なサンプリング位置(追加の境界点)を決定する。 
テクスチャマッピング時の歪みを最小限に抑えるには、カメラ正面に近い領域、すなわち射影点付近の幾何形状を重点的に保持することが望ましい。
そこで、同一エッジ上に複数の有効な射影点 $\bm{p}_{\mathrm{proj1}}, \bm{p}_{\mathrm{proj2}}$ が存在する場合には、
その点間距離とユーザが定義するサンプリング間隔 $d_{\mathrm{sample}}$ との大小関係に応じて処理を分岐する。

具体的には、射影点間の距離が十分に大きい場合($\|\bm{p}_{\mathrm{proj1}} - \bm{p}_{\mathrm{proj2}}\| > d_{\mathrm{sample}}$)には、
2つの射影点に挟まれた区間内において、それぞれの射影点から相手側へ一定距離だけ進んだ位置に境界点を追加する。
一方、射影点間の距離が小さい場合($\|\bm{p}_{\mathrm{proj1}} - \bm{p}_{\mathrm{proj2}}\| \leq d_{\mathrm{sample}}$)には、
過度な細分化を防ぐため、両者の中点を1つの境界点として追加する。
これにより、カメラ視点に対して最適な密度で床境界点を配置することが可能となる。

最後に、生成された床境界点列全体について隣接点間の距離を確認し、分布の均一化を行う。
間隔が大きすぎる部分には中点を追加し、間隔が小さすぎる部分では点を統合する。
この後処理により、床境界全体が極端に偏ることなく、おおむね一定のサンプリング間隔で分布するように調整し、メッシュ生成の安定性を確保する。


\section{床面および壁面の幾何構造}

床面は、前節までで得られた床境界頂点列 $\{\bm{P}_i\}$ によって囲まれた領域として定義される。
ここで、床境界は辺同士が互いに交差しない単純多角形を形成しているものとする。
また、各床境界頂点は床面上に存在し、その $z$ 座標は $z=0$ に固定されている。

床面に対応する天井境界点列 $\{\bm{P}_i^{\mathrm{ceil}}\}$ は、床境界頂点列の平面形状を保ったまま、
高さ方向に一定量 $H$ だけ平行移動することで生成する。
すなわち、床境界頂点 $\bm{P}_i = (x_i, y_i, 0)$ に対し、
対応する天井境界頂点は $\bm{P}_i^{\mathrm{ceil}} = (x_i, y_i, H)$ として与えられる。

\subsection{床面のメッシュ化}

実際の屋内環境における床形状は、L字路や障害物による凹凸などが存在するため、必ずしも凸多角形とはならず、凹部を含む一般多角形となる。
このような形状に対して、均質で安定したメッシュを生成するため、
本研究では「制約付き Delaunay 三角形分割(Constrained Delaunay Triangulation)」を用いて床面を分割する。
この手法は、Shewchuk によって提案された高品質な2次元メッシュ生成手法として知られている\cite{Shewchuk1996Triangle}。

具体的には、床境界の頂点集合を入力とし、隣接する床境界頂点同士を結ぶラインを拘束辺(Segment)として設定する。
これにより、多角形の外部や、本来存在しない穴の部分にメッシュが生成されるのを防ぎ、床形状のトポロジーを正確に反映した分割が可能となる。
また、三角形の最大面積を制約条件として与えることで、過度に細長い三角形(Sliver)の生成を抑制し、テクスチャマッピングに適した形状品質を確保する。

最後に、得られた三角形メッシュの各面に対して、幾何学的整合性の確認を行う。\ref{clockwise}節で述べた回転方向の判定方法に基づき、
すべての三角形頂点順序が時計回りとなるよう統一する。
これにより、全メッシュの法線ベクトルが垂直上向き($+z$ 方向)に揃い、面の向きを統一する。

\subsection{壁面のメッシュ化}

壁面の幾何構造は、床境界と天井境界の対応関係に基づいて生成される。
床境界および天井境界は同一の頂点数と順序を持つ閉ループであるため、
対応する上下の隣接頂点対を用いることで壁面を構成できる。

具体的には、床境界の隣接する頂点 $\bm{P}_i, \bm{P}_{i+1}$ と、
それらに対応する天井境界頂点 $\bm{P}_i^{\mathrm{ceil}}, \bm{P}_{i+1}^{\mathrm{ceil}}$ を結ぶことで、
一枚の壁面を表す四角形を定義する。
このようにして生成された床面および壁面の幾何構造により、屋内環境の簡易3次元ワイヤーフレームモデルが構築される。

\subsection{UV座標の定義}\label{UV}

本項では、テクスチャマッピングの基盤となるUV座標の定義について述べる。
UV座標とは、3次元モデルの表面にテクスチャ画像を対応付けるために用いられる2次元座標である。
これは、個々のテクスチャ画像ごとに独立して割り当てられた局所座標系であり、
画像の横軸を $u$、縦軸を $v$ とし、全領域が $[0, 1]$ の範囲に正規化されて表現される。

テクスチャマッピングにおいて、3次元メッシュの幾何学的形状を保存したままテクスチャを割り当てるためには、
メッシュごとにこのUV空間上で適切な座標を定義する必要がある。
そこで本研究では、各メッシュが構成する平面上に局所座標系を設定し、
頂点を投影することで正規化されたUV座標を算出する手法をとる。

あるメッシュを構成する $N$ 個の頂点の世界座標を $\bm{v}^{(i)} \in \mathbb{R}^3$ ($i=0, \dots, N-1$) とする。
まず、この面上に2つの直交する基底ベクトル($U$軸ベクトル $\bm{e}_u$ および $V$軸ベクトル $\bm{e}_v$)を定義する。
$\bm{e}_u$ は、最初の辺の方向ベクトルとして次式で定義する。

\begin{equation}
  \bm{e}_u = \frac{\bm{v}^{(1)} - \bm{v}^{(0)}}{\|\bm{v}^{(1)} - \bm{v}^{(0)}\|}
\end{equation}

次に、面の法線ベクトル $\bm{n}$ を、隣接する2辺の外積により求める。

\begin{equation}
  \bm{n} = \frac{(\bm{v}^{(1)} - \bm{v}^{(0)}) \times (\bm{v}^{(2)} - \bm{v}^{(0)})}
  {\|(\bm{v}^{(1)} - \bm{v}^{(0)}) \times (\bm{v}^{(2)} - \bm{v}^{(0)})\|}
\end{equation}

これらを用いて、$U$軸および法線 $\bm{n}$ の双方に直交するベクトルとして、$V$軸ベクトル $\bm{e}_v$ を算出する。

\begin{equation}
  \bm{e}_v = \bm{n} \times \bm{e}_u
\end{equation}

得られた基底ベクトル $\bm{e}_u, \bm{e}_v$ を用いて、各頂点 $\bm{v}^{(i)}$ をこの平面上に投影し、
一時的な2次元座標 $(u'_i, v'_i)$ を計算する。
これは、頂点 $\bm{v}^{(0)}$ を原点とした相対ベクトルと基底ベクトルの内積により求められる。

\begin{equation}
  u'_i = (\bm{v}^{(i)} - \bm{v}^{(0)})^\top \bm{e}_u, \quad
  v'_i = (\bm{v}^{(i)} - \bm{v}^{(0)})^\top \bm{e}_v
\end{equation}

最後に、得られた局所座標群がアスペクト比を維持したまま単位正方形領域 $[0, 1]^2$ に収まるよう正規化を行う。
全頂点の座標範囲に基づき、以下の変換式により最終的なUV座標 $(u_i, v_i)$ を決定する。

\begin{equation}
  u_i = \frac{u'_i - u'_{\min}}{S}, \quad v_i = \frac{v'_i - v'_{\min}}{S}
\end{equation}

ここで、$u'_{\min}, v'_{\min}$ は各軸における座標の最小値であり、
$S$ は座標群の最大辺長($S = \max_k \{ \max(u'_k - u'_{\min}, v'_k - v'_{\min}) \}$)を表す。
これにより、メッシュの実空間での形状比率を保った最大化されたUV座標が付与される。


\section{全方位画像を用いたテクスチャ取得の方針}

3次元モデルへのテクスチャ割り当てにおいては、モデル表面を十分に覆う視点から撮影された画像を効率的に取得することが重要である。 
一般的な透視投影カメラを用いる場合、多方向のテクスチャ情報を網羅するためには、カメラの向きを変えながら多数の画像を撮影する必要があり、
撮影およびデータ管理に関わるコストが増大するという課題がある。

そこで本研究では、単一の撮影によって全周囲の視覚情報を取得可能な全方位カメラを用いる。
全方位画像は、カメラ位置を中心とする前後・左右・上下の全周囲の風景を一度に記録した画像である。
幾何学的には、カメラを中心とした球面上の情報として定義され、
データ形式としては、一般に正距円筒図法(Equirectangular Projection)などを用いて2次元平面画像に展開し保存されている。

この画像に対し、任意の視線方向を持つ仮想カメラを定義し、その画像平面へ画素を再投影することで、特定の方向に対応する透視投影画像を生成できる。
この特性を利用することで、実際に複数視点から撮影を行うことなく、任意の方向を向いた多数の画像群を計算機上で効率的に生成することが可能となる。


\section{透視投影画像変換}

本研究では、全方位画像を球面上の画素情報として扱う。 
生成する透視投影画像の各画素について、その視線方向に対応する全方位画像の画素値を取得することで、任意方向の画像を生成する。

正距円筒画像の幅および高さをそれぞれ $W_e, H_e$ とする。
全方位カメラの焦点距離 $f$ は全方位画像幅 $W_e$ を用いて次式で表される。

\begin{equation}
f = \frac{W_e}{2 \pi}
\end{equation}

生成する透視投影画像の幅および高さをそれぞれ $W_p, H_p$ とし、
透視投影画像上の画素座標を $(u_p, v_p)$ とすると、
この画素に対応する光軸方向を $z$ 軸とする3次元の視線ベクトル $\bm{x}$ は次式で定義される。
ただし、$N[ \cdot ]$はベクトルのノルムを1にする正規化する正規化作用素である。

\begin{equation}
\bm{x} = N[
\begin{pmatrix}
u_p - W_p/2 \\
v_p - H_p/2 \\
f
\end{pmatrix}
]
\end{equation}

カメラの視線方向を変更するための回転行列 $\bm{R}$ を定義する。
本研究では光軸周りの回転は考慮せず、水平方向の回転$\theta_{eye}$ および 垂直方向の回転$\phi_{eye}$ のみにより姿勢を決定する。
これに対応する回転行列 $\bm{R}(\theta_{eye}, \phi_{eye})$ は次式で与えられる。

\begin{equation}
\bm{R}(\theta_{eye}, \phi_{eye}) = 
\begin{pmatrix}
\cos \theta_{eye} & 0 & \sin \theta_{eye} \\
0 & 1 & 0 \\
-\sin \theta_{eye} & 0 & \cos \theta_{eye}
\end{pmatrix}
\begin{pmatrix}
1 & 0 & 0 \\
0 & \cos \phi_{eye} & -\sin \phi_{eye} \\
0 & \sin \phi_{eye} & \cos \phi_{eye}
\end{pmatrix}
\end{equation}

回転後の視線ベクトル $\bm{x}' = (X', Y', Z')^\top$ は、次式で計算される。

\begin{equation}
\bm{x}' = \bm{R}(\theta_{eye}, \phi_{eye}) \bm{x}
\end{equation}

得られた視線ベクトル $\bm{x}'=(X,Y,Z)$ を用いて、全方位画像の球面座標系 $(\theta_e, \phi_e)$ は次式で計算される。
ただし、$(X',Y',Z')^\top$は$(X,Y,Z)^\top$のノルムを1に正規化したものである。

\begin{equation}
\theta_e = \tan^{-1}\left(\frac{X'}{Z'}\right)
\end{equation}
\begin{equation}
\phi_e = \sin^{-1}\left(\frac{Y'}{\sqrt{X'^2 + Y'^2 + Z'^2}}\right)
\end{equation}

ここで、全方位画像(正距円筒画像)の幅を $W_e$、高さを $H_e$ とすると、
球面座標 $(\theta_e, \phi_e)$ に対応する全方位画像上の画素座標 $(u_e, v_e)$ は次式で与えられる。

\begin{equation}
u_e = \left( \theta_e + \pi \right) \frac{W_e}{2\pi}
\end{equation}
\begin{equation}
v_e = \left( \phi_e + \frac{\pi}{2} \right) \frac{H_e}{\pi}
\end{equation}

以上の手順により、全方位画像の画素 $(u_e, v_e)$ の輝度値を参照することで、
透視投影画像上の画素 $(u_p, v_p)$ の画素値を決定する。


\section{透視投影画像を用いた全方位カメラの位置姿勢推定}
本節では、点特徴および線特徴の双方を用いてカメラ位置姿勢を推定するため、菅谷ら\cite{Sugaya2024}による直交射影モデルに基づく手法を採用する。
この手法は、点の共線性誤差と線の共面性誤差を統一的な枠組みで定式化することで、点と線の両特徴を同時に扱い、カメラ姿勢を推定するものである。
直交射影の共線性と共面性に基づくカメラ姿勢推定に必要な入力変数は以下の通りである。なお、透視投影画像に投影されない(画角外の)点は入力から除外する。
\begin{itemize}
  \item $\bm{p}_i$ : 世界座標系における空間点 $i$ の座標。
  \item $\bm{v}_i$ : $\bm{p}_i$ に対応する画像上の特徴点ベクトル。
  \item $\bm{d}_i$ : 世界座標系における直線 $L_i$ の方向ベクトル。
  \item $\bm{r}_i$ : 世界座標系における直線 $L_i$ 上の点の座標。
  \item $\bm{n}_i$ : 直線 $L_i$ に対応する画像上の直線の法線ベクトル。
\end{itemize}
画像上の特徴点ベクトル $\bm{v}_i$ は、透視投影画像上で $\bm{p}_i$ に対応する画素を選択し、
前節で定義したカメラモデルを用いてカメラ座標系上のベクトルとして表現することで得られる。
まず、画像中心を原点とした画素座標 $(u', v')$ は次式で与えられる。

\begin{equation}
  u' = u_p - \frac{W_p}{2}, \quad v' = v_p - \frac{H_p}{2}
\end{equation}

これに焦点距離 $f$ を合わせ、$\bm{v}_i = (u', v', f)^\top$ とする。
直線の方向ベクトル $\bm{d}_i$ は、直線を構成する2つの3次元点(モデルの頂点等) $\bm{p}_i, \bm{p}_{i+1}$ を用いて、次式で計算される。

\begin{equation}
  \bm{d}_i = \frac{\bm{p}_{i+1} - \bm{p}_i}{\|\bm{p}_{i+1} - \bm{p}_i\|}
\end{equation}

同様に、画像上の直線の法線ベクトル $\bm{n}_i$ は、直線端点に対応する 2 つの特徴点ベクトル $\bm{v}_i, \bm{v}_{i+1}$ の外積により求める。

\begin{equation}
  \bm{n}_i = \frac{\bm{v}_i \times \bm{v}_{i+1}}{\|\bm{v}_i \times \bm{v}_{i+1}\|}
\end{equation}

これらの入力を用いて、複数視線方向の透視投影画像に対する目的関数を反復的に最小化することで、自己位置推定を行う。
最終的に得られるカメラの回転行列 $\bm{R}$ と並進ベクトル $\bm{t}$ は、世界座標系の3次元座標をカメラ座標系の座標に変換する行列である。


\section{メッシュの座標系変換と投影}

テクスチャを割り当てるメッシュの頂点座標を、世界座標系から全方位画像に基づいて生成した透視投影画像の画像座標系へと変換する。
まず、メッシュを構成する頂点の世界座標 $\bm{v}_{\rm world}^{(i)} \in \mathbb{R}^3$($i=1,2,3$)を、
カメラの回転行列 $\bm{R}$ および並進ベクトル $\bm{t}$ を用いてカメラ座標系へ変換する。

\begin{equation}
  \bm{v}_{\rm cam}^{(i)} = \bm{R}\bm{v}_{\rm world}^{(i)} + \bm{t}
\end{equation}

変換後の頂点座標を用いて,メッシュの重心位置ベクトル $\bm{c}_{\rm cam}$ を次式で求める。

\begin{equation}
  \bm{c}_{\rm cam} = \frac{1}{3} \sum_{i=1}^{3} \bm{v}_{\rm cam}^{(i)}
\end{equation}

この目的のため、メッシュ重心方向 $\bm{c}_{\rm cam}$ を視線方向とする新たな回転行列 $\bm{R}_{\rm proj}$ を構成する。
$\bm{R}_{\rm proj}$ の各列ベクトルとなる直交基底 $\bm{x}_{\rm proj}, \bm{y}_{\rm proj}, \bm{z}_{\rm proj}$ は、
Gram-Schmidtの直交化法に基づき、以下の手順で算出する。
まず、重心方向ベクトルを正規化し、新たな$Z$軸 $\bm{z}_{\rm proj}$ とする。

\begin{equation}
  \bm{z}_{\rm proj} = \frac{\bm{c}_{\rm cam}}{\|\bm{c}_{\rm cam}\|}
\end{equation}

次に、決定した$Z$軸 $\bm{z}_{\rm proj}$ を基準とし、世界座標系の上方向ベクトル $\bm{u} = (0, 1, 0)^\top$ に対して直交化を行うことで $Y$軸 $\bm{y}_{\rm proj}$ を求める。
具体的には、以下の式により $\bm{u}$ から $\bm{z}_{\rm proj}$ 方向への射影成分を除去し、正規化する。

\begin{equation}
  \bm{y}' = \left( \bm{I} - \bm{z}_{\rm proj}\bm{z}_{\rm proj}^\top \right) \bm{u}, \quad \bm{y}_{\rm proj} = \frac{\bm{y}'}{\|\bm{y}'\|}
\end{equation}

最後に,$\bm{y}_{\rm proj}$ と $\bm{z}_{\rm proj}$ の外積により$X$軸 $\bm{x}_{\rm proj}$ を求め、
これらを並べて回転行列 $\bm{R}_{\rm proj}$ を定義する。

\begin{equation}
  \bm{x}_{\rm proj} = \bm{y}_{\rm proj} \times \bm{z}_{\rm proj}, \quad
  \bm{R}_{\rm proj} = 
  \begin{bmatrix} 
    \bm{x}_{\rm proj} & \bm{y}_{\rm proj} & \bm{z}_{\rm proj} 
  \end{bmatrix}
\end{equation}

この回転行列を用いて,カメラ座標系上の頂点を回転させる。

\begin{equation}
  \bm{v}_{\rm proj}^{(i)} = \bm{R}_{\rm proj}^{\top} \bm{v}_{\rm cam}^{(i)}
\end{equation}

回転後の3次元点をカメラ内部パラメータ行列 $\bm{K}$ を用いて画像座標系へ投影する。
ここで、初期設定の画像サイズ $(W, H)$ および内部パラメータに対し、メッシュ全体が画像内に収まらない場合が発生し得る。
そのため、投影された画素座標の画像中心からの最大距離に基づき、透視投影画像の画像サイズおよび内部パラメータを動的に更新する。
更新後の画像幅 $W_{tex}$ および高さ $H_{tex}$ は、画像中心 $(W/2, H/2)$ からの最大オフセット量を用いて決定し、それに伴い内部パラメータ行列の光学中心 $(c_x, c_y)$ も画像中央へ移動させる。
この処理は、焦点距離を維持したまま画角のみを拡大することに相当し、解像度を低下させることなくメッシュ全体のテクスチャ取得を可能にする。


\subsection{テクスチャ候補の評価}

メッシュ重心までの距離が遠い、あるいはメッシュを斜めから撮影している場合は、
テクスチャの解像度不足や歪みにより視覚的品質が劣化する原因となる。
そのため、テクスチャ割り当ての優先度を決定するための指標として、
以下の幾何学的特徴量を計算する。

\begin{itemize}
  \item \textbf{メッシュ重心までの距離} $d$
  \begin{equation}
    d = \|\bm{c}_{\rm cam}\|
  \end{equation}

  \item \textbf{視線方向と法線方向のなす角} $\theta$ \\
  まず、メッシュを構成する3つの頂点座標 $\bm{v}_{\rm cam}^{(0)}, \bm{v}_{\rm cam}^{(1)}, \bm{v}_{\rm cam}^{(2)}$ を用いて、
  メッシュの法線ベクトル $\bm{n}_{\rm cam}$ を算出する。
  \begin{equation}
    \bm{n}_{\rm cam} = \frac{(\bm{v}_{\rm cam}^{(1)} - \bm{v}_{\rm cam}^{(0)}) \times (\bm{v}_{\rm cam}^{(2)} - \bm{v}_{\rm cam}^{(0)})}
    {\|(\bm{v}_{\rm cam}^{(1)} - \bm{v}_{\rm cam}^{(0)}) \times (\bm{v}_{\rm cam}^{(2)} - \bm{v}_{\rm cam}^{(0)})\|}
  \end{equation}
  これを用い,メッシュ重心方向ベクトル(視線ベクトル)$\bm{c}_{\rm cam}$ とのなす角 $\theta$ を次式で求める。
  \begin{equation}
    \theta = \arccos \left( \frac{\bm{c}_{\rm cam}^\top \bm{n}_{\rm cam}}{\|\bm{c}_{\rm cam}\| \, \|\bm{n}_{\rm cam}\|} \right)
  \end{equation}
\end{itemize}

これらの評価量に基づき,距離 $d$ が小さく,かつ角度 $\theta$ が小さい(正面に近い)テクスチャを優先的に選択することで,
視覚的品質の高いテクスチャ割り当てを実現する。


\subsection{カメラとメッシュの相対位置関係の導出}

テクスチャのブレンド処理を行う際,隣接するテクスチャとの境界を適切に処理するために,
カメラに対してメッシュが左右どちらの方向にあるかを判定する必要がある。
そこで,カメラから見たメッシュの相対的な左右位置を表す指標 $s \in \{-1, 0, 1\}$ を定義する。

まず,カメラ座標系における物理的な上方向ベクトル $\bm{u}_{\rm up} = (0, -1, 0)^\top$ を定義する。
この $\bm{u}_{\rm up}$ とメッシュの法線ベクトル $\bm{n}_{\rm cam}$ の外積により,
メッシュ平面に沿った右方向ベクトル $\bm{r}$ を算出する。

\begin{equation}
  \bm{r} = \bm{u}_{\rm up} \times \bm{n}_{\rm cam}
\end{equation}

次に,正規化された視線ベクトル $\bm{v}_{\rm view} = \bm{c}_{\rm cam} / \|\bm{c}_{\rm cam}\|$ を用いて,次式により判定を行う。
ここで、$-\bm{v}_{\rm view}$ はメッシュからカメラへ向かうベクトルを表す。

\begin{equation}
  s = \mathrm{sgn}\left( (-\bm{v}_{\rm view})^\top \frac{\bm{r}}{\|\bm{r}\|} \right)
\end{equation}

この $s$ の符号により,カメラがメッシュの正面に向かって左側 ($s=-1$)、
あるいは右側 ($s=1$) のいずれに位置しているかを記述する。


\section{テクスチャ画像の形状変換}

透視投影された入力画像において、各メッシュに対応する領域は、視点方向や撮影条件に起因する透視歪みを含んでおり、画像座標系上では不規則な多角形として観測される。
この歪んだ領域をそのままテクスチャとして使用すると、3次元モデルの幾何形状と整合せず、適切なマッピングが行えない。
そこで本研究では、画像上のメッシュ領域に対して射影変換を適用し、各メッシュに定義されたUV座標系の幾何学的形状へと正規化する。

まず、入力画像上におけるメッシュ頂点の画素座標を $\bm{p}_{\rm src}^{(i)} = (x_i, y_i)^\top$ ($i=1, \dots, 4$) とする。
これらは、透視投影により歪んだ四角形を形成している。
次に、変換先の座標 $\bm{p}_{\rm dst}^{(i)}$ を決定する。
本手法では、メッシュデータに付与されている正規化UV座標 $\bm{u}^{(i)} = (u_i, v_i)^\top$ ($0 \leq u_i, v_i \leq 1$) と、
生成するテクスチャ画像の解像度 $(W_{\rm tex}, H_{\rm tex})$ を用いて、変換後のターゲット座標 $(x'_i, y'_i)$ を次式で定義する。

\begin{equation}
  \bm{p}_{\rm dst}^{(i)} = \begin{pmatrix} x'_i \\ y'_i \end{pmatrix} 
  = \begin{pmatrix} u_i \cdot W_{\rm tex} \\ v_i \cdot H_{\rm tex} \end{pmatrix}
\end{equation}

入力画像上の点 $\bm{p}_{\rm src}^{(i)}$ をテクスチャ画像上の点 $\bm{p}_{\rm dst}^{(i)}$ へ写像する射影行列 $\bm{H}$ は、
同次座標系において以下の関係を満たす $3 \times 3$ 行列として定義される。

\begin{equation}
  \begin{pmatrix} x'_i \\ y'_i \\ 1 \end{pmatrix} \sim \bm{H}\begin{pmatrix} x_i \\ y_i \\ 1 \end{pmatrix}
  , \quad
  \bm{H} = \begin{bmatrix} h_{11} & h_{12} & h_{13} \\ h_{21} & h_{22} & h_{23} \\ h_{31} & h_{32} & h_{33} \end{bmatrix}
\end{equation}

この関係式を展開すると、1組の対応点につき以下の2つの線形方程式が得られる。

\begin{equation}
  \begin{cases}
    h_{11}x_i + h_{12}y_i + h_{13} - h_{31}x_i x'_i - h_{32}y_i x'_i - h_{33}x'_i = 0 \\
    h_{21}x_i + h_{22}y_i + h_{23} - h_{31}x_i y'_i - h_{32}y_i y'_i - h_{33}y'_i = 0
  \end{cases}
\end{equation}

定数倍の不定性を除くと、行列 $\bm{H}$ の自由度は8である。
したがって、四角形メッシュの4頂点の対応関係を用いることで計8本の連立方程式を立てることができ、
DLT法 (Direct Linear Transformation) を適用することで $\bm{H}$ の各成分を一意に決定できる。

実際にテクスチャ画像を生成する際は、算出された行列 $\bm{H}$ を用いて逆写像を行う。
出力画像の各画素座標 $(u, v)$ に対し、逆行列 $\bm{H}^{-1}$ を乗算することで、入力画像上の対応座標 $(\hat{x}, \hat{y})$ を求める。

\begin{equation}
  \begin{pmatrix} \hat{x} \\ \hat{y} \\ 1 \end{pmatrix} \sim \bm{H}^{-1} \begin{pmatrix} u \\ v \\ 1 \end{pmatrix}
\end{equation}

算出された座標 $(\hat{x}, \hat{y})$ は一般に実数値となるため、
入力画像における近傍4画素の輝度値を用いた双線形補間(Bilinear Interpolation)により、当該画素の輝度値を決定する。
この逆写像方式を採用することで、変換後の画像に画素抜けが生じるのを防ぎ、
滑らかな正規化テクスチャ画像 $\bm{I}_{\rm dst}$ を生成することが可能となる。


\section{テクスチャの視覚的品質の改善}

複数の視点から取得されたテクスチャを同一メッシュに統合する際、
視点位置や撮影条件の差異に起因してテクスチャ境界に不連続が生じ、視覚的品質が低下する場合がある。
本研究では、この不連続を緩和するため、隣接するテクスチャ間の境界領域に対してブレンド処理を適用する。

ただし、視線方向が対象の正面に近い画像は射影歪みが十分に小さいため、ブレンド処理の対象外とする。
したがって、本手法におけるブレンド処理は、左右方向から撮影され、互いに重なり合うテクスチャの境界においてのみ適用される。

ブレンド対象となる各テクスチャには、前節で算出した視線方向に基づく左右情報 $s \in \{-1, +1\}$ が付与されている。
本手法では、この符号を用いてブレンド領域における左右のテクスチャ割り当てを決定する。
ここで、ブレンドを行う画像領域の左側を担当するテクスチャを $I_{\mathrm{L}}$、右側を担当するテクスチャを $I_{\mathrm{R}}$ と定義する。

一般に、対象物を斜め方向から撮影した際、カメラ正対面から離れるほど射影歪みが大きくなる。
視線方向左側にテクスチャがある場合($s=-1$)、取得されたテクスチャ画像の右側面は対象の正面側に近いため、比較的歪みが小さい。
したがって、この画像はブレンド領域の右側 $I_{\mathrm{R}}$ を担当する。
逆に、視線方向右側にテクスチャがある場合($s=+1$)、テクスチャ画像の左側面において歪みが小さくなることから、ブレンド領域の左側 $I_{\mathrm{L}}$ を担当する。

ブレンド処理は画像全体ではなく、画像幅 $W$ の中央付近に設定した水平方向の一定範囲に限定して適用する。
画素位置 $(x, y)$ におけるブレンド結果 $I(x, y)$ は、次式で与えられる。

\begin{equation}
  I(x, y) = w_{\mathrm{L}}(x)\, I_{\mathrm{L}}(x, y) + w_{\mathrm{R}}(x)\, I_{\mathrm{R}}(x, y)
  \label{eq:blend_func}
\end{equation}

ただし、重み係数は常に正規化条件 $w_{\mathrm{L}}(x) + w_{\mathrm{R}}(x) = 1$ を満たすものとする。
ブレンド領域内において、テクスチャの切り替わりを滑らかにするため、シグモイド関数に基づき右側領域用の重み $w_{\mathrm{R}}(x)$ を次式で定義する。

\begin{equation}
  w_{\mathrm{R}}(x) = \frac{1}{1 + \exp\!\left(-k\,\tilde{x}\right)}
  \label{eq:sigmoid_weight}
\end{equation}

ここで、$\tilde{x} \in [-1,1]$ はブレンド領域内で正規化された水平方向位置を表し、$k$ は重み変化の急峻さを制御するパラメータである。
左側領域用の重みは $w_{\mathrm{L}}(x) = 1 - w_{\mathrm{R}}(x)$ として一意に定まる。

この重み定義により、画像の左端から右端に向かうにつれて、支配的なテクスチャが $I_{\mathrm{L}}$ から $I_{\mathrm{R}}$ へと滑らかに遷移する。
領域外では一方のテクスチャのみを用いることで、テクスチャ本来の解像度および幾何的整合性を維持しつつ、境界付近の不連続を効果的に低減する。

