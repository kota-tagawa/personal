\chapter{簡易3次元モデルの生成}

\section{簡易3次元モデル生成の方針}

本研究では、屋内ナビゲーションに必要な自己位置推定を効率的に行うため、
現実空間を厳密に模倣するのではなく、計算コストとデータ量を最小限に抑えた「簡易三次元モデル」を生成する方針をとる。

一般に、屋内環境の3次元デジタル化には、レーザースキャナによる高密度な点群計測(図\ref{two:one} 左)や、
多数の画像を用いたフォトグラメトリによるメッシュ生成(図\ref{two:one} 中央)が用いられることが多い。
これらの手法は、環境の形状を詳細に再現できる反面、データ容量が肥大化しやすく、
モバイル端末上でのリアルタイムな描画や照合処理には不向きである。
また、モデル構築のために専門的な機材や多大な計算時間を要する点も課題となる。

これに対し、本研究で提案する簡易モデル(図\ref{two:one} 右)は、
2次元マップから抽出した少数の頂点と線分(ワイヤーフレーム)を幾何的な骨格とし、
そこに全方位画像から取得したテクスチャ情報を付与することで構成される。

このように、必要最低限の幾何構造をワイヤーフレームで、視覚情報をテクスチャとして表現するアプローチをとることで、
高密度な点群や精密なメッシュの生成に依存することなく、ナビゲーション用途として実用上十分な情報の確保と、システム運用における軽量性の両立を目指す。

\begin{figure}[H]
  \centering
  \begin{tabular}{ccc}
      \includegraphics[width=0.3\linewidth]{figures/2/pointscloud.jpg} &
      \includegraphics[width=0.3\linewidth]{figures/2/3DCG.jpg} &
      \includegraphics[width=0.3\linewidth]{figures/2/model.png} \\
      (a) 点群モデル & (b) フォトグラメトリモデル & (c) 本研究の簡易モデル
  \end{tabular}
  \caption{異なる3次元モデル表現の比較. 従来の点群(a)やメッシュ(b)と比較し、本研究(c)では構造を大幅に簡略化している.}
  \label{two:one}
\end{figure}


\section{座標系の定義}

\subsection{カメラ座標系}
簡易モデル生成に用いる各座標軸の関係を図\ref{three:one}に示す。

\begin{figure}[H]
  \centering
  \includegraphics[width=0.5\linewidth]{figures/3/axis.png}
  \caption{世界座標系とカメラ座標系、画像座標系の関係}
  \label{three:one}
\end{figure}

カメラ座標系はカメラの焦点位置を原点とし、
光軸方向を $z$ 軸、水平右方向を $x$ 軸、鉛直下方向を $y$ 軸と定める。
一方、世界座標系はモデル床面を $X$-$Y$ 平面、鉛直上向きの法線方向を $Z$ 軸として定義する。
世界座標系上の3次元点 $\bm{p}_w$ は、
カメラの回転行列 $\bm{R}$ と並進ベクトル $\bm{t}$ を用いて、次式によりカメラ座標系上の点 $\bm{p}_c$ に変換される。

\begin{equation}
\bm{p}_c = \bm{R}\bm{p}_w + \bm{t}
\end{equation}

\subsection{画像座標系}
画像座標系は画像左上を原点とし、水平方向を $u$ 軸、垂直方向を $v$ 軸と定める。
カメラ座標系上の3次元点 $\bm{p}_c = (x_c, y_c, z_c)^\top$ は、
カメラ内部パラメータ行列 $\bm{K}$ を用いて、次式により画像座標系上の同次座標 $\bm{p}_s$ へ射影される。

\begin{equation}
\bm{p}_s = \bm{K}\bm{p}_c
\end{equation}

ここで $\bm{p}_s = (u_s, v_s, w_s)^\top$ とすると、実際の透視投影画像上の画素座標 $(u, v)$ は、
$w_s$ による正規化(透視除算)により次式で得られる。

\begin{equation}
u = \frac{u_s}{w_s}, \quad v = \frac{v_s}{w_s}
\end{equation}

\subsection{カメラ内部パラメータの設定}
本研究では理想的な透視投影モデルを仮定し、カメラ内部パラメータを幾何学的に設定する。
内部パラメータ行列 $\bm{K}$ は次式で表される。

\begin{equation}
\bm{K} =
\begin{pmatrix}
f_x & 0 & c_x \\
0 & f_y & c_y \\
0 & 0 & 1
\end{pmatrix}
\end{equation}

ここで $f_x, f_y$ はピクセル単位の焦点距離を表す。
出力する透視投影画像の幅を $W_p$、高さを $H_p$ とし、
水平視野角を $\Theta$、垂直視野角を $\Phi$ と設定した場合、
各焦点距離は次式で表される。

\begin{equation}
f_x = \frac{W_p}{2 \tan(\Theta/2)}, \quad
f_y = \frac{H_p}{2 \tan(\Phi/2)}
\end{equation}

また $c_x, c_y$ は画像中心を表し、透視投影画像の中心座標 $(W_p/2, H_p/2)$ に設定する。


\section{ワイヤーフレーム生成の方針}

本節では、2次元マップを基盤として、屋内環境の幾何構造を表す3次元ワイヤーフレームモデルを構築する一連の手順について述べる。
本手法におけるワイヤーフレーム生成は、以下の3つの主要なフェーズから構成される。

\begin{enumerate}
  \item \textbf{2次元マップと3次元空間の対応付け}: \\
  画像として与えられる2次元マップ上の位置と、実際の世界座標との対応関係を定義する。
  \item \textbf{床境界の定義と最適化}: \\
  2次元マップ上で指定された床領域の輪郭に対し、カメラ位置を考慮して、適切な間隔で床境界点を追加する。
  \item \textbf{3次元幾何構造の構築}: \\
  確定した床境界に基づき、床面および壁面のメッシュ化を行うことで、最終的な3次元ワイヤーフレームを出力する。
\end{enumerate}

なお、ここでいう床境界点とは、屋内空間における床面と壁面の境界線を構成する一連の頂点を指す。
本研究では、ユーザがマップ上での対応点指示や大まかな床領域指定を行い、それ以降の詳細な頂点生成や構造化を計算機が自動で行う「半自動」なアプローチを採用する。
以下、この処理フローに必要な入力情報と前提条件について述べた後、各フェーズの詳細について説明する。

\subsection{ワイヤーフレーム生成の前提条件}\label{wire_input}

ワイヤーフレーム生成にあたっては、以下の情報が利用可能であることを前提とする。

\begin{itemize}
  \item \textbf{屋内環境を表す2次元マップ}: \\
  建物のフロアマップや設計図などの画像データ。
  \item \textbf{2次元マップと3次元空間の対応点}: \\
  マップ上の画素座標と、実空間の3次元座標との対応関係を示す4点以上の点対。これは、マップと実空間の位置合わせのために用いられる。
  \item \textbf{屋内環境内でテクスチャを取得したカメラ位置}: \\
  全方位画像の撮影位置。これは後述する床境界点のサンプリングにおいて、形状の詳細度を決定するために用いられる。
\end{itemize}

次節より、これらの入力情報に基づいた具体的な生成アルゴリズムについて述べる。


\section{2次元マップと3次元空間の対応付け}

本研究では、2次元マップ上で定義された床境界の頂点座標を、実際の3次元空間内の床面上へ写像することで位置合わせを行う。
床面は水平であると仮定し、実空間上の3次元座標は常に $Z=0$ に固定されるものとする。

この写像を行う手法として、本研究では「アフィン変換による自動推定」と、
マップの歪みを考慮した「手動による直接対応付け」の2種類のアプローチを状況に応じて使い分ける。

\subsection{アフィン変換による写像}
建築図面や正確なフロアマップが得られる場合、2次元マップと実環境との関係は、
平行移動・回転・スケーリング・せん断を含むアフィン変換によって十分に表現可能である。

まず、2次元マップ上の点 $\bm{p}_i = (x_i, y_i)^\top$ と、
それに対応する実空間(床面)上の点 $\bm{q}_i = (X_i, Y_i)^\top$ の対応ペアを $N$ 点($N \geq 3$、推奨は4点以上)取得する。
両者の関係は、アフィン変換行列 $\bm{A}$ を用いて以下のように表される。

\begin{equation}
\begin{pmatrix}
X_i \\
Y_i
\end{pmatrix}
=
\bm{A}
\begin{pmatrix}
x_i \\
y_i \\
1
\end{pmatrix}, \quad
\bm{A}
=
\begin{pmatrix}
a_{11} & a_{12} & a_{13} \\
a_{21} & a_{22} & a_{23}
\end{pmatrix}
\end{equation}

ここで、すべての対応点に対して変換誤差が最小となるような行列 $\bm{A}$ を推定する。
具体的には、以下の再投影誤差の二乗和 $E$ を最小化する最小二乗法により $\bm{A}$ を決定する。

\begin{equation}
\hat{\bm{A}} = \underset{\bm{A}}{\operatorname{argmin}} \sum_{i=1}^{N} \left\| \bm{q}_i - \bm{A} \tilde{\bm{p}}_i \right\|^2
\end{equation}

ただし、$\tilde{\bm{p}}_i = (x_i, y_i, 1)^\top$ は同次座標表現である。
このようにして求めた変換行列 $\hat{\bm{A}}$ を用いることで、
2次元マップ上で定義されたすべての頂点を、一括して3次元座標系へ変換する。

\subsection{非線形な歪みへの対応}
一方、簡易的な2次元マップを利用する場合、マップ自体が非線形な歪みを含んでいることがあり、
単一のアフィン変換行列では十分な精度で写像できない場合がある。
このような場合には、上述の自動変換を用いず、床境界の頂点に対して実測した3次元座標を手動で直接割り当てる手法を採用する。
これにより、局所的な歪みが全体の位置合わせに悪影響を与えることを防ぎ、整合性の取れたモデル生成を可能とする。

\subsection{床境界の整合性確認と補正}
上記いずれかの手法により3次元座標を得た後、床境界の頂点列としての整合性を確認する。
まず、頂点列が閉ループを形成していない場合には、始点と終点を接続することで多角形となるように修正する。
さらに、頂点列の並び順を外積により判定し、
反時計回りとなっている場合には順序を反転させることで、すべての床境界が時計回りとなるように統一する。


\section{床境界におけるカメラ位置に基づくサンプリング}

高品質なテクスチャを生成するためには、テクスチャの歪みを抑えるための詳細な頂点が必要となる。
しかし、これらすべてを手作業で指定することは多大な労力を要するため、現実的ではない。 
そこで本研究では、床境界上においてカメラ配置を考慮しながら適切な間隔で境界点を自動生成する手法を提案する。
図\ref{two:two}に、カメラ位置に基づいて床境界点をサンプリングする概念図を示す。

\begin{figure}[H]
  \centering
  \includegraphics[width=0.8\linewidth]{figures/2/sampling.png}
  \caption{床境界点のサンプリング方法. カメラからの射影点周辺に重点的に頂点を配置する.}
  \label{two:two}
\end{figure}

床境界は閉じた多角形として定義され、隣接する2点 $\bm{p}_0$ と $\bm{p}_1$ が1つの境界エッジを構成する。
本手法では、まず各エッジの始点 $\bm{p}_0$ を固定の境界点として採用した上で、カメラ位置に応じた点を動的に追加していく。

具体的には、各カメラ位置 $\bm{c}$ から境界エッジへの正射影を考える。
エッジの方向ベクトル $\bm{d}$ および、始点からの距離比を表す射影係数 $t$ は次式で与えられる。

\begin{align}
\bm{d} &= \bm{p}_1 - \bm{p}_0 \\
t &= \frac{(\bm{c} - \bm{p}_0)^\top \bm{d}}{\bm{d}^\top \bm{d}}
\end{align}

これより、境界直線上への射影点 $\bm{p}_{\mathrm{proj}}$ は次式で表される。

\begin{equation}
\bm{p}_{\mathrm{proj}} = \bm{p}_0 + t \bm{d}
\end{equation}

ここで、射影係数 $t$ が $0 \leq t \leq 1$ (エッジの範囲内)を満たし、かつカメラとの距離が閾値以下である場合、すなわち
\begin{equation}
\lVert \bm{c} - \bm{p}_{\mathrm{proj}} \rVert \leq d_{\mathrm{th}}
\end{equation}
を満たす場合に、その射影点を有効な候補とする。

次に、得られた有効な射影点に基づき、最終的なサンプリング位置(追加の境界点)を決定する。 
テクスチャマッピング時の歪みを最小限に抑えるには、カメラ正面に近い領域、すなわち射影点付近の幾何形状を重点的に保持することが望ましい。
そこで、同一エッジ上に複数の有効な射影点 $\bm{p}_{\mathrm{proj1}}, \bm{p}_{\mathrm{proj2}}$ が存在する場合には、
その点間距離とユーザが定義するサンプリング間隔 $d_{\mathrm{sample}}$ との大小関係に応じて処理を分岐する。

具体的には、射影点間の距離が十分に大きい場合($\|\bm{p}_{\mathrm{proj1}} - \bm{p}_{\mathrm{proj2}}\| > d_{\mathrm{sample}}$)には、
2つの射影点に挟まれた区間内において、それぞれの射影点から相手側へ一定距離だけ進んだ位置に境界点を追加する。
一方、射影点間の距離が小さい場合($\|\bm{p}_{\mathrm{proj1}} - \bm{p}_{\mathrm{proj2}}\| \leq d_{\mathrm{sample}}$)には、
過度な細分化を防ぐため、両者の中点を1つの境界点として追加する。
これにより、カメラ視点に対して最適な密度で床境界点を配置することが可能となる。

最後に、生成された床境界点列全体について隣接点間の距離を確認し、分布の均一化を行う。
間隔が大きすぎる部分(疎な領域)には中点を追加し、間隔が小さすぎる部分(密な領域)では点を統合する。
この後処理により、床境界全体が極端に偏ることなく、おおむね一定のサンプリング間隔で分布するように調整し、メッシュ生成の安定性を確保する。


\section{床面および壁面の幾何構造}

床面は、前節までで得られた床境界頂点列 $\{\bm{P}_i\}$ によって囲まれた領域として定義される。
ここで、床境界は辺同士が互いに交差しない「単純多角形」を形成しているものとする。
また、各床境界頂点は床面上に存在し、その $z$ 座標は $z=0$ に固定されている。

床面に対応する天井境界点列 $\{\bm{P}_i^{\mathrm{ceil}}\}$ は、床境界頂点列の平面形状を保ったまま、
高さ方向に一定量 $H$ だけ平行移動することで生成する。
すなわち、床境界頂点 $\bm{P}_i = (x_i, y_i, 0)$ に対し、
対応する天井境界頂点は $\bm{P}_i^{\mathrm{ceil}} = (x_i, y_i, H)$ として与えられる。

\subsection{床面のメッシュ化}
実際の屋内環境における床形状は、L字路や柱の出っ張りなどが存在するため、必ずしも凸多角形とはならず、凹部を含む一般多角形となる。
このような形状に対して、均質で安定したメッシュを生成するため、
本研究では「制約付き Delaunay 三角形分割(Constrained Delaunay Triangulation)」を用いて床面を分割する。
この手法は、Shewchuk によって提案された高品質な2次元メッシュ生成手法として知られている\cite{Shewchuk1996Triangle}。

具体的には、床境界頂点を入力頂点集合とし、隣接する床境界頂点同士を結ぶラインを拘束辺として設定する。
これにより、本来存在しない壁の外側や空間の穴にあたる部分にメッシュが生成されるのを防ぐことができる。
また、三角形の最大面積を制約条件として与えることで、過度に細長い三角形の生成を抑制し、テクスチャマッピングに適した形状を保つ。

得られた三角形メッシュに対しては、上方($+z$ 方向)から見たときに、すべての三角形の頂点が時計回りとなるように順序を統一する。
これは、法線ベクトルを鉛直上向きに揃えて、面の向きを統一するために必須の処理である。

\subsection{壁面の構成}
壁面の幾何構造は、床境界と天井境界の対応関係に基づいて生成される。
床境界および天井境界は同一の頂点数と順序を持つ閉ループであるため、
対応する上下の隣接頂点対を用いることで壁面を構成できる。

具体的には、床境界の隣接する頂点 $\bm{P}_i, \bm{P}_{i+1}$ と、
それらに対応する天井境界頂点 $\bm{P}_i^{\mathrm{ceil}}, \bm{P}_{i+1}^{\mathrm{ceil}}$ を結ぶことで、
一枚の壁面を表す四角形を定義する。
このようにして生成された床面および壁面の幾何構造により、屋内環境の簡易3次元ワイヤーフレームモデルが構築される。


\section{全方位画像を用いたテクスチャ取得の方針}
3次元モデルへのテクスチャ割り当てにおいては、モデル表面を十分に覆う視点から撮影された画像を効率的に取得することが重要である。 
一般的な透視投影カメラを用いる場合、多方向のテクスチャ情報を網羅するためには、カメラの向きを変えながら多数の画像を撮影する必要があり、撮影およびデータ管理に関わるコストが増大するという課題がある。

そこで本研究では、単一の撮影によって全周囲の視覚情報を取得可能な全方位カメラを用いる。
全方位画像は、カメラ位置を中心とする前後・左右・上下の全周囲の風景を一度に記録した画像である。
幾何学的には、カメラを中心とした球面上の情報として定義され、
データ形式としては、一般に正距円筒図法(Equirectangular Projection)などを用いて2次元平面画像に展開し保存されている。

この画像に対し、任意の視線方向を持つ仮想カメラを定義し、その画像平面へ画素を再投影することで、特定の方向に対応する透視投影画像を生成できる。
この特性を利用することで、実際に複数視点から撮影を行うことなく、任意の方向を向いた多数の画像群を計算機上で効率的に生成することが可能となる。

\section{透視投影画像変換}
本研究では、全方位画像を球面上の画素情報として扱う。 
生成する透視投影画像の各画素について、その視線方向に対応する全方位画像の画素値を取得することで、任意方向の画像を生成する。

正距円筒画像の幅および高さをそれぞれ $W_e, H_e$ とする。
全方位カメラの焦点距離 $f$ は全方位画像幅 $W_e$ を用いて次式で表される。

\begin{equation}
f = \frac{W_e}{2 \pi}
\end{equation}

生成する透視投影画像の幅および高さをそれぞれ $W_p, H_p$ とし、
透視投影画像上の画素座標を $(u_p, v_p)$ とすると、
この画素に対応する光軸方向を $z$ 軸とする3次元の視線ベクトル $\bm{x}$ は次式で定義される。
ただし、$N[ \cdot ]$はベクトルのノルムを1にする正規化する正規化作用素である。

\begin{equation}
\bm{x} = N[
\begin{pmatrix}
u_p - W_p/2 \\
v_p - H_p/2 \\
f
\end{pmatrix}
]
\end{equation}

カメラの視線方向を変更するための回転行列 $\bm{R}$ を定義する。
本研究では光軸周りの回転は考慮せず、水平方向の回転$\theta_{eye}$ および 垂直方向の回転$\phi_{eye}$ のみにより姿勢を決定する。
これに対応する回転行列 $\bm{R}(\theta_{eye}, \phi_{eye})$ は次式で与えられる。

\begin{equation}
\bm{R}(\theta_{eye}, \phi_{eye}) = 
\begin{pmatrix}
\cos \theta_{eye} & 0 & \sin \theta_{eye} \\
0 & 1 & 0 \\
-\sin \theta_{eye} & 0 & \cos \theta_{eye}
\end{pmatrix}
\begin{pmatrix}
1 & 0 & 0 \\
0 & \cos \phi_{eye} & -\sin \phi_{eye} \\
0 & \sin \phi_{eye} & \cos \phi_{eye}
\end{pmatrix}
\end{equation}

回転後の視線ベクトル $\bm{x}' = (X', Y', Z')^\top$ は、次式で計算される。

\begin{equation}
\bm{x}' = \bm{R}(\theta_{eye}, \phi_{eye}) \bm{x}
\end{equation}

得られた視線ベクトル $\bm{x}'=(X,Y,Z)$ を用いて、全方位画像の球面座標系 $(\theta_e, \phi_e)$ は次式で計算される。
ただし、$(X',Y',Z')^\top$は$(X,Y,Z)^\top$のノルムを1に正規化したものである。

\begin{equation}
\theta_e = \tan^{-1}\left(\frac{X'}{Z'}\right)
\end{equation}
\begin{equation}
\phi_e = \sin^{-1}\left(\frac{Y'}{\sqrt{X'^2 + Y'^2 + Z'^2}}\right)
\end{equation}

ここで、全方位画像(正距円筒画像)の幅を $W_e$、高さを $H_e$ とすると、
球面座標 $(\theta_e, \phi_e)$ に対応する全方位画像上の画素座標 $(u_e, v_e)$ は次式で与えられる。

\begin{equation}
u_e = \left( \theta_e + \pi \right) \frac{W_e}{2\pi}
\end{equation}
\begin{equation}
v_e = \left( \phi_e + \frac{\pi}{2} \right) \frac{H_e}{\pi}
\end{equation}

以上の手順により、全方位画像の画素 $(u_e, v_e)$ の輝度値を参照することで、
透視投影画像上の画素 $(u_p, v_p)$ の画素値を決定する。


\section{透視投影カメラによる全方位カメラ位置姿勢推定}
点特徴および線特徴を使ってカメラ位置を推定するため、直交射影に基づく点特徴と線特徴を用いたカメラ姿勢推定\cite{Sugaya2024}を行う。
これは、直交射影に基づく点の共線性誤差と線の共面性誤差に対して同一の定式化を行うことで点特徴と線特徴を同時に扱い、カメラ姿勢を推定する手法である。

直交射影の共線性と共面性に基づくカメラ姿勢推定に必要な入力は以下の通りである。
\begin{itemize}
  \item $\bm{p}_a$ : 世界座標系で表現された空間点の座標。透視投影画像に映らない点は除外する。
  \item $\bm{v}_a$ : $\bm{p}_a$ に対応する画像上の特徴点座標(カメラ座標系)。
  \item $\bm{d}_a$ : 世界座標系で表現された直線 $L_a$ の方向ベクトル。
  \item $\bm{r}_a$ : 世界座標系で表現された直線 $L_a$ 上の点の座標。
  \item $\bm{n}_a$ : $L_a$ に対応する画像上の直線の法線ベクトル(カメラ座標系)。
\end{itemize}

$\bm{v}_a$ は透視投影画像上で $\bm{p}_a$ に対応する画素を選択し、前節で定義したカメラモデルを用いてカメラ座標系上のベクトルとして表現する。
画像中心を原点とした画素座標 $(u', v')$ を以下のように求める。
\begin{equation}
u' = u_p - \frac{W_p}{2}, \quad v' = v_p - \frac{H_p}{2}
\end{equation}
これに焦点距離 $f$ を追加し、$\bm{v}_a = (u', v', f)^\top$ とする。

$\bm{d}_a$ は 3 次元モデルの頂点座標を用いて、以下の式で求める。
\begin{equation}
\bm{d}_i = \frac{\bm{p}_{i+1} - \bm{p}_i}{\|\bm{p}_{i+1} - \bm{p}_i\|}
\end{equation}

$\bm{n}_a$ は画像上の直線端点に対応する 2 つのベクトル $\bm{v}_i, \bm{v}_{i+1}$ から外積を計算して求める。
\begin{equation}
\bm{n}_i = \frac{\bm{v}_i \times \bm{v}_{i+1}}{\|\bm{v}_i \times \bm{v}_{i+1}\|}
\end{equation}

これらの入力を用いて、複数視線方向の透視投影画像に対する自己位置推定を反復的に行うことで、点特徴および線特徴に基づく目的関数を最小化する。
最終的に得られるカメラの回転行列 $\bm{R}$ と並進ベクトル $\bm{t}$ は、世界座標系の三次元点をカメラ座標系に変換するための変換行列である。





\section{メッシュの座標系変換と投影}

メッシュ頂点の世界座標
$\bm{v}_{\rm world}^{(i)} \in \mathbb{R}^3$($i=1,2,3$)を,
カメラの回転行列 $\bm{R}$ および並進ベクトル $\bm{t}$ を用いて,
次式によりカメラ座標系へ変換する。

\begin{equation}
\bm{v}_{\rm cam}^{(i)} = \bm{R}\bm{v}_{\rm world}^{(i)} + \bm{t}
\end{equation}

変換後の頂点座標を用いて,
メッシュの重心方向ベクトル $\bm{c}_{\rm cam}$ を次式で求める。

\begin{equation}
\bm{c}_{\rm cam} = \frac{1}{3} \sum_{i=1}^{3} \bm{v}_{\rm cam}^{(i)}
\end{equation}

本研究では,各メッシュの重心方向にカメラ視線を向けることで,
当該メッシュが透視投影画像の中心付近に投影されるよう制御する。
これにより,射影歪みを抑えた状態でテクスチャを取得することが可能となる。

この目的のため,メッシュ重心方向 $\bm{c}_{\rm cam}$ を視線方向とする
射影用回転行列 $\bm{R}_{\rm proj}$ を構成する。
具体的には,重心方向を前方軸とし,
上方向ベクトルとの外積により左右方向を定めることで,
右手系の直交基底を形成する。
得られた直交基底を列ベクトルとして並べることで,
射影用回転行列を定義する。

この回転行列を用いて,
カメラ座標系上の頂点を回転させる。

\begin{equation}
\bm{v}_{\rm proj}^{(i)} = \bm{R}_{\rm proj}^{\top} \bm{v}_{\rm cam}^{(i)}
\end{equation}

回転後の3次元点をカメラ内部パラメータ行列 $\bm{K}$ を用いて画像座標系へ投影する。
得られる画素座標が画像範囲内に収まらない場合には,画像中心からの最大距離に基づいて,
透視投影画像のサイズおよび投影スケールを調整する。
この処理により,メッシュが画像外に切り落とされることを防ぎ,単一の透視投影画像からメッシュ全体のテクスチャを確実に取得できる。

\subsection{テクスチャ候補の評価}
距離が遠いテクスチャや、正面方向から撮影していないテクスチャはテクスチャの視覚的品質劣化の原因となるため、テクスチャ割り当ての条件として、以下の物理量を計算する。

\begin{itemize}
  \item メッシュ重心までの距離
  \begin{equation}
  d = \|\bm{c}_{\rm cam}\|
  \end{equation}

  \item メッシュ重心方向ベクトル $\bm{c}_{\rm cam}$ とメッシュ法線ベクトル $\bm{n}$ のなす角
  \begin{equation}
    \begin{aligned}
    \bm{n} &= \frac{(\bm{v}_{\rm cam}^{(2)} - \bm{v}_{\rm cam}^{(1)}) \times (\bm{v}_{\rm cam}^{(3)} - \bm{v}_{\rm cam}^{(1)})}
    {\|(\bm{v}_{\rm cam}^{(2)} - \bm{v}_{\rm cam}^{(1)}) \times (\bm{v}_{\rm cam}^{(3)} - \bm{v}_{\rm cam}^{(1)})\|}
    \\
    \theta &= \arccos \left( \frac{\bm{c}_{\rm cam} \cdot \bm{n}}{\|\bm{c}_{\rm cam}\| \, \|\bm{n}\|} \right)
    \end{aligned}
  \end{equation}
\end{itemize}
これらの評価量を用いて、各メッシュに対して距離が近く、かつ正面に近い視点から取得された
テクスチャを優先的に選択することで、視覚的品質の高いテクスチャ割り当てを実現する。

\subsection{テクスチャの射影変換}

透視投影画像におけるメッシュの対応領域は、視点方向や撮影条件に起因する透視歪みを含み、
画像座標系上では任意形状の多角形となる。 
この歪んだ領域をそのまま使用すると、モデルの三次元形状との幾何学的な整合性が損なわれ、
適切なテクスチャマッピングが困難となる。

そこで本研究では、透視投影画像上のメッシュ領域に対して幾何変換を適用し、
正規化されたテクスチャ座標系上で定義される本来の形状へと変換する。 
本処理により透視歪みを除去し、テクスチャ画素と三次元メッシュ上の点との正しい対応関係を確立する。

変換手法はメッシュの形状に応じて決定する。 三角形メッシュにはアフィン変換を適用し、元の三角形形状へとマッピングを行う。
 一方、四角形メッシュにはホモグラフィ変換を適用し、正規化された長方形テクスチャを生成する。

最後に、変換後の画像に対してマスク処理を行い、メッシュ領域外の不要な画素を除去することで、
各メッシュ形状に適合したテクスチャ画像を抽出する。


\section{テクスチャの視覚的品質の改善}

複数の視点から取得されたテクスチャを同一メッシュに割り当てる場合、視点差や撮影条件の違いにより、
テクスチャ境界に不連続が生じ、四角的品質が悪化してしまう。
本研究では、この不連続を低減するため、隣接する2つのテクスチャに対してブレンド処理を行う。

各テクスチャには視線方向に基づく左右情報が付与されており、これを用いて左右関係を判定する。
左側テクスチャを $I_{\mathrm{L}}$、右側テクスチャを $I_{\mathrm{R}}$とする。

ブレンドは画像全体ではなく、画像幅 $W$ の中央付近に設定した、$x$ 方向の狭い領域に限定して適用する。
画素位置 $(x, y)$ におけるブレンド結果 $I(x, y)$ は、次式で与えられる。
\begin{equation}
I(x, y)
=
w_{\mathrm{L}}(x)\, I_{\mathrm{L}}(x, y)
+
w_{\mathrm{R}}(x)\, I_{\mathrm{R}}(x, y),
\end{equation}
ただし、
\begin{equation}
w_{\mathrm{L}}(x) + w_{\mathrm{R}}(x) = 1
\end{equation}
を満たす。

ブレンド領域内では,$x$ 方向に沿って連続的に変化する重みを用いるため、
シグモイド関数に基づき右側テクスチャの重み $w_{\mathrm{R}}(x)$ を次式で定義する。
\begin{equation}
w_{\mathrm{R}}(x)
=
\frac{1}{1 + \exp\!\left(-k\,\tilde{x}\right)},
\end{equation}
ここで,$\tilde{x} \in [-1,1]$ はブレンド領域内で正規化された水平方向位置を表し、
$k$ は重みの遷移の急峻さを制御するパラメータである。
左側テクスチャの重みは
\begin{equation}
w_{\mathrm{L}}(x) = 1 - w_{\mathrm{R}}(x)
\end{equation}
として与える。

ブレンド領域外では、一方のテクスチャのみを用いることで、テクスチャ全体の解像感および幾何的整合性を維持する。
このように、境界付近に限定した方向性を持つブレンドを行うことで、テクスチャ境界に生じる輝度不連続を効果的に低減できる。

