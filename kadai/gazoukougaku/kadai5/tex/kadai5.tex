\documentclass[]{jarticle}          % 一段組
%\documentclass[twocolumn]{jarticle} % 二段組

\textwidth 180mm
\textheight 255mm
\oddsidemargin -12mm
\topmargin -15mm
\columnsep 10mm

%\vspace{0.5cm} % 一段組の場合はコメントアウトした方が体裁がよいx
%] % 一段組の場合はコメントアウトする

\usepackage{styles/labheadings}
\usepackage[dvipdfmx]{graphicx,color}
\usepackage{amsmath,amssymb}
\usepackage{url}
% 追加
\usepackage[hang,small,bf]{caption}
\usepackage[subrefformat=parens]{subcaption}
\usepackage{indentfirst}
\captionsetup{compatibility=false}

\input{numerical_definition.tex}
% report.texと同じディレクトリにnumerical_definition.texを入れておけば上の書き方でもいいはずです

\usepackage[
  dvipdfm,
  bookmarks=true,
  bookmarksnumbered=true,
  colorlinks=true]{hyperref}
\AtBeginDvi{\special{pdf:tounicode EUC-UCS2}}

\pagestyle{labheadings}
\headerleft{課題5}   % ヘッダの左側のタイトル
\headerright{2024年6月13日}  % ヘッダの右側のタイトル

\begin{document}

%\twocolumn % 一段組の場合はコメントアウトする

\vspace*{2ex}
\begin{center}
 {\Large \bf 画像工学特論 課題5}\\ % タイトル
 \vspace*{5mm}
 {\large M1 田川幸汰}% 発表者名
\end{center}

%\vspace{0.5cm} % 一段組の場合はコメントアウトした方が体裁がよいx
%] % 一段組の場合はコメントアウトする

%新しく作成したコマンド
% \newcommand{\reffig}[1]{\hyperref[#1]{図\ref{#1}}}
% \newcommand{\refeq}[1]{\hyperref[#1]{式(\ref{#1})}}
% \newcommand{\reftab}[1]{\hyperref[#1]{表\ref{#1}}}
% \newcommand{\refsec}[1]{\hyperref[#1]{\ref{#1}章}}
% \newcommand{\refsubsec}[1]{\hyperref[#1]{\ref{#1}節}}

\section{課題1}
オリジナルの因子分解法において、画像に含まれる歪みはなぜ生じるのか、
またどのように生じるのか考察せよ。
\subsection{解答}
歪みが生じる原因として最も重要な点は、線形モデルでの投影を行っている点である。
透視投影で撮影された画像を正確に復元するためには、透視投影の特徴を考慮する必要がある。
具体的には、カメラからの距離に応じて物体が小さくなることや、カメラの位置姿勢の変化によって
透視投影の特性が変化することである。
今回の画像の復元では、透視投影特性が考慮されていないため、カメラからの距離が離れるほど
実際の物体の大きさとの誤差が大きくなり、歪みが発生する。
カメラからの距離に応じて物体が小さくなることに対処するためには、逆透視投影を行う必要がある。
また、カメラの位置姿勢の変化によって透視投影の特性が変化することに対処するためには、
カメラの位置姿勢、復元形状を推定する必要があり、バンドル調整や同時位置と姿勢推定(SLAM)などの手法が使われる。

\section{課題2(option)}
Structure from Motion and Multi-View Stereoのフ
リーな実装であるcolmapで、自分の撮影した複数の映像を使った
結果を考察と共に示せ。
\subsection{解答}
Linux環境で実行を試みたが、colmapをビルドする際のcuda関連のエラーが解決できず
実行することができなかった。
\end{document}
