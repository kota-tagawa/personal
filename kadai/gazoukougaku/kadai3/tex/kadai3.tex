\documentclass[]{jarticle}          % 一段組
%\documentclass[twocolumn]{jarticle} % 二段組

\textwidth 180mm
\textheight 255mm
\oddsidemargin -12mm
\topmargin -15mm
\columnsep 10mm

%\vspace{0.5cm} % 一段組の場合はコメントアウトした方が体裁がよいx
%] % 一段組の場合はコメントアウトする

\usepackage{styles/labheadings}
\usepackage[dvipdfmx]{graphicx,color}
\usepackage{amsmath,amssymb}
\usepackage{url}
% 追加
\usepackage{listings,jvlisting} 
\usepackage[hang,small,bf]{caption}
\usepackage[subrefformat=parens]{subcaption}
\usepackage{indentfirst}
\captionsetup{compatibility=false}

\input{numerical_definition.tex}
% report.texと同じディレクトリにnumerical_definition.texを入れておけば上の書き方でもいいはずです

\usepackage[
  dvipdfm,
  bookmarks=true,
  bookmarksnumbered=true,
  colorlinks=true]{hyperref}
\AtBeginDvi{\special{pdf:tounicode EUC-UCS2}}

%ここからソースコードの表示に関する設定
\lstset{
  basicstyle={\ttfamily},
  identifierstyle={\small},
  commentstyle={\smallitshape},
  keywordstyle={\small\bfseries},
  ndkeywordstyle={\small},
  stringstyle={\small\ttfamily},
  frame={tb},
  breaklines=true,
  columns=[l]{fullflexible},
  numbers=left,
  xrightmargin=0zw,
  xleftmargin=3zw,
  numberstyle={\scriptsize},
  stepnumber=1,
  numbersep=1zw,
  lineskip=-0.5ex
}
%ここまでソースコードの表示に関する設定

\pagestyle{labheadings}
\headerleft{課題3}   % ヘッダの左側のタイトル
\headerright{2024年5月15日}  % ヘッダの右側のタイトル

\begin{document}

%\twocolumn % 一段組の場合はコメントアウトする

\vspace*{2ex}
\begin{center}
 {\Large \bf 画像工学特論 課題3}\\ % タイトル
 \vspace*{5mm}
 {\large M1 田川幸汰}% 発表者名
\end{center}

%\vspace{0.5cm} % 一段組の場合はコメントアウトした方が体裁がよいx
%] % 一段組の場合はコメントアウトする

%新しく作成したコマンド
% \newcommand{\reffig}[1]{\hyperref[#1]{図\ref{#1}}}
% \newcommand{\refeq}[1]{\hyperref[#1]{式(\ref{#1})}}
% \newcommand{\reftab}[1]{\hyperref[#1]{表\ref{#1}}}
% \newcommand{\refsec}[1]{\hyperref[#1]{\ref{#1}章}}
% \newcommand{\refsubsec}[1]{\hyperref[#1]{\ref{#1}節}}

\section{課題}
$\sum^N_{i=1}(\boldsymbol{\xi}_i,\fU)^2\rightarrow min$の最小化の解が、行列$\MU=\sum^N_{i=1}(\boldsymbol{\xi}_i,\boldsymbol{\xi}_i^\top)^2$の最小固有値に対する
単位固有ベクトルで与えられることを示せ。
\subsection{解答}
左辺を展開し、内積の交換則を用いて順番を入れ替える。
\begin{gather}
  \sum^N_{i=1}(\boldsymbol{\xi}_i,\fU)^2=\sum^N_{i=1}(\fU,\boldsymbol{\xi}_i)(\boldsymbol{\xi}_i,\fU) \\
  =\sum^N_{i=1}\fU^\top\boldsymbol{\xi}_i,\boldsymbol{\xi}_i^\top\fU \\
  =\fU^\top(\sum^N_{i=1}\boldsymbol{\xi}_i,\boldsymbol{\xi}_i^\top)\fU \\
  =(\fU,(\sum^N_{i=1}\boldsymbol{\xi}_i,\boldsymbol{\xi}_i^\top)\fU) \\
  =(\fU,\MU\fU)
\end{gather}
ここで、$(\fU,\MU\fU)$はレイリー商を適用することで、式(6)の形に書き換えられる。
\begin{equation}
  R(\MU,\fU)=\frac{\fU^\top\MU\fU}{\fU^\top\fU}
\end{equation}
このとき、ベクトル$\fU$が行列$MU$の固有値に対応する固有ベクトルとすると、
$R(\MU,\fU)$の解は行列$\MU$の固有値となる。したがって、
$\sum^N_{i=1}(\boldsymbol{\xi}_i,\fU)^2$の最小化の解は、行列$\MU$の最小固有値に対応する
最小固有ベクトル$\fU$を与えることで算出できる。
\end{document}
