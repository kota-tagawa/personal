\documentclass[]{jarticle}          % 一段組
%\documentclass[twocolumn]{jarticle} % 二段組

\textwidth 180mm
\textheight 255mm
\oddsidemargin -12mm
\topmargin -15mm
\columnsep 10mm

%\vspace{0.5cm} % 一段組の場合はコメントアウトした方が体裁がよいx
%] % 一段組の場合はコメントアウトする

\usepackage{styles/labheadings}
\usepackage[dvipdfmx]{graphicx,color}
\usepackage{amsmath,amssymb}
\usepackage{url}
% 追加
\usepackage[hang,small,bf]{caption}
\usepackage[subrefformat=parens]{subcaption}
\captionsetup{compatibility=false}

\input{numerical_definition.tex}
% report.texと同じディレクトリにnumerical_definition.texを入れておけば上の書き方でもいいはずです

\usepackage[
  dvipdfm,
  bookmarks=true,
  bookmarksnumbered=true,
  colorlinks=true]{hyperref}
\AtBeginDvi{\special{pdf:tounicode EUC-UCS2}}

\pagestyle{labheadings}
\headerleft{第1回レポート課題}   % ヘッダの左側のタイトル
\headerright{2024年8月1日}  % ヘッダの右側のタイトル

\begin{document}

%\twocolumn % 一段組の場合はコメントアウトする

\vspace*{2ex}
\begin{center}
 {\Large \bf 2023年度 「 シミュレーション特論 」第2回レポート課題}\\ % タイトル
 \vspace*{5mm}
 {\large M1 田川幸汰}% 発表者名
\end{center}

%\vspace{0.5cm} % 一段組の場合はコメントアウトした方が体裁がよいx
%] % 一段組の場合はコメントアウトする

%新しく作成したコマンド
% \newcommand{\reffig}[1]{\hyperref[#1]{図\ref{#1}}}
% \newcommand{\refeq}[1]{\hyperref[#1]{式(\ref{#1})}}
% \newcommand{\reftab}[1]{\hyperref[#1]{表\ref{#1}}}
% \newcommand{\refsec}[1]{\hyperref[#1]{\ref{#1}章}}
% \newcommand{\refsubsec}[1]{\hyperref[#1]{\ref{#1}節}}

% 数式
%\begin{equation}
%  数式記述  
%  \label{ラベル名}
%\end{equation}

% 図
% \begin{figure}[!ht]
%   \begin{center}
%     \includegraphics[scale=0.5]{figures/画像ファイル名}
%     \caption{キャプション名}
%     \label{ラベル名}
%   \end{center}
% \end{figure}

% リスト
% \begin{enumerate or itemize}
%   \item 
% \end{enumerate or itemize}

\section{課題1}
連立一次方程式$\AU\xU=\bU$を考える。正方行列$\AU$とベクトル$\bU$が以下の通りに与えられたとき、この5元連立一次方程式をヤコビ法で解くプログラムを作成し、結果を示せ。
\begin{equation}
  \AU =
  \begin{pmatrix}
    3.0 & 1.0 & 0.0 & 0.0 & 0.0 \\
    1.0 & 2.0 & 1.0 & 0.0 & 0.0 \\
    0.0 & 1.0 & 1.0 & 1.0 & 0.0 \\
    0.0 & 0.0 & 1.0 & 4.0 & 1.0 \\
    0.0 & 0.0 & 0.0 & 1.0 & 5.0
  \end{pmatrix}
  \quad
  \bU = 
  \begin{pmatrix}
    7.0 \\
    9.0 \\
    9.0 \\
    21.0 \\
    23.0
  \end{pmatrix}
\end{equation}
\subsection{解答}
連立方程式をヤコビ法で解いた場合の解$\xU$と、繰り返し計算における$\xU$の初期値、
繰り返し計算の収束条件と収束回数を\hyperref[one]{表\ref{one}}に示す。
\begin{table}[ht!]
  \begin{center}
    \begin{tabular}{cccc}
      解$\xU$ & 初期値 & 収束条件 & 収束回数 \\
      (2,1,5,3,4) & (0,0,0,0,0) & $10^{-10}$ & 404 
    \end{tabular}
    \caption{ヤコビ法の結果}
    \label{one}
  \end{center}
\end{table}

\section{課題2}
課題1の連立一次方程式$\AU\xU=\bU$をガウス・ザイデル法で解くプログラムを完成し、結果を示せ。
\subsection{解答}
連立方程式をガウス・ザイデル法で解いた場合の解$\xU$と、繰り返し計算における$\xU$の初期値、
繰り返し計算の収束条件と収束回数を\hyperref[two]{表\ref{two}}に示す。
\begin{table}[ht!]
  \begin{center}
    \begin{tabular}{cccc}
      解$\xU$ & 初期値 & 収束条件 & 収束回数 \\
      (2,1,5,3,4) & (0,0,0,0,0) & $10^{-10}$ & 168
    \end{tabular}
    \caption{ガウス・ザイアル法の結果}
    \label{two}
  \end{center}
\end{table}

\section{課題3}
課題1の連立一次方程式$\AU\xU=\bU$を共役勾配法で解くプログラムを完成し、結果を示せ。
\subsection{解答}
連立方程式を共役勾配法で解いた場合の解$\xU$と、繰り返し計算における$\xU$の初期値、
繰り返し計算の収束条件と収束回数を\hyperref[three]{表\ref{three}}に示す。
\begin{table}[ht!]
  \begin{center}
    \begin{tabular}{cccc}
      解$\xU$ & 初期値 & 収束条件 & 収束回数 \\
      (2,1,5,3,4) & (0,0,0,0,0) & $10^{-10}$ & 4
    \end{tabular}
    \caption{共役勾配法の結果}
    \label{three}
  \end{center}
\end{table}

\section{考察}
3実験では、3つの解法を比べると収束回数は共役勾配法が最も少なく、次にガウス・ザイデル法、ヤコビ法となった。
共役勾配法は他の反復法と比べ非常に早く収束することが知られていて、理論的にはn次元の連立方程式に対してn回以内に収束する。
ガウス・ザイデル法、ヤコビ法は連立方程式を解く際の反復法であるが、ヤコビ法が各反復で古い値を使用するのに対し、
ガウス・ザイデル法は新しい値を使用するため、一般的に反復回数が小さくなる。
そのため、今回の実験の結果と等しくなり、実験は正しく行うことができたと考察する。
なお、今回はプログラムのコーディングの際にOpenAI社のChatGPT-4oを使用した

\end{document}
